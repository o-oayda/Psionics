\documentclass[
    10pt,
    twoside,
    twocolumn,
    openany,
    nodeprecatedcode,
    bg=full,
    justified%
    ]{dndbook}

\usepackage[english]{babel}
\usepackage[utf8]{inputenc}
\usepackage{hyperref}
\usepackage{rotating}
\newcommand{\secref}[1]{\ref{#1}. \nameref{#1}}
\newcommand{\define}[1]{\textbf{#1}}
\newcommand{\augment}[1]{\newline\indent\textbf{Augment.} #1}
\newcommand{\upcast}[1]{\newline\indent\textbf{At Higher Levels.} #1}
\newcommand{\spell}[1]{\textit{#1}}
\newcommand{\power}[1]{\textit{#1}}
\usepackage{lipsum}
\usepackage{adjustbox}
\usepackage{rotating}
\usepackage{amsmath}
\usepackage{epigraph}
\usetikzlibrary{intersections}
\renewcommand{\textflush}{flushepinormal}

\hypersetup{%
	colorlinks = true,%
	linkcolor =blue,%
	anchorcolor = red,%
	citecolor = blue,%
	urlcolor = blue%
}

\title{\Huge\scshape\fontsize{50}{40}\selectfont
    The Old Way \\
    \medskip
    \normalfont\scshape
    \Huge
    A Psionics Supplement\\
    \normalfont\Large\vspace{5mm}\texttt{Version 1.3.2}}
\author{DM Oayda}
% \usepackage[capitalise,nameinlink,noabbrev]{cleveref}
\setcounter{tocdepth}{2}
\setcounter{secnumdepth}{2}

\cftsetindents{subsection}{1em}{14mm} % toc spacing

% power costs variables
\newcommand\lvlone{1}
\newcommand\lvltwo{3}
\newcommand\lvlthree{5}
\newcommand\lvlfour{7}
\newcommand\lvlfive{9}
\newcommand\lvlsix{11}
\newcommand\lvlseven{13}
\newcommand\lvleight{15}
\newcommand\lvlnine{17}

\newcommand\pklvlnine{11}

\begin{document}

\frontmatter
\maketitle
\tableofcontents

\mainmatter%

\chapter{Introduction}
% TODO
% - finish powers
% - add spells interacting with psionics
% - add psionics interacting with spells
% - psionic feats
% - innate talents for certain races e.g. Gith

\chapter{Using Psionics}

\DndDropCapLine{T}{his chapter explains the nature of}
psionics, as well as the key rules behind
manifesting and maintaining psionic powers.
In \secref{sec:what_are_psionics},
a brief introduction to psionics is given
and their role in their game.
\secref{sec:manifesting_powers} explains how
psionic abilities---called powers---are produced
and the restrictions surrounding their use.
Lastly,
in \secref{sec:power_attributes},
the terminology used in power descriptions is explained.
The reader should familiarise themselves with each of these in turn
before they diving into
the list of powers in \secref{chap:list_of_powers}.

\section{What are Psionics?}
\label{sec:what_are_psionics}
Psionics tap into the latent ability of the mind
to master itself, its body and its environment.
Whereas spellcasters manipulate the magical essence
innate to the cosmological order,
users of psionics rely on their mind's power alone.
For many creatures,
manifesting any psionic power,
let alone mastering a psionic discipline,
takes years, even decades of rigorous mental training.
That being said,
some creatures have a natural propensity for psionics,
for example the children of Gith, illithids and aboleths.

Many refer to psionics as \emph{the Old Way}.
This is because there is a school of thought that
the seeds of all creatures---the progenitor beings
which gave rise to existence itself---were
masters of the mind.
According to this doctrine,
magic and spellcraft were created by these
beings when the reality was moulded to as it is today.
To do this, they used psionics.

\subsection{Types of Psionic Abilities}
There are two types of psionic abilities:
\define{powers} and \define{innate talents}.

\subsubsection{Powers}
Powers are what most psionics users will be
interacting with.
The rules around powers are described
later in this chapter.

\subsubsection{Innate Talents}
A second type of psionic ability is an innate talent.
Innate talents are generally possessed by NPCs or
monsters which have some psionic ability.
Innate talents are produced on a per day basis;
that is, a creature can manifest a certain innate talent
a number of times per day.
Innate talents can be used as a convenience,
representing an NPCs psionic ability but
foregoing the need to keep track of mental bandwidth,
psionic thresholds and psi dice.

\begin{DndSidebar}[float=htbp]{Interaction with Magic}
    \tocside{Interaction with Magic}
    Since psionics and magic are disparate entities,
    how do they interact together?
    A creature that manifests a psionic power is not
    casting a spell,
    and therefore any ability which negates the effects
    of magic cannot negate the power.
    For example, psionic powers cannot be affected by
    \spell{counterspell} or \spell{dispel magic}.

    However, certain psionic powers are explicitly stated
    to interact with magic.
    In addition, this supplement adds spells to the existing
    grimoire which explicitly interact with psionics.
\end{DndSidebar}

\section{Manifesting Powers}
\label{sec:manifesting_powers}
In order to manifest a psionic power,
the \define{power level} must be equal to or below
the manifesting character's \define{psionic threshold}.
The psionic threshold is determined by the number of levels
a character has in classes with psionics.
If the power satisfies this test,
it can be manifested if they have sufficient \define{psi dice}
to expend.
The manifesting character will also need to note their
current \define{mental bandwidth} and whether or not manifesting
the power will exceed this bandwidth.
Each of these concepts are explained below.

\subsection{Psionic Threshold}
The psionic threshold represents the strongest powers
that a psionics user can learn and manifest.
The psionic threshold is determined by a character's class
and level,
as shown in the class tables of \secref{chap:classes}.
When levelling,
a character can only learn powers with a level up to
their psionic threshold.
This is explained in more detail in the same chapter.

Each power has a level written in its description.
If a character attempts to manifest a power with level
greater than their psionic threshold,
the power fails to manifest,
however no cost is incurred in doing so,
see \secref{sub:psi_dice}.

\subsection{Mental Bandwidth}
Manifesting powers strains the mind and the body.
Mental bandwidth represents the amount of psionic strain that
a character can endure while using psionics,
and is determined by the character's psionic class and level,
see \secref{chap:classes}.
Mental bandwidth is divided into slots,
and each power takes up a different number of slots.

When a character manifests a power,
they note their total mental bandwidth
as determined by their class and level.
This is called the
\define{maximum mental bandwidth}.
They then subtract from the maximum mental bandwidth
any slots which are taken up by powers they are already concentrating on,
see \secref{sub:concentration}.
The number determined after this subtraction is called the
\define{effective mental bandwidth}.
If the power to be manifested has a mental bandwidth
greater than the effective mental bandwidth,
the character must make a Constitution saving throw with
DC equal to 10 plus twice the number of slots
in excess of their effective bandwidth.

For example,
suppose Kadoth is a 10th level psion
and therefore has a maximum mental bandwidth of 6.
They are already concentrating on \power{adapt body},
which takes up 4 mental bandwidth slots,
so their effective mental bandwidth is 2.
They then attempt to manifest the mind slam power,
which takes up 3 slots.
Since this puts them in excess of their effective bandwidth by 1,
they make a DC 12 Constitution saving throw ($10 + 2 \times 1)$.

\subsubsection{Becoming Strained}
If a character fails this Constitution saving throw,
the power fails to manifest.
In addition, the manifester gains the \textbf{strained} condition.
The severity of the condition depends on the number of slots
that were in excess of the available bandwidth when the attempt
to manifest the power was made.
This is shown in the table below.
In Kadoth's example,
they were only in excess by 1,
and therefore would only incur the malus shown in the
first row of the table.
Note that the maluses are cumulative.
For example,
with two slots in excess,
a character incurs the malus from being in excess by 1
as well as the malus for being in excess by 2.

\begin{table}[htbp]%
    \begin{DndTable}[width=\columnwidth,
                     header=Strained Condition]{
                     X X}
        Slots in Excess & Cumulative Maluses \\
        1  & AC bonus from Dexterity reduced to 0 \\
        2  & Gain level of exhaustion \\
        3  & MB reduced by 1 \\
        4  & MB reduced by 1 \\
        5+ & Gain level of exhaustion
    \end{DndTable}
\end{table}

\subsubsection{Recovering from Strain}
A character loses the strained condition,
including all of the accumulated maluses,
when they finish a long rest.

In addition, during a short rest,
a character may spend hit dice to remove
cumulative levels of strain on a 1 to 1 basis.
For example,
if a character was three slots in excess of their
effective mental bandwidth when they attempted
to manifest a power,
and as a result they have accrued the first three
rows of maluses in the table above,
they can spend one hit die to remove the malus
from the third row,
another to remove the malus from the second row,
and a final one to remove the malus from the first row.
In this latter case, the strained condition
is removed altogether. 

\subsection{Psi Dice}
\label{sub:psi_dice}
When manifesting a power,
a character must expend a number of psi dice
equal to the number stated in the power's description.
The number of psi dice a character has is determined
by their class and level, see \secref{chap:classes}.
If a character has less psi dice than the total psi dice cost
of the power, the power fails (but no dice are expended).

The total base psi dice cost of a power is determined by the its level.
The cost is equal to one less than double the power's level.
This is shown in the table below.
For example, a 1st level power incurs a cost of 1 psi die,
whereas a 5-th level power incurs a cost of 9 psi dice.
\begin{table*}[htbp]%
    \begin{DndTable}[width=\textwidth,
                     header=Psi Dice Cost by Level]{
                     X X X X X X X X X X}
         Level         & 1 & 2 & 3 & 4 & 5 & 6  & 7  & 8  & 9 \\
        \textbf{Cost}  & 1 & 3 & 5 & 7 & 9 & 11 & 13 & 15 & 17
    \end{DndTable}
\end{table*}

Some powers can be \define{augmented} with extra psi dice, in which
case they usually have a stronger effect, see \secref{sub:augmenting}.
If a power can be augmented,
this will be explicitly stated in its description.
The cost for augmenting a power is also given in its description.

\subsection{Concentration}
\label{sub:concentration}
If stipulated in the power description,
a power requires uninterrupted concentration
in order to be continually used.
There is no limit to the number of powers that may be
simultaneously concentrated on,
however each power concentrated on will take away from the
available mental bandwidth slots, as mentioned above.

Whenever a manifester takes damage while
concentrating on any number of powers,
they must make a Constitution saving throw.
The DC of the save is either equal to 10 or half the damage taken,
whichever is higher.
Each separate source of damage prompts a separate save.
If the save fails,
the manifester loses concentration on all the powers they were
concentrating on.
In addition, the DM may call the manifester to make a
save with any DC in a circumstance that would
jeopardise the manifester's concentration,
like being thrown off a flying broom or
being battered by an intense blizzard.

A manifester may at any time on their turn
stop concentrating on any power.
However, a manifester will automatically stop concentrating on a power
if they become incapacitated or if they die.

\section{Power Attributes}
\label{sec:power_attributes}
In \secref{chap:list_of_powers},
the available powers are given.
In this section,
the terminology and structure of the power descriptions are explained.

\subsection{Level and Discipline}
Beneath the name of the power,
the power level and discipline are given.
Each power is associated with a discipline,
of which there are six.
These are described below.

\subsubsection{Prescience}
Strictly, the discipline of prescience relates to psionic powers
which allow the manifester to peer through time and see future events.
Prescient manifesters describe the future as
an undulating cloth blowing in a swift breeze,
with the hills and valleys of its surface representing
future possibilities. 

However, in more recent times,
the discipline of prescience has incorporated additional tenets,
and its adherents are also characterised by an uncanny ability to
understand the past and know things not normally known to
the untrained mind.

\subsubsection{Prana Bindu}
The discipline of Prana Bindu stresses that the mind and body
must work in close concert with each other.
Adherents have a profound mastery of nerve and muscle,
and can react to situations with speeds nearly imperceptible
to the untrained eye.
The discipline also allows one to alter their internal equilibrium
and metabolism at will,
granting the body the ability to heal and adapt to hostile environments.

The discipline of Prana Bindu is especially favoured by Psi Knights,
powerful warriors who fuse martial and mental prowess.
See \secref{sec:psi_knight} for details on the Psi Knight class.

\subsubsection{Voice}
The Voice is a dangerous but supremely powerful tool
used to dominate the minds and wills of others.
Adherents hone their speech in such a way that,
only by subtle changes in pitch, volume and speed,
their Voice can control the behaviour of others.

In addition, adherents to the discipline can manifest their Voice
in the minds of others, allowing them to telepathically
communicate with other creatures.

\subsubsection{Metacreativity}
The discipline of metacreativity allows the manifester to mould
matter as they desire,
turning the mundane into useful tools or deadly weapons.
A metacreator can also in this manner craft constructs
which bend to their mind,
allowing myriad servants to carry out their bidding.

\subsubsection{Spacefolding}
A spacefolder warps the fabric of space and time to
transport themselves and others.
This also allows them to move effortlessly in conditions
that would otherwise impede or slow down movement. 
Many powerful spacefolders can even cross the gap between
the planes themselves.

\subsubsection{Psychokinesis}
The discipline of psychokinesis allows one to transform
their raw mental energy into destructive power,
unleashing havoc on the battlefield.
Psychokineticists can deal terrifying amounts of damage
and leave their enemies trembling in awe and fear.

\subsection{Manifesting Time}
Each power has a \define{manifesting time},
which may be a bonus action, action, or reaction.
Unlike spellcasting,
there is no level restriction to manifesting powers
on the same turn where one has been manifested
with a bonus action. 

In addition, some powers may have longer casting times,
in which case on each round the manifester must use their action
to manifest the power while maintaining concentration.
While the manifester is maintaining concentration in this way,
the power will take up the specified number
of mental bandwidth slots given in the power's description.
If their concentration is broken, the power fails,
but no psi dice are expended,
and the mental bandwidth slots are freed up.

\subsection{Range}
A power's \define{range} indicates the extent
to which its effects can reach when manifested.
The \define{target} (see below) of the power must be within range.

There are three types of ranges: self, touch and a range
specified in feet.
Powers with a range of self will always target the manifester.

\subsubsection{Targets}
All powers have a target or targets.
A power's target is typically mentioned in the power description.
The description will also specify whether or not the target
of a power needs to be seen.
Thus, not all powers require sight of the target.
However, in addition to the range requirement mentioned above,
the manifester \textit{must} have \define{line of effect}
towards the target.
This means that the manifester needs to be be able to draw a line
to the target without that line passing through an obstacle,
even if the target can be seen through that obstacle.
Thus, a target cannot be behind total cover.
\emph{Exception:} some powers will explicitly state that
the manifester need not have line of effect towards the target.

Powers with a range of self or touch
are not bound by the line of effect requirement.

\subsection{Cost}
The \define{cost} specifies the mental bandwidth and psi die
that must be expended to manifest the power.
For example,
`MB 1, PD 1'
means that the power takes up 1 slot of mental bandwidth
and costs one psi die to manifest.

\subsection{Duration}
The \define{duration} of a power is how long the effects
of a power remain active.
Powers can be instantaneous,
while others can last for any length of time
as specified in the powers description.

If the power requires concentration throughout its duration,
this will be indicated alongside the duration.

\subsection{Areas of Effect}
Some powers specify in their description an area of effect
over which the power manifests.
In this case, the rules governing these areas of effect
are identical to those of spellcasting, which begin at
phb 202.

\subsection{Affecting Targets}
Some powers specify that the manifester makes an attack
roll against each target,
whereas some specify that all targets need to make
a type of saving throw.

If the power specifies an attack roll,
then the manifester rolls a d20
and adds their \define{psionic attack modifier},
which is their \define{psionic ability modifier} (determined by class)
plus their proficiency bonus.

If the power specifies a saving throw,
then each target must succeed on a saving throw
with DC equal to 8 + the manifester's psionic attack modifier.
The type of saving throw
is specified in the power's description.

\subsection{Augmenting Powers}
\label{sub:augmenting}
Some powers will indicate that they can be \define{augmented},
which is analogous to upcasting for a spellcaster.
When augmenting a power,
the manifester spends additional psi dice on top of the cost
incurred from the level of the power itself
(see \secref{sub:psi_dice} for the base cost by power level).

The cost required to augment a power will be specified
in the power's description.
For example, the description might mention
`For every two additional psi psi dice you spend\dots'
In this case, the cost to augment the power is 2 psi dice.
Each time the power is augmented,
its level increases by 1.
In the example above,
spending an additional 6 psi dice increases the power's level by 3.

A power cannot be augmented to a level beyond the manifester's
psionic threshold.
For example,
if a manifester has a psionic threshold of 3,
they may spend 4 psi dice to augment a power from
1st to 3rd level with an augment cost of 2,
but may not go beyond 3rd level i.e. spend beyond 4 psi dice.

\subsection{Components}
Unlike spells,
powers generally do not have any components.
When a power is manifested,
quite often there are scarcely any hints of its existence,
save for the actual effect or effects it produces.
However, there are two key exceptions to this rule.

\subsubsection{Voice Powers}
Powers belonging to the Voice discipline sometimes involve
the manifester altering the frequencies, pitch and
speed of their natural voice such that it compels
creatures to act in certain ways.
Whenever such a power makes reference to the manifester
using their Voice,
the power's manifestation is distinctly audible.
It can be heard 2d6 times 10 feet away with normal
levels of background noise,
half as far away in loud conditions,
and twice as far away in quiet conditions.

\subsubsection{Metacreativity Powers}
All powers of the Metacreativity discipline require
\define{mundane matter}, which is moulded by the manifester
as dependent on the power.
Mundane matter refers to common substances on the
Material Plane,
like soil, plant matter, rock, metal,
glass, and so on.
The matter must be in the solid phase,
and not the liquid, gas or plasma phase.
It does not include matter imbued with magic
or some other arcane effect,
unless that matter is the raw essence of Limbo,
which can be warped with one's mind.

Thus,
if the manifester does not have access to mundane matter
when they manifest a Metacreativity power,
the power fails (but no cost is incurred in the attempt).

\chapter{List of Powers}
\label{chap:list_of_powers}
\DndDropCapLine{I}{n this chapter, the list of available}
powers are given.
\secref{sec:class_lists} breaks down the powers into those
available by class,
and \secref{sub:discipline_sublists} gives the powers
available to psions who have dedicated themselves to a
particular discipline at 1st level.
Lastly,
\secref{sec:power_descriptions} sets out the powers in full.


\section{Class Lists}
\label{sec:class_lists}
\DndSetThemeColor[DmgLavender]
\subsection{Psion}
\begin{dndlongtable}[
    % X X X]
    >{\raggedright\arraybackslash}p{0.22\linewidth} c p{0.55\linewidth}]
    Name & Level & Short Description \\
\nameref{pwr:attraction} & 1 & Incept an item of attraction in a creature \\
\nameref{pwr:blow_up_object_with_mind} & 1 & Blow up object, damaging those nearby \\
\nameref{pwr:conceal_thoughts} & 1 & Conceal motives of target \\
\nameref{pwr:create_sound} & 1 & Create unique sound \\
\nameref{pwr:cushion_fall} & 1 & Reduce fall damage \\
\nameref{pwr:deceleration} & 1 & Reduce target's speed \\
\nameref{pwr:demoralise} & 1 & Enemies suffer malus to attacks and saving throws \\
\nameref{pwr:detect_psionics} & 1 & Detect nearby presence of psionic powers and effects \\
\nameref{pwr:diamond_bullet} & 1 & Fire diamond bullet at target \\
\nameref{pwr:dissipating_touch} & 1 & Deal force damage to target on touch \\
\nameref{pwr:distract} & 1 & Give target disadvantage to perception and insight \\
\nameref{pwr:déjà_vu} & 1 & Make target repeat previous actions \\
\nameref{pwr:empaths_understanding} & 1 & Detect target's emotions to gain bonus to social checks \\
\nameref{pwr:energy_ray} & 1 & Strike target with ray of energy of chosen type \\
\nameref{pwr:float} & 1 & Float in water and gain swim speed \\
\nameref{pwr:force_screen} & 1 & Gain bonus to AC \\
\nameref{pwr:hammer} & 1 & Deal 1d12 with unarmed strikes \\
\nameref{pwr:identify_psionics} & 1 & Identify psionic powers and items \\
\nameref{pwr:inertial_armour} & 1 & Gain bonus to AC \\
\nameref{pwr:matter_agitation} & 1 & Heat matter \\
\nameref{pwr:mind_thrust} & 1 & Deal psychic damage to target \\
\nameref{pwr:missive} & 1 & Send telepathic message to creature \\
\nameref{pwr:orient_self} & 1 & Learn where you are \\
\nameref{pwr:precognition} & 1 & Gain slight bonus to defence or offence \\
\nameref{pwr:psi_hand} & 1 & Move small objects at short distance \\
\nameref{pwr:recall} & 1 & Repeat check to recall details \\
\nameref{pwr:sense_link} & 1 & Perceive through one of the target's senses \\
\nameref{pwr:shockwave} & 1 & Pummel target behind cover with force damage \\
\nameref{pwr:skate} & 1 & Skate along ground and move faster \\
\nameref{pwr:stillness_of_mind} & 1 & Gain bonus to wisdom save as reaction \\
\nameref{pwr:vigour} & 1 & Gain temporary hit points \\
\nameref{pwr:bestow_power} & 2 & Grant a creature psi dice \\
\nameref{pwr:biofeedback} & 2 & Reduce incoming damage \\
\nameref{pwr:body_equilibrium} & 2 & Walk on otherwise untraversable surfaces \\
\nameref{pwr:cloud_mind} & 2 & Erase yourself from target's mind \\
\nameref{pwr:concealing_membrane} & 2 & Become lightly obscured in membrane \\
\nameref{pwr:control_sound} & 2 & Mould existing sound as you desire \\
\nameref{pwr:detect_hostile_intent} & 2 & Detect any nearby creatures that are hostile to you \\
\nameref{pwr:energy_push} & 2 & Push target back with wave of energy of chosen type \\
\nameref{pwr:hyperphantasm} & 2 & Create an illusory image in the mind of another \\
\nameref{pwr:inflict_pain} & 2 & Target has disadvantage on attack rolls and skill checks \\
\nameref{pwr:levitate} & 2 & Levitate yourself \\
\nameref{pwr:psi_vision} & 2 & Gain darkvision \\
\nameref{pwr:psionic_step} & 2 & Teleport a short distance \\
\nameref{pwr:psychic_beacon} & 2 & Know distance and location of marked target \\
\nameref{pwr:reveal_agony} & 2 & Deal psychic damage to target \\
\nameref{pwr:sense_link_forced} & 2 & Forcibly perceive through one of the target's senses \\
\nameref{pwr:stunning_wave} & 2 & Daze creatures surrounding you \\
\nameref{pwr:swarm_of_crystals} & 2 & Slice targets with cone of crystals which always hit \\
\nameref{pwr:confusion_psionic} & 3 & Target behaves uncontrollably \\
\nameref{pwr:counterpower} & 3 & Interrupt the manifesting of a power \\
\nameref{pwr:danger_sense} & 3 & Reduce susceptibility to traps \\
\nameref{pwr:dispel_psionics} & 3 & Cancel psionic powers and effects \\
\nameref{pwr:energy_bolt} & 3 & Strike targets with line of energy of chosen type \\
\nameref{pwr:energy_burst} & 3 & Produce burst of energy of chosen type centred on yourself \\
\nameref{pwr:manipulate_object_with_mind} & 3 & Move object telekinetically \\
\nameref{pwr:rapid_healing} & 3 & Heal a small number of hit points \\
\nameref{pwr:sacrificial_shell} & 3 & Gain immediate protection from energy \\
\nameref{pwr:self_alchemy} & 3 & Alter your physical appearance \\
\nameref{pwr:share_pain} & 3 & Half of the damage you suffer is carried to the target \\
\nameref{pwr:time_warp} & 3 & Move target briefly forward in time \\
\nameref{pwr:touchsight} & 3 & Perceive surroundings by touch \\
\nameref{pwr:correspond} & 4 & Communicate with a creature that is anywhere \\
\nameref{pwr:detect_prying_eyes} & 4 & Learn who is attempting to scry you \\
\nameref{pwr:drain_power} & 4 & Drain targets psi dice \\
\nameref{pwr:empathic_feedback} & 4 & Target takes damage when they hit you \\
\nameref{pwr:manipulate_creature_with_mind} & 4 & Shove, disarm or grapple creature telekinetically \\
\nameref{pwr:mind_palace_power} & 4 & Shelter allies from psionic powers \\
\nameref{pwr:psi_door} & 4 & Teleport over a moderate distance \\
\nameref{pwr:psi_wall} & 4 & Create large protective barrier \\
\nameref{pwr:trace_teleport} & 4 & Learn location targets teleported to \\
\nameref{pwr:adapt_body} & 5 & Adapt your body to dangerous environments \\
\nameref{pwr:alter_ego} & 5 & Creating a copy of yourself \\
\nameref{pwr:create_object_with_mind} & 5 & Create object up to certain size \\
\nameref{pwr:plane_shift_psionic} & 5 & Move targets across the planes \\
\nameref{pwr:psychic_crush} & 5 & Deal large amount of psychic damage to creature \\
\nameref{pwr:shatter_mind_blank} & 5 & Cancels the effects of mind blank \\
\nameref{pwr:third_eye} & 5 & Gain truesight out to 120 feet \\
\nameref{pwr:tower_of_power} & 5 & Grant you and allies significant resistance to Voice powers \\
\nameref{pwr:breath_of_numinex} & 6 & Powerful breath weapon dealing acid damage \\
\nameref{pwr:cloud_mind_mass} & 6 & Erase yourself from many targets' mind \\
\nameref{pwr:mind_blank_psionic} & 6 & Conceal targets mind from external forces \\
\nameref{pwr:power_pirate} & 6 & Take control of another manifester's active power \\
\nameref{pwr:remote_viewing_trap} & 6 & Deal damage to those who attempt to scry you \\
\nameref{pwr:retrieve} & 6 & Teleport item into your hand \\
\nameref{pwr:divert_teleport} & 7 & Change destination of target's teleportation \\
\nameref{pwr:energy_transformation} & 7 & Convert energy damage into jets of plasma \\
\nameref{pwr:energy_wave} & 7 & Destructive cone of chosen energy type \\
\nameref{pwr:timeless_body} & 9 & Become impervious to all effects
\end{dndlongtable}

\subsection{Psion Discipline Sublists}
\label{sub:discipline_sublists}
\DndSetThemeColor[PhbLightCyan]
\begin{DndTable}[header=Prescience Sublist, bold=false]{
    >{\raggedright\arraybackslash}p{0.22\linewidth} c p{0.55\linewidth}}
    Name & Level & Short Description \\
\nameref{pwr:destiny_dissonance} & 1 & Inflict psychic damage and the poisoned condition with your touch \\
\nameref{pwr:distant_gaze} & 2 & See distant locations known or proximate to you \\
\nameref{pwr:object_reading} & 2 & Learn details about an object's former owners \\
\nameref{pwr:psychic_archaeology} & 2 & Glean information from the psychic impressions left in a locale \\
\nameref{pwr:possible_futures} & 5 & Gain the ability to reroll once per round \\
\nameref{pwr:scrying_psionic} & 5 & Remotely perceive a target through an ethereal form of yourself \\
\nameref{pwr:sublime_insight} & 9 & Learn uncannily accurate details of a creature you have seen before
\end{DndTable}

\DndSetThemeColor[PhbMauve]
\begin{DndTable}[header=Prana Bindu Sublist, bold=false]{
    >{\raggedright\arraybackslash}p{0.22\linewidth} c p{0.55\linewidth}}
    Name & Level & Short Description \\
Chameleon & 1 & Blend in with environment, bonus to stealth \\
Thicken Skin & 1 & Gain bonus to AC \\
Swiftsteed & 2 & Double movement speed \\
Psychofeedback & 5 & Alter ability scores temporarily \\
Restore Extremity & 5 & Restore lost extremity to creature you touch\ \\
Fusion & 8 & Fuse yourself and another creature into one entity temporarily
\end{DndTable}

\DndSetThemeColor[PhbTan]
\begin{DndTable}[header=Voice Sublist, bold=false]{
    >{\raggedright\arraybackslash}p{0.22\linewidth} c p{0.55\linewidth}}
    Name & Level & Short Description \\
\nameref{pwr:aversion} & 2 & Plant an object of aversion in the mind of a creature \\
\nameref{pwr:detect-thoughts-psionic} & 2 & Read creature's surface level thoughts or probe deeper into their psyche \\
\nameref{pwr:crisis-of-breath} & 3 & Disrupt a creature's breathing \\
\nameref{pwr:resuscitation} & 3 & Return creature that has just died to life \\
\nameref{pwr:compel} & 4 & Compel a creature to pursue a course of action
\end{DndTable}

\DndSetThemeColor[DmgSlateGrey]
\begin{DndTable}[header=Metacreativity Sublist, bold=false]{
    >{\raggedright\arraybackslash}p{0.22\linewidth} c p{0.55\linewidth}}
    Name & Level & Short Description \\
\nameref{pwr:psionic-golem} & 1 & Create a psionic golem that you control \\
\nameref{pwr:psionic-golem-repair} & 2 & Repair damage to a psionic golem
\end{DndTable}

\DndSetThemeColor[DmgLilac]
\begin{DndTable}[header=Spacefolding Sublist, bold=false]{
    >{\raggedright\arraybackslash}p{0.22\linewidth} c p{0.55\linewidth}}
    Name & Level & Short Description \\
\nameref{pwr:burst} & 1 & Gain speed boost as free action \\
\nameref{pwr:dimension_swap} & 2 & Swap positions between yourself and ally or two allies \\
\nameref{pwr:banishment_psionic} & 4 & Return a creature to their native plane of existence or a demiplane \\
\nameref{pwr:teleport_psionic} & 6 & Teleport to a destination on the same plane of existence as you
\end{DndTable}

\DndSetThemeColor[PhbLightGreen]
\begin{DndTable}[header=Psychokinesis Sublist, bold=false]{
    >{\raggedright\arraybackslash}p{0.22\linewidth} c p{0.55\linewidth}}
    Name & Level & Short Description \\
\nameref{pwr:control_air} & 2 & Alter the wind and potentially knock creatures prone \\
\nameref{pwr:energy_missile} & 2 & Hurl missiles of energy at up to 5 distinct targets \\
\nameref{pwr:control_body} & 4 & Rudimentarily take control a creature \\
\nameref{pwr:tornado} & 9 & Create an immense vortex of air dealing great damage and knocking enemies about
\end{DndTable}

\DndSetThemeColor[DmgCoral]
\subsection{Psi Knight}
\begin{dndlongtable}[
    >{\raggedright\arraybackslash}p{0.22\linewidth} c p{0.55\linewidth}]
    Name & Level & Short Description \\
\nameref{pwr:bite-of-the-wolf} & 1 & Gain bite attack \\
\nameref{pwr:blow-up-object-with-mind} & 1 & Blow up object, damaging those nearby \\
\nameref{pwr:bolt} & 1 & Create ammunition \\
\nameref{pwr:boots-of-the-demon-king} & 1 & Kick target with lightning, can knock prone or send target flying \\
\nameref{pwr:burst} & 1 & Gain speed boost as free action \\
\nameref{pwr:chameleon} & 1 & Blend in with environment, bonus to stealth \\
\nameref{pwr:conceal-thoughts} & 1 & Conceal motives of target \\
\nameref{pwr:create-armour-with-mind} & 1 & Create any armour from mundane matter \\
\nameref{pwr:create-weapon-with-mind} & 1 & Create any weapon from mundane matter \\
\nameref{pwr:cushion-fall} & 1 & Reduce fall damage \\
\nameref{pwr:detect-psionics} & 1 & Detect nearby presence of psionic powers and effects \\
\nameref{pwr:dissipating-touch} & 1 & Deal force damage to target on touch \\
\nameref{pwr:distract} & 1 & Give target disadvantage to perception and insight \\
\nameref{pwr:envenom} & 1 & Weapon is coated with corrosive bile \\
\nameref{pwr:flame-choke} & 1 & Grapple a creature and deal fire damage over time \\
\nameref{pwr:float} & 1 & Float in water and gain swim speed \\
\nameref{pwr:force-screen} & 1 & Gain bonus to AC \\
\nameref{pwr:grip-of-iron} & 1 & Gain bonus to grapple checks \\
\nameref{pwr:hammer} & 1 & Deal 1d12 with unarmed strikes \\
\nameref{pwr:inertial-armour} & 1 & Gain bonus to AC \\
\nameref{pwr:precognition} & 1 & Gain slight bonus to defence or offence \\
\nameref{pwr:shockwave} & 1 & Pummel target behind cover with force damage \\
\nameref{pwr:skate} & 1 & Skate along ground and move faster \\
\nameref{pwr:stillness-of-mind} & 1 & Gain bonus to wisdom save as reaction \\
\nameref{pwr:thicken-skin} & 1 & Gain bonus to AC \\
\nameref{pwr:vigour} & 1 & Gain temporary hit points \\
\nameref{pwr:biofeedback} & 2 & Reduce incoming damage \\
\nameref{pwr:body-equilibrium} & 2 & Walk on otherwise untraversable surfaces \\
\nameref{pwr:concealing-membrane} & 2 & Become lightly obscured in membrane \\
\nameref{pwr:detect-hostile-intent} & 2 & Detect any nearby creatures that are hostile to you \\
\nameref{pwr:dimension-swap} & 2 & Swap positions between yourself and ally or two allies \\
\nameref{pwr:dissolving-touch} & 2 & Deal acid damage to creature you hit with attack \\
\nameref{pwr:kinaesthetic-disruption} & 2 & Invert a creature's sense of direction \\
\nameref{pwr:leap} & 2 & Gain fly speed on your turn \\
\nameref{pwr:levitate} & 2 & Levitate yourself \\
\nameref{pwr:psi-vision} & 2 & Gain darkvision \\
\nameref{pwr:psionic-step} & 2 & Teleport a short distance \\
\nameref{pwr:spacetime-redirection} & 2 & Redirect an enemy's attack to another creature \\
\nameref{pwr:swiftsteed} & 2 & Double movement speed \\
\nameref{pwr:trap-clairvoyance} & 2 & Forsee danger from hidden traps \\
\nameref{pwr:tunnel-vision} & 2 & Restrict vision of a creature \\
\nameref{pwr:uncanny-reach} & 2 & Increase your reach with unarmed attacks \\
\nameref{pwr:wall-walker} & 2 & Walk on walls and ceilings \\
\nameref{pwr:danger-sense} & 3 & Reduce susceptibility to traps \\
\nameref{pwr:fist-of-fury} & 3 & Critical on 18--20 with unarmed attacks \\
\nameref{pwr:manipulate-creature-with-mind} & 3 & Shove, disarm or grapple creature telekinetically \\
\nameref{pwr:rapid-healing} & 3 & Heal a small number of hit points \\
\nameref{pwr:resuscitation} & 3 & Return creature that has just died to life \\
\nameref{pwr:sacrificial-shell} & 3 & Gain immediate protection from energy \\
\nameref{pwr:steadfast-perception} & 3 & See through visual deceptions \\
\nameref{pwr:unrelenting-voice} & 3 & Wave of force slams targets, pushing them back and knocking them over \\
\nameref{pwr:empathic-feedback} & 4 & Target takes damage when they hit you \\
\nameref{pwr:immovability} & 4 & Make yourself harder to move involuntarily \\
\nameref{pwr:inertial-dampener} & 4 & Become immune to bludgeoning damage \\
\nameref{pwr:psi-door} & 4 & Teleport over a moderate distance \\
\nameref{pwr:weapon-of-energy} & 4 & Imbue weapon with energy damage \\
\nameref{pwr:adapt-body} & 5 & Adapt your body to dangerous environments \\
\nameref{pwr:ironflesh} & 5 & Harden your body and improve its resistance to damage \\
\nameref{pwr:psychofeedback} & 5 & Alter ability scores temporarily \\
\nameref{pwr:breath-of-numinex} & 6 & Powerful breath weapon dealing acid damage \\
\nameref{pwr:mind-blank-psionic} & 6 & Conceal targets mind from external forces \\
\nameref{pwr:suspend-life} & 6 & Slow metabolism and appear dead
\end{dndlongtable}

\DndSetThemeColor[PhbLightGreen]
\section{Power Descriptions}
\label{sec:power_descriptions}
\DndPowerHeader%
  {Demoralise}
  {1st-level Voice}
  {1 action}
  {30 feet}
  {MB 1, PD 1}
  {Concentration, up to 1 minute}
Choose 3 hostile creatures that you can see within range.
Each creature must make a Wisdom saving throw.
If they fail, they target takes a 1d4 penalty to attack rolls and
saving throws until the power ends.
\higher{You may target one additional creature for
        each additional level.}

\chapter{Classes}
\label{chap:classes}

\section{Psion}
\DndSetThemeColor[DmgLavender]
In the audience chamber of the Duke of Baradstone,
a half-elf pries into the pathways of Duke's mind---their
thoughts, feelings, desires revealed to her
with lucid precision.
All the while, the Duke's retinue remain totally oblivious
to the psionic intrusion.

In an vanishing instant of time,
the githzerai projects the movements of his
githyanki foes two seconds into the future,
giving himself the edge he needs desperately needs
to stop their advance.

The tiefling crouches imperceptibly behind the garrison,
recalling her Prana Bindu training regimen to force
her body to make no sound.
Then, she focuses on the sword carried by the guardsmen;
before he takes his next step,
the steel burns a brilliant red before exploding in his
very hand.

In the underbelly of the city,
the dwarf brushes his hand against the ancient stone.
Its feeling, its texture, ignite a subconscious
awareness within his mind,
and he is borne back to the memory of countless aeons
past.
He watches in striking detail as the genetic memory unravels,
showing his forebears laying the same stone---one piece
of a much larger city which was lost to the entropic
assault of time.

\subsection{Masters of the Mind}
While psions are diverse in their abilities and training,
all are unified by a common thread---their unparalleled
mastery over the latent powers that lie within their mind.
While some turn to the magic essence which suffuses the cosmos,
psions peer inwards,
unlocking powers which might have laid dormant over many years.
Only through strict training and profound mental discipline
can these powers be unlocked,
which all psions have been through in one way or another.

Psions manifest distinct effects known as powers, which,
depending on their discipline, achieve myriad effects.
Powers are integral to the effectiveness of a psion,
and the more experience they accrue,
the more powers they unlock from deep within.

\subsection{Creating a Psion}
Creating a character with the psion class prompts
important questions.
How did your character first become aware of the
powers of the mind?
Perhaps they found an old, forgotten tome
describing the \emph{Old Way} and the monks who adhered
to it.
Or they might have been awoken to it abruptly,
awakening a dormant power which manifested fleetingly.

One additional line of inquiry should be about the nature
of their training.
How did your character hone their psionic abilities?
Did they train with the githzerai in their floating citadels
on Limbo?
Did they live with a reclusive order who have knowledge
of psionic powers?
Or did they train by themselves,
experimenting with their mind through trial and error?

\subsubsection{Quick Build}
You can make a psion quickly by following these suggestions.
First, Intelligence should be your highest ability score,
followed by Constitution.
Second, choose the following three 1st-level psion powers:
\power{blow up object with mind},
\power{mind thrust}
and \power{precognition}.
If you have access to a 1st level power on your discipline
sublist (see \secref{subs:psion_discipline}),
choose that over one of the powers given above.

\begin{DndSidebar}[float=htbp]{Multiclassing with the Psion}
    \tocside{Multiclassing with the Psion}
    In order to multiclass into the psion class,
    a character must have a minimum Intelligence score of 13
    (in addition to the usual score requirement for their base class).
    When you multiclass into the psion class,
    you gain proficiency with
    light armour, quarterstaffs and daggers.
\end{DndSidebar}

\begin{figure*}[t]
    \begin{ornamentedtabular}{c c p{4.3cm} c c c c}[title={The Psion}]
        \textbf{Level} & \textbf{P. Bonus} & \textbf{Features} & \textbf{Psi Dice} & \textbf{Mental Bandwidth} & \textbf{Powers Known} & \textbf{Psionic Threshold} \\
        1st  & +2 & Powers, Discipline        & 2   & 2  & 3  & 1 \\
        2nd  & +2 & Microsleep (one use)      & 5   & 2  & 5  & 1 \\
        3rd  & +2 & ---                       & 10  & 3  & 7  & 2 \\
        4th  & +2 & Ability Score Improvement & 13  & 3  & 9  & 2 \\
        5th  & +3 & ---                       & 23  & 4  & 11 & 3 \\
        6th  & +3 & Discipline Tenet          & 28  & 4  & 13 & 3 \\
        7th  & +3 & Mind Palace               & 35  & 4  & 15 & 4 \\
        8th  & +3 & Ability Score Improvement & 42  & 5  & 17 & 4 \\
        9th  & +4 & ---                       & 58  & 5  & 19 & 5 \\
        10th & +4 & Discipline Tenet          & 67  & 6  & 21 & 5 \\
        11th & +4 & Microsleep (two uses)     & 78  & 6  & 22 & 6 \\
        12th & +4 & Ability Score Improvement & 78  & 6  & 24 & 6 \\
        13th & +5 & ---                       & 91  & 7  & 25 & 7 \\
        14th & +5 & Discipline Tenet          & 91  & 7  & 27 & 7 \\
        15h  & +5 & ---                       & 106 & 7  & 28 & 8 \\
        16th & +5 & Ability Score Improvement & 106 & 8  & 30 & 9 \\
        17th & +6 & ---                       & 123 & 8  & 31 & 9 \\
        18th & +6 & ---                       & 132 & 8  & 33 & 9 \\
        19th & +6 & Ability Score Improvement & 143 & 9  & 34 & 9 \\
        20th & +6 & Ascendance                & 156 & 10 & 36 & 9 \\
    \end{ornamentedtabular}
\end{figure*}

\subsection{Class Features}
As a psion, you gain the following features.

\subsubsection{Hit Points}

\listparagraph[4pt]{Hit Dice}{
    1d6 per psion level
}
\listparagraph{Hit Points at 1st Level}{
    6 + your Constitution modifier
}
\listparagraph{Hit Points at Higher Levels}{
    1d6 (or 4) + your Constitution modifier per psion level
    after 1st
}

\subsubsection{Proficiencies}

\listparagraph[4pt]{Armour}{
    Light armour
}
\listparagraph{Weapons}{
    Clubs, daggers, heavy crossbows, light crossbows,
    quarterstaffs, spears
}
\listparagraph{Tools}{
    None
}
\vspace{6pt}
\listparagraph{Saving Throws}{
    Intelligence, Wisdom
}
\listparagraph{Skills}{
    Choose two from History, Religion, Nature,
    Insight and Perception
}

In addition, depending on the psion's chosen discipline,
they acquire an additional skill proficiency.
This is shown in the table below.
\begin{table}[htbp]%
    \begin{DndTable}[width=\columnwidth,
                     header=Additional Profciency]{
                     X X}
        Discipline & Proficiency \\
        Prescience & Investigation \\
        Prana Bindu & Athletics, Acrobatics or Stealth \\
        Voice & Persuasion \\
        Metacreativity & Survival \\
        Spacefolding & Arcana \\
        Psychokinesis & Intimidation
    \end{DndTable}
\end{table}

\subsubsection{Equipment}
You start with the following equipment,
in addition to the equipment granted by your background:
\begin{itemize}
    \item (a) a club or (b) a dagger
    \item a light crossbow and 20 bolts
    \item (a) a scholar's pack or (b) a dungeoneer's pack
    \item Leather armour
\end{itemize}

\subsection{Powers}
\subsubsection{Psi Dice}
The number of powers that you can manifest
is limited by the number of psi dice that
you have available.
This is the principal limit on the output of a psion.
For example, a 9-th level psion with 72 psi dice
can manifest a power costing 1 psi dice 72 times.

The number of available psi dice per level
is shown in the psion class table.
You regain all expended psi dice when you finish
a long rest.

\subsubsection{Mental Bandwidth}
The powers you can manifest without straining yourself
is reflected by your mental bandwidth.
The total number of mental bandwidth slots you have
is shown in the class table.

\subsubsection{Discipline}
\label{subs:psion_discipline}
At 1st level,
you must decide upon a discipline.
Your discipline grants you a specific sublist of powers
which cannot be accessed by other disciplines.
This is called the \define{discipline sublist}.
In addition, your discipline grants you different abilities
when you reach certain levels.
These are called \define{tenets}.

\subsubsection{Powers Known}
At 1st level,
you choose three psion powers of your choice
from either the psion class power list
or your chosen discipline sublist.
You cannot choose powers from discipline sublists
other than that of the discipline you have nominated.

\subparagraph{Psionic Threshold}
You cannot learn nor manifest powers with a level
greater than your psionic threshold,
shown in the class table.
As mentioned in \secref{sub:augmenting},
you also cannot augment powers to a level
beyond your psionic threshold.

\subsubsection{Manifesting Ability}
Your psionic ability modifier is your Intelligence modifier.
The save DC against your psionic powers and your
psionic attack modifier are therefore respectively:
\small\begin{equation*}
    \begin{gathered}
        \text{\textbf{Psionics save DC}}
            = 8 + \text{your proficiency bonus} + \\
                  \text{your Intelligence modifier} \\
        \text{\textbf{Psionics attack modifier}}
            = \text{your proficiency bonus} + \\
              \text{your Intelligence modifier}
    \end{gathered}
\end{equation*}\normalsize

\subsection{Microsleep}
Starting at 2nd level,
as a bonus action on your turn,
you may recover a number of psi dice equal to
your Intelligence modifier multiplied by your psionic threshold.

Once you use this feature,
you must finish a long rest before you can use it again.
Starting at 11th level,
you can use it twice before a long rest.

\subsection{Ability Score Improvement}
When you reach 4th level,
and again at 8th, 12th, 16th and 19th level,
you can increase one ability score of your choice by 2,
or you can increase two of your ability scores by 1.
As normal,
you can't increase an ability score above 20 using this feature.

\subsection{Mind Palace}
Starting at 7th level,
your mind becomes your sanctuary---a palace constructed
from pure thought.
Whenever you are asleep,
you return to this sanctuary,
which exists in a perfect equilibrium between
external and internal influences.
You are fully alert to your surroundings
while you are asleep.
In addition, you can communicate telepathically to other
creatures who are also sleeping within 30 feet of you,
in which case you appear as a character in their dreams.

\subsection{Ascendance}
Starting at 20th level,
your body becomes only a vessel for the vast potential
of your mind.
Your mind alters your biological makeup to better resist damage,
granting you resistance to bludgeoning, piercing and slashing damage.
In addition,
you no longer age,
but you can still be killed otherwise.

\subsection{Prescience Tenets}

\subsubsection{Precognitive Step}
% phb3(4e) 89
% prescience
Starting at 6th level,
you can peer through the mists obscuring the future,
learning how one event unfolds.
At the start of every encounter, roll a d20.
Once during the same encounter,
you can at any time replace one of your attack rolls,
saving throws or skill checks with that roll,
adding modifiers on top of it as usual.
You can also replace an enemy's attack roll against you.

\subsubsection{Epitaph}
Starting at 10th level,
your hone your ability to see through time.
Once during an encounter,
if a hostile creature takes a turn immediately following yours,
you may use a free action at the start of your turn
to ask the DM what the creature will do on their turn,
as conditioned by your explicit choices.
For example,
you may ask the DM what the creature will do
if you move 30 feet in a given direction and manifest a power.
If you deviate at all from your stated conditions,
or if you leave any ambiguity or uncertainty as to those choices,
the prediction may not necessarily come true.

If multiple hostile creatures follow your turn,
you may use this ability for each of those creatures
until the turn order reaches a friendly creature
or the round ends.

\subsubsection{Other Memory}
Starting at 14th level,
your unlock the hereditary memories of your forebears,
allowing you to use their knowledge in your time of need.
Once per day, as an action you may call upon a particular ancestor
with a certain background as shown in the table below.
The nature of the benefit conferred to you depends on the background.
The benefits conferred to you end when you finish a long rest.

\begin{table}[htbp]%
    \begin{DndTable}[width=\columnwidth,
                     header=Other Memory Benefit]{
                     l X}
        Background          & Benefit                                                                   \\
        Warrior             & Gain proficiency in all armours,
                                martial weapons and Strength (Athletics) \\
        Battlemage          & You learn three 3rd-level or lower
                                evocation spells of your choice
                                which can be cast without expending a spell slot;
                                once you cast any of these spells,
                                you can't cast the same spell again;
                                the spellcasting ability for these
                                spells is Intelligence \\
        Commander           & You learn three Uberium faction abilities \\
        Thief               & You gain an extra bonus action \\ 
        Priest              & You learn three 3rd-level or lower
                                spells from the cleric spell list
                                of your choice;
                                these can be cast without expending a spell slot;
                                once you cast any of these spells,
                                you can't cast the same spell again;
                                the spellcasting ability for these
                                spells is Intelligence \\\
        Musician            & You gain proficiency in a musical instrument
                                and Charisma (Performance), or expertise if
                                you are already proficient in that skill
    \end{DndTable}
\end{table}

If you spend subsequent days relying on the same ancestor
and their abilities,
you run the risk of having your personality usurped by them.
For each consecutive day you use the same ancestor,
make a Wisdom saving throw with DC equal to 10 plus the number
of consecutive days.
If you fail the save,
hand your character sheet to the DM.

\subsection{Prana Bindu Tenets}
\label{sub:prana_bindu_tenets}

\subsubsection{Lightning Warfare}
Starting at 6th level,
your mastery of nerve and muscle
allows you to move at incredible speeds.
Your movement speed increases by 20 ft.
In addition, your movement no longer provokes
opportunity attacks.

\subsubsection{Optimised Reflexes}
Starting at 10th level,
your reaction times become almost negligible.
You gain an additional reaction on top of the one
you already have, which can be used in the same way
as the latter.

\subsubsection{Alter Metabolism}
Starting at 14th level,
your mastery over body now extends to your metabolism,
and you are able to by thought alone
alter the chemical makeup of malign substances.
You become immune to the effects of all poisons,
poison damage, the poisoned condition, and any toxic venoms
or secretions.
You also gain resistance to acid damage,
and you become immune to disease.

Additionally, you become immune to the maluses of extreme heat,
extreme cold and high altitude,
unless the conditions are so severe
that the DM decides otherwise.

\subsection{Voice Tenets}

\subsubsection{Inner Voice}
Starting at 6th level,
you gain telepathy up to 120 feet.
Once per day,
you may also send a message up to 10 words long
to a creature you have seen before and that is
presently within 10 miles of you.

\subsubsection{Subtle Voice}
Starting at 10th level,
your mastery of the Voice allows you to manifest Voice
powers without arousing suspicion.
Any non-telepathic Voice power---which ordinarily alerts
creatures that can hear you to your Voice---now is
indistinguishable from normal speech.
\emph{Exception:} Psions who have chosen the Voice discipline
cannot be affected by Subtle Voice.

\subsubsection{Enthrall Creature}
Starting at 14th level,
you gain the ability to dominate other creatures and have them
carry out your bidding.
Twice per day,
you may manifest the \spell{geas} spell as a power,
which counts as a 5th-level Voice power with no PD or MB cost.
Subtle Voice also applies for this power.

\subsection{Metacreativity Tenets}
\label{sub:metacreativity_tenets}
\subsubsection{Create Golem}
\label{subs:menu_a}
Starting at 6th level,
you learn the \power{psionic golem} power,
which allows you to manifest a psionic golem
from any raw material.
You are mentally linked to the psionic golem,
and they answer to your bidding as described in the power.
The statistics of the golem depends on the level
at which \power{psionic golem} is cast.
For example,
casting the power at 3rd-level creates a 3rd-level psionic golem.
The statblocks by golem level are given in \secref{sec:psionic_golem}.

Each manifested golem has a certain number of
\define{metacreative points} (MP).
This is called the \define{MP potential} of the golem.
Metacreative points allow you to augment the golem
with particular abilities.
At 6th level,
you have access to the abilities given in Menu A (see the table below),
all of which have an MP cost of 1.
You cannot add abilities to the golem such that the total MP cost
is above the MP potential of the golem.
For example,
a 4th-level golem with an MP potential of 2
can have two Menu A abilities.

\begin{table}[htbp]%
    \begin{DndTable}[width=\columnwidth,
                     header=Psionic Golem Abilities (Menu A)]{
                     X X}
        Name                & Benefit   \\
        Resilient           & The golem gains an extra 10 hit points \\
        Sturdy              & The golem gains a +2 bonus to its AC \\
        Vertical Propulsion & The golem gains a fly speed equal to 15 feet \\ 
        Celerity            & The golem gains the mobile feat \\
        Resistance          & The golem gains resistance to fire, lightning, cold
                                or thunder damage \\
        Seaworthy           & The golem gains a swim speed of 30 feet \\
        Aggressive          & If the golem hits a creature with a melee attack,
                                it can use its reaction to attempt to shove
                                that creature prone
    \end{DndTable}
\end{table}

\subsubsection{Improved Psionic Golem}
\label{subs:menu_b}
Starting at 10th level,
you gain access to psionic golem abilities on Menu B,
as shown in the table below.
The abilities on Menu B cost 2 MP.

\begin{table}[htbp]%
    \begin{DndTable}[width=\columnwidth,
                     header=Psionic Golem Abilities (Menu B)]{
                     X X}
        Name                & Benefit   \\
        Energy Touch        & The golem deals an extra 1d4 points of
                                fire, cold, electricity or thunder damage when it
                                hits with a melee attack \\
        Extra attack        & Whenever the golem takes the \define{Multiattack} action,
                                it can make one additional slam attack \\
        Auto-reassembly     & The golem heals 5 hit points each round \\ 
        Heavy Deflection    & The golem gains a +3 bonus to its AC  \\
        Highly Resilient    & The golem gains an extra 15 hit points \\
        Improved Critical   & The golem lands a critical hit on a roll of either
                                19 or 20 \\
        Muscular            & The golem gains a +4 bonus to its Strength score \\
    \end{DndTable}
\end{table}

\subsubsection{Heart of Ascension}
\label{subs:menu_c}
Starting at 14th level,
any golem you manifest with \power{psionic golem}
becomes Ascendant.
You gain access to psionic golem abilities on Menu C,
as shown in the table below.
The abilities on Menu C cost 4 MP.

\begin{table}[htbp]%
    \begin{DndTable}[width=\columnwidth,
                     header=Psionic Golem Abilities (Menu C)]{
                     X X}
        Name                & Benefit   \\
        WTF is Truesight!?  & The golem gains truesight out to 60 feet \\
        Supreme Resilience  & The golem gains an extra 30 hit points \\
        Metal Carapace      & All incoming damage from separate sources
                                is reduced by 5 \\ 
        Shield!             & The golem gains a +5 bonus to its AC \\
        Active Camouflage   & The golem becomes naturally invisible \\
        Energy Bolt         & The golem can manifest the \power{energy bolt} power
                                as an action once per round; the energy type is
                                chosen when the golem is initially manifested
    \end{DndTable}
\end{table}

\subsection{Spacefolding Tenets}
\subsubsection{Delete Space}
Starting at 6th level,
you become capable of removing the space between you
and another creature with your mind alone. 
Choose a creature you can see within 60 feet and spend 5 psi dice.
That creature is pulled up to 30 feet in your direction as the space
between you and them is eliminated.

\subsubsection{Wormhole}
Starting at 10th level,
you alter the fabric of space and time such that
you can link two different locations together.
Once per day as an action,
you may create a wormhole mouth at a point you are standing on.
The mouth is a 10-foot radius sphere and is presently closed.
At a later point in time,
you may create another mouth at a different location.
The location need not be on the same plane of existence
as the mouth.
When you do so, the two mouths become adjacent to each other.
Creatures on either side can see to the other side of the wormhole
through the 10-foot sphere,
and can step through using their movement.

You may choose to collapse the wormhole at any time.
However, it is not stable.
For every hour that has elapsed since you placed the second mouth,
roll a d10.
On a roll of 1--5,
the wormhole collapses and the bridge between the two locations
is lost.
You must then create a new wormhole.

\subsubsection{Master of Spacetime}
Starting at 14th level,
you can effortlessly warp space and time locally to
move yourself across the battlefield.
The \power{psionic step} power costs no psi dice and MB
whenever you manifest it.
The power still costs 4 psi dice to augment,
however if you do so,
you can manifest the power as a free action
instead of a bonus action
(limit of once per turn).

\subsection{Psychokinesis Tenets}
\subsubsection{Sculpt Psioncs}
Starting at 6th level,
akin to the evocation wizard's ability to sculpt spells,
you can alter the direction that hazardous effects
created by your psychokinesis powers propagate.
When you manifest a psychokinesis power that affects
other creatures that you can see,
you can choose a number of them equal to your
psionic threshold.
The chosen creatures automatically succeed on any saving
throws they make to resist the power,
taking no damage on a successful save if they would
otherwise take half.
If the power does not prompt a saving throw
(e.g. if it automatically hits),
the chosen creatures also take no damage.

\subsubsection{Signature Damage}
Starting at 10th level,
you become highly proficient in dealing damage
with a specific energy manifested by your psionic powers.
Choose a damage type from fire, cold, lightning and thunder.
Each die of damage you deal with the chosen type
become \define{exploding die}.
If you roll the maximum result on an exploding die,
you can reroll that die again and add the result to the
total damage.
If the result is again the maximum,
you may `explode' the die again.
This continues indefinitely or until you have
no longer rolled the maximum over all dice.

\subsubsection{Signature Power}
Starting at 14th level,
one psychokinesis power become so familiar to you that you
manifest it without straining yourself.
Choose one 3rd-level psychokinesis power.
The MB cost for that power is reduced to 0 whenever you manifest it.
In addition, you can manifest this power without expending
any psi die.
Once you do so,
you can't do so again until you finish a short or long rest.

\clearpage\section{Psi Knight}
\label{sec:psi_knight}
\DndSetThemeColor[DmgCoral]
With a thought alone, the silver-armoured half-orc effortlessly
shunts aside the first gnoll before plunging his longsword
into the abdomen of the next.
Then, with near-imperceptible speed, he appears behind the third gnoll,
who scarcely processes the gruesome end he meets.

In the silver-purple depths of the Astral Plane,
the githyanki astral skiff careens into the illithid nautiloid.
Before long, the erstwhile slaves of shattered Astromundi Nova
engage the mind flayers,
battling each other with thoughts alone.
Then the githyanki vanguard charges,
their silver swords slicing into the exposed flesh of their former
masters.

In the subterranean halls of a long-forgotten city,
the elf uses her mind to propel herself across a seemingly
bottomless crevasse.
Then, landing on the other side,
she shapes the very earth itself into a glowing handaxe,
which she picks up and throws at the awakened undead
bearing down upon her.

\subsection{The Body and the Mind}
Psi Knights supplement their martial abilities
with the latent power of their mind to
achieve victory over their enemies.
Formidable opponents,
psi knights are rare and highly accomplished warriors
who achieve an almost unparalleled balance between mind
and body.
Adventuring parties often readily accept a noble psi knight
by their side,
and the psi knight's enemies on the battlefield tremble
as they approach.

Like psions, psi knights manifest powers with myriad effects.
While they cannot manifest as many powers as psions,
they have substantial martial capacity
and have a greater ability to weather the tide of battle. 

\subsubsection{Creating a Psi Knight}
When creating a psi knight,
ask yourself why your character has chosen the path of psionics.
Perhaps they belonged to a small community of like-minded warriors
who understood the untapped potential of psionic powers.
Or perhaps they ventured down a solitary path,
testing the powers they learnt piecemeal in the din of battle.

Also ask yourself how they interact with the broader community.
Do they see themselves as an ordinary person with a fortuitous
talent?
Or do they feel that they rise above the masses?

\subsubsection{Quick Build}
You can make a psi knight quickly by following these suggestions.
First, Strength or Dexterity should be your highest ability score,
followed by Wisdom then Constitution.
Second, choose the soldier background.
Third, pick the \power{boots of the demon king} power
from the psi knight class power list.

\begin{DndSidebar}[float=htbp]{Multiclassing with the Psi Knight}
    \tocside{Multiclassing with the Psi Knight}
    In order to multiclass into the psi knight class,
    a character must have a minimum Strength or Dexterity score of 13
    and a minimum Wisdom score of 13
    (in addition to the usual score requirement for their base class).
    When you multiclass into the psi knight class,
    you gain proficiency with
    light armour, medium armour, shields, simple weapons
    and martial weapons.
\end{DndSidebar}

\begin{figure*}[t]
    \begin{ornamentedtabular}{c c p{4.3cm} c c c c}[title={The Psi Knight}]
        \textbf{Level} & \textbf{P. Bonus} & \textbf{Features} & \textbf{Psi Dice} & \textbf{Mental Bandwidth} & \textbf{Powers Known} & \textbf{Psionic Threshold} \\
        1st  & +2 & Powers, Fighting Style    & 1   & 1  & 1  & 1 \\
        2nd  & +2 & Prana Bindu Initiate      & 2   & 1  & 2  & 1 \\
        3rd  & +2 & Psionic Tradition         & 3   & 1  & 3  & 1 \\
        4th  & +2 & Ability Score Improvement & 4   & 2  & 4  & 2 \\
        5th  & +3 & Extra attack              & 6   & 2  & 5  & 2 \\
        6th  & +3 & Psionic Tradition feature & 9   & 2  & 6  & 2 \\
        7th  & +3 & Iron Will                 & 12  & 3  & 7  & 3 \\
        8th  & +3 & Ability Score Improvement & 15  & 3  & 8  & 3 \\
        9th  & +4 & ---                       & 18  & 3  & 9  & 3 \\
        10th & +4 & Psionic Tradition feature & 21  & 3  & 10 & 4 \\
        11th & +4 & Calmness of Self          & 25  & 4  & 11 & 4 \\
        12th & +4 & Ability Score Improvement & 29  & 4  & 12 & 4 \\
        13th & +5 & ---                       & 33  & 4  & 13 & 5 \\
        14th & +5 & ---                       & 37  & 4  & 14 & 5 \\
        15th & +5 & Psionic Tradition feature & 42  & 4  & 15 & 5 \\
        16th & +5 & Ability Score Improvement & 47  & 5  & 16 & 6 \\
        17th & +6 & ---                       & 52  & 5  & 17 & 6 \\
        18th & +6 & Psionic Tradition feature & 58  & 5  & 18 & 6 \\
        19th & +6 & Ability Score Improvement & 64  & 5  & 19 & 6 \\
        20th & +6 & Psionic Reserve           & 70  & 5  & 20 & 6 \\
    \end{ornamentedtabular}
\end{figure*}

\subsection{Class Features}
As a psi knight, you gain the following features.

\subsubsection{Hit Points}

\listparagraph[4pt]{Hit Dice}{
    1d10 per psi knight level
}
\listparagraph{Hit Points at 1st Level}{
    10 + your Constitution modifier
}
\listparagraph{Hit Points at Higher Levels}{
    1d10 (or 6) + your Constitution modifier per psion level
    after 1st
}

\subsubsection{Proficiencies}

\listparagraph[4pt]{Armour}{
    All armour, shields
}
\listparagraph{Weapons}{
    Simple weapons, martial weapons
}
\listparagraph{Tools}{
    None
}
\vspace{6pt}
\listparagraph{Saving Throws}{
    Constitution, Wisdom
}
\listparagraph{Skills}{
    Choose two from Athletics, Acrobatics, Insight,
    Perception, Animal Handling, Intimidation
    and Survival
}

\subsubsection{Equipment}
You start with the following equipment,
in addition to the equipment granted by your background:
\begin{itemize}
    \item (a) chain mail or
          (b) leather armour, longbow
            and 20 arrows
    \item a martial weapon and a shield or
          (b) two martial weapons
    \item (a) a light crossbow and 20 bolts or
          (b) two handaxes
    \item (a) a dungeoneer's pack or
          (b) an explorer's pack
\end{itemize}

\subsection{Fighting Style}
You adopt a particular fighting style as your specialty.
Choose one of the following options.
You can't take a Fighting Style option more than once,
even if you later get to choose again.

\subsubsection{Blind Fighting}
You have blindsight with a range of 10 feet.
Within that range, you can effectively see anything that
isn't behind total cover,
even if you're blinded or in darkness.
Moreover, you can see an invisible creature within that range,
unless the creature successfully hides from you.

\subsubsection{Defence}
While you are wearing armour, you gain a +1 bonus to AC.

\subsubsection{Dueling}
When you are wielding a melee weapon in one hand
and no other weapons,
you gain a +2 bonus to damage rolls with that weapon.

\subsubsection{Great Weapon Fighting}
When you roll a 1 or 2 on a damage die for an attack you make
with a melee weapon that you are wielding with two hands,
you can reroll the die and must use the new roll,
even if the new roll is a 1 or a 2.
The weapon must have the two-handed or versatile property
for you to gain this benefit.

\subsubsection{Interception}
When a creature you can see hits a target,
other than you,
within 5 feet of you with an attack,
you can use your reaction to reduce the damage the target takes
by 1d10 + your proficiency bonus (to a minimum of 0 damage).
You must be wielding a shield or a simple or martial weapon
to use this reaction.

\subsubsection{Protection}
When a creature you can see attacks a target other than you
that is within 5 feet of you,
you can use your reaction to impose disadvantage on the attack roll.
You must be wielding a shield.

\subsection{Powers}
\subsubsection{Psi Dice}
The number of powers that you can manifest
is limited by the number of psi dice that
you have available.
This is the principal limit on the output of a psion.
For example, a 9-th level psi knight with 18 psi dice
can manifest a power costing 1 psi dice 18 times.

The number of available psi dice per level
is shown in the psion class table.
You regain all expended psi dice when you finish
a long rest.

\subsubsection{Mental Bandwidth}
The powers you can manifest without straining yourself
is reflected by your mental bandwidth.
The total number of mental bandwidth slots you have
is shown in the class table.

\subsubsection{Powers Known}
At 1st level,
you choose one psion power of your choice
from the psi knight class power list.

\subparagraph{Psionic Threshold}
You cannot learn nor manifest powers with a level
greater than your psionic threshold,
shown in the class table.
As mentioned in \secref{sub:augmenting},
you also cannot augment powers to a level
beyond your psionic threshold.

\subsubsection{Manifesting Ability}
Your psionic ability modifier is your Wisdom modifier.
The save DC against your psionic powers and your
psionic attack modifier are therefore respectively:
\small\begin{equation*}
    \begin{gathered}
        \text{\textbf{Psionics save DC}}
            = 8 + \text{your proficiency bonus} + \\
                  \text{your Wisdom modifier} \\
        \text{\textbf{Psionics attack modifier}}
            = \text{your proficiency bonus} + \\
              \text{your Wisdom modifier}
    \end{gathered}
\end{equation*}\normalsize

\subsection{Prana Bindu Initiate}
Starting at 2nd level,
your mind-body training makes manifesting powers
belonging to the Prana Bindu discipline
vastly easier.
Prana Bindu powers require one less mental bandwidth slot
than mentioned in their description.

\subsection{Psionic Tradition}
When you reach 3rd level,
you embark on a journey down a particular
psionic tradition.
The tradition represents the aspects of
your mind-body mastery that you have decided
to focus on.
At this stage, two traditions are offered:
the Weirding Way, which focuses on your mastery
of devastatingly precise and supremely fast
Prana Bindu techniques,
and the Commander,
who couples their experience in battle
with profound psionic powers to meticulously
plan and execute combat stratagems.
Your tradition grants you features at 3rd level,
and again at 6th, 10th, 15th and 18th level.

\subsection{Ability Score Improvement}
When you reach 4th level,
and again at 8th, 12th, 16th and 19th level,
you can increase one ability score of your choice by 2,
or you can increase two of your ability scores by 1.
As normal,
you can't increase an ability score above 20 using this feature.

\subsection{Extra Attack}
Beginning at 5th level, you can attack twice,
instead of once,
whenever you take the Attack action on your turn.

\subsection{Iron Will}
Starting at 7th level,
your mind is able to override your body at times
when the untrained would falter.
Whenever you are reduced to hit points,
but not killed outright,
you can spend 5 psi dice to immediately regain hit points
equal to your Wisdom modifier plus the number of levels
you have in the psi knight class.

\subsection{Calmness of Self}
Starting at 11th level,
your unfettered desire to improve mind and body
has granted you a sense of inner calm.
You have resistance to psychic damage.
Moreover, if you start your turn charmed or frightened,
you can expend a 3 psi dice and end every effect on yourself
subjecting you to those conditions.
In addition, you have advantage on saving throws to resist
effects from Voice psionic powers.

\subsection{Psionic Reserve}
Starting at 20th level,
when you roll for initiative and have no psi dice remaining,
you regain 10 psi dice.

\subsection{Weirding Way}
The Weirding Way champions incredible control over one's nerves
and reaction times---granted by Prana Bindu techniques---to
manoeuvre and strike at opponents before they can retaliate.
The name `Weirding Way' comes from the fact that,
to untrained opponents, the movements of those versed in
the Weirding Way appear unnatural and superhuman,
almost as if they are teleporting across the battlefield.

\subsubsection{Prana Bindu Prodigy}
Starting at 3rd level,
you can choose to learn any Prana Bindu tenet from
the psion discipline at \secref{sub:prana_bindu_tenets}.
The level requirement stated there does not apply to you.

\subsubsection{Seize the Initiative}
Starting at 6th level,
whenever a fight breaks out,
you can reliably act before your enemies.
If you spend two psi dice,
you can grant yourself advantage on an initiative roll.

\subsubsection{CQC Dominance}
Starting at 10th level,
you make quick use of off-hand and blunt strikes to
achieve dominance in CQC (`close quarters combat').
You have advantage on all checks when taking the
grapple, shove, shove aside, or overrun action.

In addition,
whenever a creature within 5 feet of you ends their turn,
you can use your reaction and spend 2 psi die
to make a single attack against that creature.
The damage type is replaced with bludgeoning damage,
since you use your off-hand,
a kick, a punch or the blunt end of your weapon
to make the attack.
Whether or not the attack is made with a weapon
or is an unarmed attack is your choice.

\subsubsection{Slippery Movement}
Starting at 15th level,
your incredible agility makes its difficult for your
enemies to pin you down.
You have advantage on all saving throws against effects
which would slow your movement or reduce it to 0.
In addition,
if you become restrained or grappled,
you can use a bonus action on your turn
to attempt to repeat the save
or contest respectively which caused that condition.
You do not have advantage from Slippery Movement
on either of these rolls,
but can have advantage from other sources as usual.

\subsubsection{Lisan al Gaib}
Starting at 18th level,
your perfection of Prana Bindu methods becomes legendary,
and in the eyes of many your are considered a
sublime master of the body and mind.
You may pick an additional Prana Bindu tenet from
\secref{sub:prana_bindu_tenets}.

\subsection{Commander}
The Commander uses their experience and psionic abilities
to control and manipulate the battlefield as they desire.
Enemies fear a psi knight commander,
for they know that they will struggle to ever
gain tactical advantage over them and their allies.

\subsubsection{Wise Strikes}
Starting at 3rd level,
your battlefield experience ensures that your attacks
are as damaging as they can be for your enemy.
You may add your Wisdom modifier to all damage
rolls for melee attacks. 

\subsubsection{Knowledge is Power}
Starting at 6th level,
once per day as an action,
you choose a hostile creature you can see within
30 feet of you.
Your mind links with them,
and you gain a deep understanding of their
strengths, weaknesses and peculiarities.
The DM then hands you the statblock of that creature.
When the encounter ends,
or after 1 in-game minute if an encounter is not occurring,
you must return the statblock.

\subsubsection{Order of Battle}
Starting at 10th level,
your mind's eye expands,
and you begin to see the battlefield from a top-down perspective.
Enemies and allies are pieces on a board, which you can manipulate
with thought alone.
Once per day,
at the start of an encounter but before initiative is rolled,
you may forego initiative rolls and instead
declare the initiative order for all combatants.

This ability does not apply when you are surprised.
You also cannot alter the initiative order of creatures
in the encounter that you have no knowledge of,
those that join later in the fight (such as those
that are summoned),
and environmental effects or lair actions
which act on a set initiative count.

\subsubsection{Gambit}
Starting at 15th level,
your see fighters on a battlefield as pieces on a chessboard
that can be effortlessly manipulated---you simply turn your
mind to them, and they bend to your will.
You have a pool of 120 feet of movement.
On your turn,
you may use your action to attempt to move any number of creatures
cumulatively up to 120 feet in any direction,
including vertically.
Each creature you wish to move can resist this movement by
succeeding a Wisdom saving throw with DC equal to your
psionics save DC.

Any unspent movement remaining in your pool persists
and can be carried over to subsequent turns.
Your pool of movement replenishes whenever you finish a long rest.

\subsubsection{Only a Simulation}
Sometimes the tide of battle can be unpredictable.
Fortunately for you,
starting at 18th level,
your supreme mental abilities confer you the ability to
run combat simulations viewed internally by your mind's eye.
These simulations allow you to decide the best course of action.

Once per day, on your turn,
you can use an action to undo the previous round of combat,
allowing you to `replay' that round.
This ability reverses time to the point in the past
just prior to your previous turn,
undoing the effects of everyone else's actions in the meantime.
Once you have used Only a Simulation,
only you retain knowledge of what happened during the round
that is being replayed; after all, it was just a simulation.
However, you can communicate that knowledge verbally to your companions,
if desired,
and you can act on that privileged knowledge
in any way you desire.

\chapter{Miscellany}
\section{Psion/Psi Knight Multiclassing}
There is the possibility that a character chooses to multiclass
into both the psion and psi knight classes.
If this is the case,
the following rules apply with respect to the Powers feature
of each class.
\subparagraph{Psi Dice}
    Psi Dice add cumulatively between the two classes.
    For example,
    if a character has 3 levels in psion (10 psi dice)
    and then multiclass into psi knight,
    they would gain 0 psi dice,
    since at first level the psi knight starts at 0 psi dice.
    If they took a further level in psi knight,
    they would gain 1 psi dice,
    and so on.
\subparagraph{Mental Bandwidth}
    Your mental bandwidth is the highest mental bandwidth you
    have across both classes.
\subparagraph{Powers Known}
    The same logic as for psi dice applies for
    determining the number of powers you know.
\subparagraph{Psionic Threshold}
    The same logic as for mental bandwidth applies for
    determining your psionic threshold.

\section{Feats}
This section adds the following feats on top of the already
existing ones.

\DndFeatHeader{Psionics Initiate}[
    Prerequisite: Wisdom 13 (psi knight); Intelligence 13 (psion)]
    Choose either the psi knight or psion class.
    You learn two 1st-level powers from the powers list
    of that class assuming you meet the prerequisites
    for the nominated class given above.
    Each power counts as a (1/day) innate talent for you.

    Your psionics ability for these powers depends on the class
    you choose:
    Wisdom for psi knight; Intelligence for psion.

\DndFeatHeader{Voice Resistance}[
    Prerequisite: None]
    You have trained extensively and rigorously in the ways
    of resisting the dominating effects of the Voice.
    You gain advantage on all saving throws against
    Voice powers.
    You also know whenever a Voice power is manifested,
    even if the manifester has the \emph{Subtle Voice} ability.

\DndFeatHeader{Psionic Litany}[
    Prerequisite: The ability to manifest at least one power
]
You have committed a psionic litany to memory, granting you respite
and focus in fear-inducing and hostile conditions.
You gain advantage on Constitution saving throws that you make
to maintain concentration on a power when you take damage.

\section{Psionic Golems}
\label{sec:psionic_golem}
The statblocks for the psionic golems described in
\secref{sub:metacreativity_tenets}
are given here, beginning on the following page.

\clearpage\begin{DndMonster}[float*=b,width=\textwidth + 8pt]{1st-Level Psionic Golem}
\begin{multicols}{2}
    \DndMonsterType{Medium construct, unaligned}
  
    \DndMonsterBasics[
        armor-class = {18 (natural armour)},
        hit-points  = {\DndDice{3d10 + 5}},
        speed       = {25 ft.},
      ]
  
    \DndMonsterAbilityScores[
        str = 14,
        dex = 11,
        con = 13,
        int = 1,
        wis = 3,
        cha = 1,
      ]
  
    \DndMonsterDetails[
        damage-immunities = {poison},
        condition-immunities = {blinded, charmed, deafened, exhaustion,
                                frightened, paralyzed, petrified, poisoned},
        senses = {darkvision 30 ft., passive Perception 6},
        languages = {---},
        challenge = 1,
        proficiency = +2
        ]
    % Traits
    \DndMonsterAction{Golem}
    The psionic golem cannot be put to sleep, is immune to disease
    and cannot be raised by necromancy or similar spells.
    The golem also cannot be healed by magic or other means,
    but can be repaired by manifested powers.

    \DndMonsterAction{MP Potential}
    The psionic golem has 1 MP, which may be spent on abilities
    in Menus you have access to.
    
    % Actions
    \DndMonsterSection{Actions}
    \DndMonsterAction{Multiattack}
    The psionic golem makes two slam attacks.
  
    \DndMonsterMelee[
      name=Slam,
      mod=+4,
      reach=5,
      targets=one target,
      dmg=\DndDice{1d6+2},
      dmg-type=bludegoning,
    ]
\end{multicols}
\end{DndMonster}

\begin{DndMonster}[float*=b,width=\textwidth + 8pt]{2nd-Level Psionic Golem}
\begin{multicols}{2}
    \DndMonsterType{Medium construct, unaligned}
  
    \DndMonsterBasics[
        armor-class = {18 (natural armour)},
        hit-points  = {\DndDice{4d10 + 6}},
        speed       = {30 ft.},
      ]
  
    \DndMonsterAbilityScores[
        str = 16,
        dex = 11,
        con = 15,
        int = 1,
        wis = 3,
        cha = 1,
      ]
  
    \DndMonsterDetails[
        damage-immunities = {poison},
        condition-immunities = {blinded, charmed, deafened, exhaustion,
                                frightened, paralyzed, petrified, poisoned},
        senses = {darkvision 30 ft., passive Perception 6},
        languages = {---},
        challenge = 1,
        proficiency = +2
      ]
    % Traits
    \DndMonsterAction{Golem}
    The psionic golem cannot be put to sleep, is immune to disease
    and cannot be raised by necromancy or similar spells.
    The golem also cannot be healed by magic or other means,
    but can be repaired by manifested powers.

    \DndMonsterAction{MP Potential}
    The psionic golem has 1 MP, which may be spent on abilities
    in Menus you have access to.
    
    % Actions
    \DndMonsterSection{Actions}
    \DndMonsterAction{Multiattack}
    The psionic golem makes two slam attacks.
  
    \DndMonsterMelee[
      name=Slam,
      mod=+5,
      reach=5,
      targets=one target,
      dmg=\DndDice{1d6+3},
      dmg-type=bludegoning,
    ]
\end{multicols}
\end{DndMonster}

\begin{DndMonster}[float*=b,width=\textwidth + 8pt]{3rd-Level Psionic Golem}
\begin{multicols}{2}
    \DndMonsterType{Medium construct, unaligned}
  
    \DndMonsterBasics[
        armor-class = {18 (natural armour)},
        hit-points  = {\DndDice{5d10 + 6}},
        speed       = {30 ft.},
      ]
  
    \DndMonsterAbilityScores[
        str = 16,
        dex = 11,
        con = 15,
        int = 1,
        wis = 3,
        cha = 1,
      ]
  
    \DndMonsterDetails[
        damage-immunities = {poison},
        condition-immunities = {blinded, charmed, deafened, exhaustion,
                                frightened, paralyzed, petrified, poisoned},
        senses = {darkvision 30 ft., passive Perception 6},
        languages = {---},
        challenge = 1,
        proficiency = +2
      ]
    % Traits
    \DndMonsterAction{Golem}
    The psionic golem cannot be put to sleep, is immune to disease
    and cannot be raised by necromancy or similar spells.
    The golem also cannot be healed by magic or other means,
    but can be repaired by manifested powers.

    \DndMonsterAction{MP Potential}
    The psionic golem has 1 MP, which may be spent on abilities
    in Menus you have access to.
    
    % Actions
    \DndMonsterSection{Actions}
    \DndMonsterAction{Multiattack}
    The psionic golem makes two slam attacks.
  
    \DndMonsterMelee[
      name=Slam,
      mod=+5,
      reach=5,
      targets=one target,
      dmg=\DndDice{2d6+3},
      dmg-type=bludegoning,
    ]
\end{multicols}  
\end{DndMonster}

\begin{DndMonster}[float*=b,width=\textwidth + 8pt]{4th-Level Psionic Golem}
\begin{multicols}{2}
    \DndMonsterType{Medium construct, unaligned}
  
    \DndMonsterBasics[
        armor-class = {18 (natural armour)},
        hit-points  = {\DndDice{6d10 + 12}},
        speed       = {30 ft.},
      ]
  
    \DndMonsterAbilityScores[
        str = 18,
        dex = 13,
        con = 17,
        int = 1,
        wis = 3,
        cha = 1,
      ]
  
    \DndMonsterDetails[
        damage-immunities = {poison},
        condition-immunities = {blinded, charmed, deafened, exhaustion,
                                frightened, paralyzed, petrified, poisoned},
        senses = {darkvision 30 ft., passive Perception 6},
        languages = {---},
        challenge = 4,
        proficiency = +2
      ]
    % Traits
    \DndMonsterAction{Golem}
    The psionic golem cannot be put to sleep, is immune to disease
    and cannot be raised by necromancy or similar spells.
    The golem also cannot be healed by magic or other means,
    but can be repaired by manifested powers.

    \DndMonsterAction{MP Potential}
    The psionic golem has 2 MP, which may be spent on abilities
    in Menus you have access to.
    
    % Actions
    \DndMonsterSection{Actions}
    \DndMonsterAction{Multiattack}
    The psionic golem makes two slam attacks.
  
    \DndMonsterMelee[
      name=Slam,
      mod=+6,
      reach=5,
      targets=one target,
      dmg=\DndDice{2d6+4},
      dmg-type=bludegoning,
    ]
\end{multicols}  
\end{DndMonster}

\begin{DndMonster}[float*=b,width=\textwidth + 8pt]{5th-Level Psionic Golem}
\begin{multicols}{2}
    \DndMonsterType{Large construct, unaligned}
  
    \DndMonsterBasics[
        armor-class = {19 (natural armour)},
        hit-points  = {\DndDice{7d10 + 12}},
        speed       = {35 ft.},
      ]
  
    \DndMonsterAbilityScores[
        str = 18,
        dex = 15,
        con = 17,
        int = 1,
        wis = 3,
        cha = 1,
      ]
  
    \DndMonsterDetails[
        damage-immunities = {poison},
        condition-immunities = {blinded, charmed, deafened, exhaustion,
                                frightened, paralyzed, petrified, poisoned},
        senses = {darkvision 30 ft., passive Perception 6},
        languages = {---},
        challenge = 5,
        proficiency = +3
      ]
    % Traits
    \DndMonsterAction{Golem}
    The psionic golem cannot be put to sleep, is immune to disease
    and cannot be raised by necromancy or similar spells.
    The golem also cannot be healed by magic or other means,
    but can be repaired by manifested powers.

    \DndMonsterAction{MP Potential}
    The psionic golem has 2 MP, which may be spent on abilities
    in Menus you have access to.

    \DndMonsterAction{Damage Threshold (5)}
    The psionic golem has immunity to all damage unless it takes
    an amount of damage from a single attack or effect equal to
    or greater than its damage threshold,
    in which case it takes damage as normal.
    
    % Actions
    \DndMonsterSection{Actions}
    \DndMonsterAction{Multiattack}
    The psionic golem makes two slam attacks.
  
    \DndMonsterMelee[
      name=Slam,
      mod=+7,
      reach=5,
      targets=one target,
      dmg=\DndDice{3d6+4},
      dmg-type=bludegoning,
    ]
\end{multicols}  
\end{DndMonster}

\begin{DndMonster}[float*=b,width=\textwidth + 8pt]{6th-Level Psionic Golem}
\begin{multicols}{2}
    \DndMonsterType{Large construct, unaligned}
  
    \DndMonsterBasics[
        armor-class = {19 (natural armour)},
        hit-points  = {\DndDice{9d10 + 12}},
        speed       = {40 ft.},
      ]
  
    \DndMonsterAbilityScores[
        str = 18,
        dex = 15,
        con = 19,
        int = 1,
        wis = 3,
        cha = 1,
      ]
  
    \DndMonsterDetails[
        damage-immunities = {poison},
        condition-immunities = {blinded, charmed, deafened, exhaustion,
                                frightened, paralyzed, petrified, poisoned},
        senses = {darkvision 30 ft., passive Perception 6},
        languages = {---},
        challenge = 6,
        proficiency = +3
      ]
    % Traits
    \DndMonsterAction{Golem}
    The psionic golem cannot be put to sleep, is immune to disease
    and cannot be raised by necromancy or similar spells.
    The golem also cannot be healed by magic or other means,
    but can be repaired by manifested powers.

    \DndMonsterAction{MP Potential}
    The psionic golem has 2 MP, which may be spent on abilities
    in Menus you have access to.

    \DndMonsterAction{Damage Threshold (8)}
    The psionic golem has immunity to all damage unless it takes
    an amount of damage from a single attack or effect equal to
    or greater than its damage threshold,
    in which case it takes damage as normal.
    
    % Actions
    \DndMonsterSection{Actions}
    \DndMonsterAction{Multiattack}
    The psionic golem makes two slam attacks.
  
    \DndMonsterMelee[
      name=Slam,
      mod=+7,
      reach=5,
      targets=one target,
      dmg=\DndDice{4d6+4},
      dmg-type=bludegoning,
    ]
\end{multicols}  
\end{DndMonster}

\begin{DndMonster}[float*=b,width=\textwidth + 8pt]{7th-Level Psionic Golem}
\begin{multicols}{2}
    \DndMonsterType{Large construct, unaligned}
  
    \DndMonsterBasics[
        armor-class = {19 (natural armour)},
        hit-points  = {\DndDice{11d10 + 12}},
        speed       = {40 ft.},
      ]
  
    \DndMonsterAbilityScores[
        str = 19,
        dex = 15,
        con = 19,
        int = 1,
        wis = 3,
        cha = 1,
      ]
  
    \DndMonsterDetails[
        damage-immunities = {poison},
        condition-immunities = {blinded, charmed, deafened, exhaustion,
                                frightened, paralyzed, petrified, poisoned},
        senses = {darkvision 30 ft., passive Perception 6},
        languages = {---},
        challenge = 7,
        proficiency = +3
      ]
    % Traits
    \DndMonsterAction{Golem}
    The psionic golem cannot be put to sleep, is immune to disease
    and cannot be raised by necromancy or similar spells.
    The golem also cannot be healed by magic or other means,
    but can be repaired by manifested powers.

    \DndMonsterAction{MP Potential}
    The psionic golem has 4 MP, which may be spent on abilities
    in Menus you have access to.

    \DndMonsterAction{Damage Threshold (12)}
    The psionic golem has immunity to all damage unless it takes
    an amount of damage from a single attack or effect equal to
    or greater than its damage threshold,
    in which case it takes damage as normal.
    
    % Actions
    \DndMonsterSection{Actions}
    \DndMonsterAction{Multiattack}
    The psionic golem makes two slam attacks.
  
    \DndMonsterMelee[
      name=Slam,
      mod=+7,
      reach=5,
      targets=one target,
      dmg=\DndDice{4d6+4},
      dmg-type=bludegoning,
    ]
\end{multicols}  
\end{DndMonster}

\begin{DndMonster}[float*=b,width=\textwidth + 8pt]{8th-Level Psionic Golem}
\begin{multicols}{2}
    \DndMonsterType{Large construct, unaligned}
  
    \DndMonsterBasics[
        armor-class = {19 (natural armour)},
        hit-points  = {\DndDice{15d10 + 12}},
        speed       = {40 ft.},
      ]
  
    \DndMonsterAbilityScores[
        str = 20,
        dex = 15,
        con = 20,
        int = 1,
        wis = 3,
        cha = 1,
      ]
  
    \DndMonsterDetails[
        damage-immunities = {poison},
        condition-immunities = {blinded, charmed, deafened, exhaustion,
                                frightened, paralyzed, petrified, poisoned},
        senses = {darkvision 30 ft., passive Perception 6},
        languages = {---},
        challenge = 8,
        proficiency = +3
      ]
    % Traits
    \DndMonsterAction{Golem}
    The psionic golem cannot be put to sleep, is immune to disease
    and cannot be raised by necromancy or similar spells.
    The golem also cannot be healed by magic or other means,
    but can be repaired by manifested powers.

    \DndMonsterAction{MP Potential}
    The psionic golem has 4 MP, which may be spent on abilities
    in Menus you have access to.

    \DndMonsterAction{Damage Threshold (15)}
    The psionic golem has immunity to all damage unless it takes
    an amount of damage from a single attack or effect equal to
    or greater than its damage threshold,
    in which case it takes damage as normal.
    
    % Actions
    \DndMonsterSection{Actions}
    \DndMonsterAction{Multiattack}
    The psionic golem makes two slam attacks.
  
    \DndMonsterMelee[
      name=Slam,
      mod=+8,
      reach=5,
      targets=one target,
      dmg=\DndDice{4d6+5},
      dmg-type=bludegoning,
    ]
\end{multicols}  
\end{DndMonster}

\begin{DndMonster}[float*=b,width=\textwidth + 8pt]{9th-Level Psionic Golem}
\begin{multicols}{2}
    \DndMonsterType{Huge construct, unaligned}
  
    \DndMonsterBasics[
        armor-class = {20 (natural armour)},
        hit-points  = {\DndDice{18d10 + 12}},
        speed       = {40 ft.},
      ]
  
    \DndMonsterAbilityScores[
        str = 20,
        dex = 15,
        con = 20,
        int = 1,
        wis = 3,
        cha = 1,
      ]
  
    \DndMonsterDetails[
        damage-immunities = {poison},
        condition-immunities = {blinded, charmed, deafened, exhaustion,
                                frightened, paralyzed, petrified, poisoned},
        senses = {darkvision 30 ft., passive Perception 6},
        languages = {---},
        challenge = 9,
        proficiency = +4
      ]
    % Traits
    \DndMonsterAction{Golem}
    The psionic golem cannot be put to sleep, is immune to disease
    and cannot be raised by necromancy or similar spells.
    The golem also cannot be healed by magic or other means,
    but can be repaired by manifested powers.

    \DndMonsterAction{MP Potential}
    The psionic golem has 8 MP, which may be spent on abilities
    in Menus you have access to.

    \DndMonsterAction{Damage Threshold (20)}
    The psionic golem has immunity to all damage unless it takes
    an amount of damage from a single attack or effect equal to
    or greater than its damage threshold,
    in which case it takes damage as normal.
    
    % Actions
    \DndMonsterSection{Actions}
    \DndMonsterAction{Multiattack}
    The psionic golem makes two slam attacks.
  
    \DndMonsterMelee[
      name=Slam,
      mod=+9,
      reach=5,
      targets=one target,
      dmg=\DndDice{4d6+5},
      dmg-type=bludegoning,
    ]
\end{multicols}
\end{DndMonster}

\end{document}