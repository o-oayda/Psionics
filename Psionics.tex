\documentclass[
    10pt,
    twoside,
    twocolumn,
    openany,
    nodeprecatedcode,
    bg=full,
    justified%
    ]{dndbook}

\usepackage[english]{babel}
\usepackage[utf8]{inputenc}
\usepackage{hyperref}
\usepackage{rotating}
\newcommand{\secref}[1]{\ref{#1}. \nameref{#1}}
\newcommand{\define}[1]{\textbf{#1}}
\newcommand{\augment}[1]{\newline\indent\textbf{Augment.} #1}
\newcommand{\spell}[1]{\textit{#1}}
\newcommand{\power}[1]{\textit{#1}}
\usepackage{lipsum}
\usepackage{adjustbox}
\usepackage{rotating}
\usepackage{amsmath}
\usetikzlibrary{intersections}

\hypersetup{%
	colorlinks = true,%
	linkcolor =blue,%
	anchorcolor = red,%
	citecolor = blue,%
	urlcolor = blue%
}

\title{\Huge\scshape The Old Way \\
    \large A Psionics Supplement}
\author{DM Oayda}
% \usepackage[capitalise,nameinlink,noabbrev]{cleveref}
\setcounter{tocdepth}{2}
\setcounter{secnumdepth}{2}

\cftsetindents{subsection}{1em}{14mm} % toc spacing

\begin{document}

\frontmatter
\maketitle
\tableofcontents

\mainmatter%

\chapter{Introduction}
% TODO
% - finish powers
% - add spells interacting with psionics
% - add psionics interacting with spells
% - psionic feats
% - innate talents for certain races e.g. Gith

\chapter{Using Psionics}

\DndDropCapLine{T}{his chapter explains} the nature of psionics,
as well as the key rules behind
manifesting and maintaining psionic powers.

\section{What are Psionics?}
Psionics tap into the latent ability of the mind
to master itself, its body and its environment.
Whereas spellcasters manipulate the magical essence
innate to the cosmological order,
users of psionics rely on their mind's power alone.
For many creatures,
manifesting any psionic power,
let alone mastering a psionic discipline,
takes years, even decades of rigorous mental training.
That being said,
some creatures have a natural propensity for psionics,
for example the children of Gith, illithids and aboleths.

Many refer to psionics as \emph{the Old Way}.
This is because there is a school of thought that
the seeds of all creatures---the progenitor beings
which gave rise to existence itself---were
masters of the mind.
According to this doctrine,
magic and spellcraft were created by these
beings when the reality was moulded to as it is today.
To do this, they used psionics.

\section{Types of Psionic Abilities}
There are two types of psionic abilities.
The main type are \textbf{powers},
which are what most psionics users will be
interacting with the most.
The rules around powers are described in this chapter.

A second type is an \textbf{innate talent}.

\subsection{Interaction with Magic}
Since psionics and magic are disparate entities,
how do they interact together?
A creature that manifests a psionic power is not
casting a spell,
and therefore any ability which negates the effects
of magic cannot negate the power.
For example, psionic powers cannot be affected by
\spell{counterspell} or \spell{dispel magic}.

However, certain psionic powers are explicitly stated
to interact with magic.
In addition, this supplement adds spells to the existing
grimoire which explicitly interact with psioics.

\section{Manifesting Powers}
In order to manifest a psionic power,
the \define{power level} must be equal to or below
the manifesting character's \define{psionic threshold}.
The psionic threshold is determined by the number of levels
a character has in classes with psionics.
If the power satisfies this test,
it can be manifested if they have sufficient \define{psi dice}
to expend.
The manifesting character will also need to note their
current \define{mental bandwidth} and whether or not manifesting
the power will exceed this bandwidth.
Each of these concepts are explained below.

\subsection{Psionic Threshold}
The psionic threshold represents the strongest powers
that a psionics user can learn and manifest.
The psionic threshold is determined by a character's class
and level,
as shown in the class tables of \secref{chap:classes}.
When levelling,
a character can only learn powers with a level up to
their psionic threshold.
This is explained in more detail in the same chapter.

Each power has a level written in its description.
If a character attempts to manifest a power with level
greater than their psionic threshold,
the power fails to manifest,
however no cost is incurred in doing so,
see \secref{sub:psi_dice}.

\subsection{Mental Bandwidth}
Manifesting powers strains the mind and the body.
Mental bandwidth represents the amount of psionic strain that
a character can endure while using psionics,
and is determined by the character's psionic class and level,
see \secref{chap:classes}.
Mental bandwidth is divided into slots,
and each power takes up a different number of slots.

When a character manifests a power,
they note their total mental bandwidth
as determined by their class and level.
This is called the
\define{maximum mental bandwidth}.
They then subtract from the maximum mental bandwidth
any slots which are taken up by powers they are already concentrating on,
see \secref{sub:concentration}.
The number determined after this subtraction is called the
\define{effective mental bandwidth}.
If the power to be manifested has a mental bandwidth
greater than the effective mental bandwidth,
the character must make a Constitution saving throw with
DC equal to 10 plus twice the number of slots
in excess of their effective bandwidth.

For example,
suppose Kadoth is a 10th level psion
and therefore has a maximum mental bandwidth of 6.
They are already concentrating on \power{adapt body},
which takes up 4 mental bandwidth slots,
so their effective mental bandwidth is 2.
They then attempt to manifest the mind slam power,
which takes up 3 slots.
Since this puts them in excess of their effective bandwidth by 1,
they make a DC 12 Constitution saving throw ($10 + 2 \times 1)$.

\subsubsection{Becoming Strained}
If a character fails this Constitution saving throw,
the power fails to manifest.
In addition, the manifester gains the \textbf{strained} condition.
The severity of the condition depends on the number of slots
that were in excess of the available bandwidth when the attempt
to manifest the power was made.
This is shown in the table below.
In Kadoth's example,
they were only in excess by 1,
and therefore would only incur the malus shown in the
first row of the table.
Note that the maluses are cumulative.
For example,
with two slots in excess,
a character incurs the malus from being in excess by 1
as well as the malus for being in excess by 2.

\begin{table}[htbp]%
    \begin{DndTable}[width=\columnwidth,
                     header=Strained Condition]{
                     X X}
        Slots in Excess & Cumulative Maluses \\
        1  & AC bonus from Dexterity reduced to 0 \\
        2  & Gain level of exhaustion \\
        3  & MB reduced by 1 \\
        4  & MB reduced by 1 \\
        5+ & Gain level of exhaustion
    \end{DndTable}
\end{table}

\subsubsection{Recovering from Strain}
A character loses the strained condition,
including all of the accumulated maluses,
when they finish a long rest.

In addition, during a short rest,
a character may spend hit dice to remove
cumulative levels of strain on a 1 to 1 basis.
For example,
if a character was three slots in excess of their
effective mental bandwidth when they attempted
to manifest a power,
and as a result they have accrued the first three
rows of maluses in the table above,
they can spend one hit die to remove the malus
from the third row,
another to remove the malus from the second row,
and a final one to remove the malus from the first row.
In this latter case, the strained condition
is removed altogether. 

\subsection{Psi Dice}
\label{sub:psi_dice}
When manifesting a power,
a character must expend a number of psi dice
equal to the number stated in the power's description.
The number of psi dice a character has is determined
by their class and level, see \secref{chap:classes}.
If a character has less psi dice than the total psi dice cost
of the power, the power fails (but no dice are expended).

The total base psi dice cost of a power is determined by the its level.
The cost is equal to one less than double the power's level.
This is shown in the table below.
For example, a 1st level power incurs a cost of 1 psi die,
whereas a 5-th level power incurs a cost of 9 psi dice.
\begin{table*}[htbp]%
    \begin{DndTable}[width=\textwidth,
                     header=Psi Dice Cost by Level]{
                     X X X X X X X X X X}
         Level         & 1 & 2 & 3 & 4 & 5 & 6  & 7  & 8  & 9 \\
        \textbf{Cost}  & 1 & 3 & 5 & 7 & 9 & 11 & 13 & 15 & 17
    \end{DndTable}
\end{table*}

Some powers can be \define{augmented} with extra psi dice, in which
case they usually have a stronger effect, see \secref{sub:augmenting}.
If a power can be augmented,
this will be explicitly stated in its description.
The cost for augmenting a power is also given in its description.

\subsection{Concentration}
\label{sub:concentration}
If stipulated in the power description,
a power requires uninterrupted concentration
in order to be continually used.
There is no limit to the number of powers that may be
simultaneously concentrated on,
however each power concentrated on will take away from the
available mental bandwidth slots, as mentioned above.

Whenever a manifester takes damage while
concentrating on any number of powers,
they must make a Constitution saving throw.
The DC of the save is either equal to 10 or half the damage taken,
whichever is higher.
Each separate source of damage prompts a separate save.
If the save fails,
the manifester loses concentration on all the powers they were
concentrating on.
In addition, the DM may call the manifester to make a
save with any DC in a circumstance that would
jeopardise the manifester's concentration,
like being thrown off a flying broom or
being battered by an intense blizzard.

A manifester may at any time on their turn
stop concentrating on any power.
However, a manifester will automatically stop concentrating on a power
if they become incapacitated or if they die.

\section{Power Attributes}
In \secref{sec:list_of_powers},
the available powers are given.
In this section,
the terminology and structure of the power descriptions are explained.

\subsection{Level and Discipline}
Beneath the name of the power,
the power level and discipline are given.
Each power is associated with a discipline,
of which there are six.
These are described below.

\subsubsection{Prescience}
Strictly, the discipline of prescience relates to psionic powers
which allow the manifester to peer through time and see future events.
Prescient manifesters describe the future as
an undulating cloth blowing in a swift breeze,
with the hills and valleys of its surface representing
future possibilities. 

However, in more recent times,
the discipline of prescience has incorporated additional tenets,
and its adherents are also characterised by an uncanny ability to
understand the past and know things not normally known to
the untrained mind.

\subsubsection{Prana Bindu}
The discipline of Prana Bindu stresses that the mind and body
must work in close concert with each other.
Adherents have a profound mastery of nerve and muscle,
and can react to situations with speeds nearly imperceptible
to the untrained eye.
The discipline also allows one to alter their internal equilibrium
and metabolism at will,
granting the body the ability to heal and adapt to hostile environments.

The discipline of Prana Bindu is especially favoured by Psi Knights,
powerful warriors who fuse martial and mental prowess.
See \secref{sec:psi_knight} for details on the Psi Knight class.

\subsubsection{Voice}
The Voice is a dangerous but supremely powerful tool
used to dominate the minds and wills of others.
Adherents hone their speech in such a way that,
only by subtle changes in pitch, volume and speed,
their Voice can control the behaviour of others.

In addition, adherents to the discipline can manifest their Voice
in the minds of others, allowing them to telepathically
communicate with other creatures.

\subsubsection{Metacreativity}
The discipline of metacreativity allows the manifester to mould
matter as they desire,
turning the mundane into useful tools or deadly weapons.
A metacreator can also in this manner craft constructs
which bend to their mind,
allowing myriad servants to carry out their bidding.

\subsubsection{Spacefolding}
A spacefolder warps the fabric of space and time to
transport themselves and others.
This also allows them to move effortlessly in conditions
that would otherwise impede or slow down movement. 
Many powerful spacefolders can even cross the gap between
the planes themselves.

\subsubsection{Psychokinesis}
The discipline of psychokinesis allows one to transform
their raw mental energy into destructive power,
unleashing havoc on the battlefield.
Psychokineticists can deal terrifying amounts of damage
and leave their enemies trembling in awe and fear.

\subsection{Manifesting Time}
Each power has a \define{manifesting time},
which may be a bonus action, action, or reaction.
Unlike spellcasting,
there is no level restriction to manifesting powers
on the same turn where one has been manifested
with a bonus action. 

In addition, some powers may have longer casting times,
in which case on each round the manifester must use their action
to manifest the power while maintaining concentration.
While the manifester is maintaining concentration in this way,
the power will take up the specified number
of mental bandwidth slots given in the power's description.
If their concentration is broken, the power fails,
but no psi dice are expended,
and the mental bandwidth slots are freed up.

\subsection{Range}
A power's \define{range} indicates the extent
to which its effects can reach when manifested.
The \define{target} of the power must be within range.

There are three types of ranges: self, touch and a range
specified in feet.
Powers with a range of self will always target the manifester.

\subsubsection{Targets}
When selecting a target for a power,
in addition to the range requirement mentioned above,
the manifester must have \define{line of sight}
towards the target.
The meaning line of sight has two prongs:
\define{sight} and \define{uninterruption}.
Respectively, these mean that the manifester must be able to
(a) see the target
and (b) draw a line to the target
without that line passing through an obstacle,
even if the target can be seen through that obstacle.

Powers with a range of self (powers that target the manifester)
are not bound by this requirement.
In addition, a power may mention that the manifester does not need
line of sight towards the target,
or that the manifester does not need either of the
two prongs of line of sight. 

\subsection{Cost}
The \define{cost} specifies the mental bandwidth and psi die
that must be expended to manifest the power.
For example,
`MB 1, PD 1'
means that the power takes up 1 slot of mental bandwidth
and costs one psi die to manifest.

\subsection{Duration}
The \define{duration} of a power is how long the effects
of a power remain active.
Powers can be instantaneous,
while others can last for any length of time
as specified in the powers description.

If the power requires concentration throughout its duration,
this will be indicated alongside the duration.

\subsection{Areas of Effect}
Some powers specify in their description an area of effect
over which the power manifests.
In this case, the rules governing these areas of effect
are identical to those of spellcasting, which begin at
phb 202.

\subsection{Affecting Targets}
Some powers specify that the manifester makes an attack
roll against each target,
whereas some specify that all targets need to make
a type of saving throw.

If the power specifies an attack roll,
then the manifester rolls a d20
and adds their \define{psionic attack modifier},
which is their \define{psionic ability modifier} (determined by class)
plus their proficiency bonus.

If the power specifies a saving throw,
then each target must succeed on a saving throw
with DC equal to 8 + the manifester's psionic attack modifier.
The type of saving throw
is specified in the power's description.

\subsection{Augmenting Powers}
\label{sub:augmenting}
Some powers will indicate that they can be \define{augmented},
which is analogous to upcasting for a spellcaster.
When augmenting a power,
the manifester spends additional psi dice on top of the cost
incurred from the level of the power itself
(see \secref{sub:psi_dice} for the base cost by power level).

The cost required to augment a power will be specified
in the power's description.
For example, the description might mention
`For every two additional psi psi dice you spend\dots'
In this case, the cost to augment the power is 2 psi dice.
Each time the power is augmented,
its level increases by 1.
In the example above,
spending an additional 6 psi dice increases the power's level by 3.

A power cannot be augmented to a level beyond the manifester's
psionic threshold.
For example,
if a manifester has a psionic threshold of 3,
they may spend 4 psi dice to augment a power from
1st to 3rd level with an augment cost of 2,
but may not go beyond 3rd level i.e. spend beyond 4 psi dice.

\chapter{List of Powers}
\label{sec:list_of_powers}

\section{Class Lists}
\subsection{Psion}
\begin{dndlongtable}[
    >{\raggedright\arraybackslash}p{0.22\linewidth} c p{0.55\linewidth}]
    Name & Level & Short Description \\
\nameref{pwr:attraction} & 1 & Incept an item of attraction in a creature \\
\nameref{pwr:blow_up_object_with_mind} & 1 & Blow up object, damaging those nearby \\
\nameref{pwr:conceal_thoughts} & 1 & Conceal motives of target \\
\nameref{pwr:create_armour_with_mind} & 1 & Create any armour from mundane matter \\
\nameref{pwr:create_sound} & 1 & Create unique sound \\
\nameref{pwr:create_weapon_with_mind} & 1 & Create any weapon from mundane matter \\
\nameref{pwr:cushion_fall} & 1 & Reduce fall damage \\
\nameref{pwr:deceleration} & 1 & Reduce target's speed \\
\nameref{pwr:demoralise} & 1 & Enemies suffer malus to attacks and saving throws \\
\nameref{pwr:detect_psionics} & 1 & Detect nearby presence of psionic powers and effects \\
\nameref{pwr:diamond_bullet} & 1 & Fire diamond bullet at target \\
\nameref{pwr:dissipating_touch} & 1 & Deal force damage to target on touch \\
\nameref{pwr:distract} & 1 & Give target disadvantage to perception and insight \\
\nameref{pwr:déjà_vu} & 1 & Make target repeat previous actions \\
\nameref{pwr:empaths_understanding} & 1 & Detect target's emotions to gain bonus to social checks \\
\nameref{pwr:energy_ray} & 1 & Strike target with ray of energy of chosen type \\
\nameref{pwr:float} & 1 & Float in water and gain swim speed \\
\nameref{pwr:force_screen} & 1 & Gain bonus to AC \\
\nameref{pwr:hammer} & 1 & Deal 1d12 with unarmed strikes \\
\nameref{pwr:identify_psionics} & 1 & Identify psionic powers and items \\
\nameref{pwr:inertial_armour} & 1 & Gain bonus to AC \\
\nameref{pwr:matter_agitation} & 1 & Heat matter \\
\nameref{pwr:mind_thrust} & 1 & Deal psychic damage to target \\
\nameref{pwr:missive} & 1 & Send telepathic message to creature \\
\nameref{pwr:orient_self} & 1 & Learn where you are \\
\nameref{pwr:precognition} & 1 & Gain slight bonus to defence or offence \\
\nameref{pwr:psi_hand} & 1 & Move small objects at short distance \\
\nameref{pwr:recall} & 1 & Repeat check to recall details \\
\nameref{pwr:sense_link} & 1 & Perceive through one of the target's senses \\
\nameref{pwr:shockwave} & 1 & Pummel target behind cover with force damage \\
\nameref{pwr:skate} & 1 & Skate along ground and move faster \\
\nameref{pwr:stillness_of_mind} & 1 & Gain bonus to wisdom save as reaction \\
\nameref{pwr:vigour} & 1 & Gain temporary hit points \\
\nameref{pwr:bestow_power} & 2 & Grant a creature psi dice \\
\nameref{pwr:biofeedback} & 2 & Reduce incoming damage \\
\nameref{pwr:body_equilibrium} & 2 & Walk on otherwise untraversable surfaces \\
\nameref{pwr:cloud_mind} & 2 & Erase yourself from target's mind \\
\nameref{pwr:concealing_membrane} & 2 & Become lightly obscured in membrane \\
\nameref{pwr:control_sound} & 2 & Mould existing sound as you desire \\
\nameref{pwr:detect_hostile_intent} & 2 & Detect any nearby creatures that are hostile to you \\
\nameref{pwr:energy_push} & 2 & Push target back with wave of energy of chosen type \\
\nameref{pwr:hyperphantasm} & 2 & Create an illusory image in the mind of another \\
\nameref{pwr:inflict_pain} & 2 & Target has disadvantage on attack rolls and skill checks \\
\nameref{pwr:levitate} & 2 & Levitate yourself \\
\nameref{pwr:psi_vision} & 2 & Gain darkvision \\
\nameref{pwr:psionic_step} & 2 & Teleport a short distance \\
\nameref{pwr:psychic_beacon} & 2 & Know distance and location of marked target \\
\nameref{pwr:reveal_agony} & 2 & Deal psychic damage to target \\
\nameref{pwr:sense_link_forced} & 2 & Forcibly perceive through one of the target's senses \\
\nameref{pwr:stunning_wave} & 2 & Daze creatures surrounding you \\
\nameref{pwr:swarm_of_crystals} & 2 & Slice targets with cone of crystals which always hit \\
\nameref{pwr:confusion_psionic} & 3 & Target behaves uncontrollably \\
\nameref{pwr:counterpower} & 3 & Interrupt the manifesting of a power \\
\nameref{pwr:danger_sense} & 3 & Reduce susceptibility to traps \\
\nameref{pwr:dispel_psionics} & 3 & Cancel psionic powers and effects \\
\nameref{pwr:energy_bolt} & 3 & Strike targets with line of energy of chosen type \\
\nameref{pwr:energy_burst} & 3 & Produce burst of energy of chosen type centred on yourself \\
\nameref{pwr:manipulate_object_with_mind} & 3 & Move object telekinetically \\
\nameref{pwr:rapid_healing} & 3 & Heal a small number of hit points \\
\nameref{pwr:sacrificial_shell} & 3 & Gain immediate protection from energy \\
\nameref{pwr:self_alchemy} & 3 & Alter your physical appearance \\
\nameref{pwr:share_pain} & 3 & Half of the damage you suffer is carried to the target \\
\nameref{pwr:time_warp} & 3 & Move target briefly forward in time \\
\nameref{pwr:touchsight} & 3 & Perceive surroundings by touch \\
\nameref{pwr:correspond} & 4 & Communicate with a creature that is anywhere \\
\nameref{pwr:detect_prying_eyes} & 4 & Learn who is attempting to scry you \\
\nameref{pwr:drain_power} & 4 & Drain targets psi dice \\
\nameref{pwr:empathic_feedback} & 4 & Target takes damage when they hit you \\
\nameref{pwr:manipulate_creature_with_mind} & 4 & Shove, disarm or grapple creature telekinetically \\
\nameref{pwr:mind_palace_power} & 4 & Shelter allies from psionic powers \\
\nameref{pwr:psi_door} & 4 & Teleport over a moderate distance \\
\nameref{pwr:psi_wall} & 4 & Create large protective barrier \\
\nameref{pwr:trace_teleport} & 4 & Learn location targets teleported to \\
\nameref{pwr:adapt_body} & 5 & Adapt your body to dangerous environments \\
\nameref{pwr:alter_ego} & 5 & Creating a copy of yourself \\
\nameref{pwr:create_object_with_mind} & 5 & Create object up to certain size \\
\nameref{pwr:ironflesh} & 5 & Harden your body and improve its resistance to damage \\
\nameref{pwr:plane_shift_psionic} & 5 & Move targets across the planes \\
\nameref{pwr:psychic_crush} & 5 & Deal large amount of psychic damage to creature \\
\nameref{pwr:shatter_mind_blank} & 5 & Cancels the effects of mind blank \\
\nameref{pwr:third_eye} & 5 & Gain truesight out to 120 feet \\
\nameref{pwr:tower_of_power} & 5 & Grant you and allies significant resistance to Voice powers \\
\nameref{pwr:affinity_field} & 6 & Transfer damage, healing and powers onto others within the field \\
\nameref{pwr:breath_of_numinex} & 6 & Powerful breath weapon dealing acid damage \\
\nameref{pwr:cloud_mind_mass} & 6 & Erase yourself from many targets' mind \\
\nameref{pwr:mind_blank_psionic} & 6 & Conceal targets mind from external forces \\
\nameref{pwr:power_pirate} & 6 & Take control of another manifester's active power \\
\nameref{pwr:remote_viewing_trap} & 6 & Deal damage to those who attempt to scry you \\
\nameref{pwr:retrieve} & 6 & Teleport item into your hand \\
\nameref{pwr:divert_teleport} & 7 & Change destination of target's teleportation \\
\nameref{pwr:energy_transformation} & 7 & Convert energy damage into jets of plasma \\
\nameref{pwr:energy_wave} & 7 & Destructive cone of chosen energy type \\
\nameref{pwr:duplicate_power} & 8 & Recreate the effect of powers potentially unknown to you \\
\nameref{pwr:timeless_body} & 9 & Become impervious to all effects
\end{dndlongtable}

\columnbreak\subsection{Psi Knight}
\begin{dndlongtable}[
    >{\raggedright\arraybackslash}p{0.22\linewidth} c p{0.55\linewidth}]
    Name & Level & Short Description \\
\nameref{pwr:bite-of-the-wolf} & 1 & Gain bite attack \\
\nameref{pwr:blow-up-object-with-mind} & 1 & Blow up object, damaging those nearby \\
\nameref{pwr:bolt} & 1 & Create ammunition \\
\nameref{pwr:boots-of-the-demon-king} & 1 & Kick target with lightning, can knock prone or send target flying \\
\nameref{pwr:burst} & 1 & Gain speed boost as free action \\
\nameref{pwr:chameleon} & 1 & Blend in with environment, bonus to stealth \\
\nameref{pwr:conceal-thoughts} & 1 & Conceal motives of target \\
\nameref{pwr:create-armour-with-mind} & 1 & Create any armour from mundane matter \\
\nameref{pwr:create-weapon-with-mind} & 1 & Create any weapon from mundane matter \\
\nameref{pwr:cushion-fall} & 1 & Reduce fall damage \\
\nameref{pwr:detect-psionics} & 1 & Detect nearby presence of psionic powers and effects \\
\nameref{pwr:dissipating-touch} & 1 & Deal force damage to target on touch \\
\nameref{pwr:distract} & 1 & Give target disadvantage to perception and insight \\
\nameref{pwr:envenom} & 1 & Weapon is coated with corrosive bile \\
\nameref{pwr:flame-choke} & 1 & Grapple a creature and deal fire damage over time \\
\nameref{pwr:float} & 1 & Float in water and gain swim speed \\
\nameref{pwr:force-screen} & 1 & Gain bonus to AC \\
\nameref{pwr:grip-of-iron} & 1 & Gain bonus to grapple checks \\
\nameref{pwr:hammer} & 1 & Deal 1d12 with unarmed strikes \\
\nameref{pwr:inertial-armour} & 1 & Gain bonus to AC \\
\nameref{pwr:precognition} & 1 & Gain slight bonus to defence or offence \\
\nameref{pwr:shockwave} & 1 & Pummel target behind cover with force damage \\
\nameref{pwr:skate} & 1 & Skate along ground and move faster \\
\nameref{pwr:stillness-of-mind} & 1 & Gain bonus to wisdom save as reaction \\
\nameref{pwr:thicken-skin} & 1 & Gain bonus to AC \\
\nameref{pwr:vigour} & 1 & Gain temporary hit points \\
\nameref{pwr:biofeedback} & 2 & Reduce incoming damage \\
\nameref{pwr:body-equilibrium} & 2 & Walk on otherwise untraversable surfaces \\
\nameref{pwr:concealing-membrane} & 2 & Become lightly obscured in membrane \\
\nameref{pwr:detect-hostile-intent} & 2 & Detect any nearby creatures that are hostile to you \\
\nameref{pwr:dimension-swap} & 2 & Swap positions between yourself and ally or two allies \\
\nameref{pwr:dissolving-touch} & 2 & Deal acid damage to creature you hit with attack \\
\nameref{pwr:kinaesthetic-inversion} & 2 & Invert a creature's sense of direction \\
\nameref{pwr:leap} & 2 & Gain fly speed on your turn \\
\nameref{pwr:levitate} & 2 & Levitate yourself \\
\nameref{pwr:psi-vision} & 2 & Gain darkvision \\
\nameref{pwr:psionic-step} & 2 & Teleport a short distance \\
\nameref{pwr:spacetime-redirection} & 2 & Redirect an enemy's attack to another creature \\
\nameref{pwr:swiftsteed} & 2 & Double movement speed \\
\nameref{pwr:trap-clairvoyance} & 2 & Forsee danger from hidden traps \\
\nameref{pwr:tunnel-vision} & 2 & Restrict vision of a creature \\
\nameref{pwr:uncanny-reach} & 2 & Increase your reach with unarmed attacks \\
\nameref{pwr:wall-walker} & 2 & Walk on walls and ceilings \\
\nameref{pwr:danger-sense} & 3 & Reduce susceptibility to traps \\
\nameref{pwr:fist-of-fury} & 3 & Critical on 18--20 with unarmed attacks \\
\nameref{pwr:manipulate-creature-with-mind} & 3 & Shove, disarm or grapple creature telekinetically \\
\nameref{pwr:rapid-healing} & 3 & Heal a small number of hit points \\
\nameref{pwr:resuscitation} & 3 & Return creature that has just died to life \\
\nameref{pwr:sacrificial-shell} & 3 & Gain immediate protection from energy \\
\nameref{pwr:steadfast-perception} & 3 & See through visual deceptions \\
\nameref{pwr:unrelenting-voice} & 3 & Wave of force slams targets, pushing them back and knocking them over \\
\nameref{pwr:empathic-feedback} & 4 & Target takes damage when they hit you \\
\nameref{pwr:immovability} & 4 & Make yourself harder to move involuntarily \\
\nameref{pwr:inertial-dampener} & 4 & Become immune to bludgeoning damage \\
\nameref{pwr:psi-door} & 4 & Teleport over a moderate distance \\
\nameref{pwr:weapon-of-energy} & 4 & Imbue weapon with energy damage \\
\nameref{pwr:adapt-body} & 5 & Adapt your body to dangerous environments \\
\nameref{pwr:ironflesh} & 5 & Harden your body and improve its resistance to damage \\
\nameref{pwr:psychofeedback} & 5 & Alter ability scores temporarily \\
\nameref{pwr:breath-of-numinex} & 6 & Powerful breath weapon dealing acid damage \\
\nameref{pwr:mind-blank-psionic} & 6 & Conceal targets mind from external forces \\
\nameref{pwr:suspend-life} & 6 & Slow metabolism and appear dead
\end{dndlongtable}

\section{Discipline Sublists}


\clearpage\section{Power Descriptions}
% P5, PK5
\DndPowerHeader%
  {Adapt Body}
  {5th-level Prana Bindu}
  {1 action}
  {Self}
  {MB 4, PD \lvlfive}
  {Concentration, up to 1 hour}
Your body automatically adapts to hostile environments.
You can adapt to underwater, extremely hot, extremely cold,
or airless environments, allowing you to survive
as if you were a creature native to that environment.
You can breathe and move (though penalties to movement and attacks,
if any for a particular environment, remain),
and you take no damage simply from being in that environment.
You need not specify what environment you are adapting to
when you manifest this power;
simply activate it,
and your body will instantly adapt to any hostile environment
as needed throughout the duration.

You can somewhat adapt to extreme environmental features such as acid,
lava, fire, and lightning.
Any environmental feature that normally directly deals
1 or more dice of damage per round
deals you only half the usual amount of damage.

In any case,
you have advantage on Constitution saving throws to maintain
concentration on powers if damage or an otherwise
hazardous effect from the environment you are adapting to
prompted the save.

% P
\DndPowerHeader%
  {Attraction}
  {1st-level Voice}
  {1 action}
  {30 feet}
  {MB 1, PD \lvlone}
  {Concentration, up to 1 hour}
Your Voice plants a compelling attraction
in the mind of a creature
that you can see and you.
The attraction can be toward a particular person or an object.

The target makes a Wisdom saving throw.
If the target fails, they will take reasonable steps to meet,
get close to, attend, or find the object of its implanted attraction.
Otherwise, the power has no effect.

For the purpose of this power, `reasonable' means that, while attracted,
the target doesn't suffer from blind obsession.
They will act on this attraction only when not engaged in combat.
The target won't perform obviously suicidal actions.
They can still recognize danger but will not flee
unless the threat is immediate.

If you make the target feel an attraction to yourself,
you can't command them indiscriminately,
although they will be willing to listen to you (even if they disagree).
Accordingly, this power grants you advantage on any social skill checks
you make involving the target.
\augment{For every 2 additional psi dice you spend,
  this power's save DC increases by 1
  and the bonus on social skill checks increases by 1.}

\DndPowerHeader%
  {Aversion}
  {2nd-level Voice}
  {1 action}
  {30 feet}
  {MB 2, PD \lvltwo}
  {Concentration, up to 1 hour}
Your Voice plants a powerful aversion in the mind of
a creature you can see in range.
If the object of the implanted aversion is an individual
or a physical object,
the target will prefer not to approach within 30 feet of it.
If it is a word,
they will try not to utter it;
if it is an action,
they will not willingly attempt to perform it;
and if it is an event,
they will not willingly attend it.
The subject will take reasonable steps to avoid the object of its aversion,
but will not put themselves in jeopardy by doing so.

If the subject is forced into taking an action they have an aversion to,
they have disadvantage on any attack rolls,
ability checks,
or skill checks involved.
\augment{For every 2 additional psi dice you spend,
  this power's save DC increases by 1
  and the duration increases by 1 hour.}

\DndPowerHeader%
  {Banishment, Psionic}
  {4th-level Spacefolding}
  {1 action}
  {120 feet}
  {MB 4, PD \lvlfour}
  {Instantaneous}
As the \spell{banishment} spell, except as noted here.
\augment{For every 3 additional psi dice you spend,
  you can target one additional creature.}

\DndPowerHeader%
  {Bestow Power}
  {2nd-level Voice}
  {1 action}
  {20 feet}
  {MB 2, PD \lvltwo}
  {Instantaneous}
You Voice creates a link between your own mind
and that of another psionic creature mind you can see,
creating a brief conduit
through which mental energy can be shared.
When you manifest this power,
the subject gains up to 2 psi dice.
You can then transfer psi dice to a subject
on a 1 to 1 basis up to their psionic threshold.

\DndPowerHeader%
  {Biofeedback}
  {2nd-level Prana Bindu}
  {1 action}
  {Self}
  {MB 2, PD \lvltwo}
  {Concentration, up to 1 minute}
  You can toughen your body against wounds,
  lessening their impact.
  For the duration of this power, all incoming damage
  from separate sources is reduced by 2.
\augment{For every 3 additional psi dice you spend,
the amount of damage reduced increases by 1.}

% PK1
\DndPowerHeader%
  {Bite of the Wolf}
  {1st-level Prana Bindu}
  {1 action}
  {Self}
  {MB 1, PD \lvlone}
  {Concentration, up to 1 minute}
You temporarily alter your genomic makeup
such that your posture becomes stooped forward,
and you grow a muzzle complete with fangs.
You gain one bite attack each round which
can be used as a bonus action or as a replacement
for one of your attacks.
Treat the bite as a melee weapon attack.
The bite deals 1d8 piercing damage
(assuming you are a Medium creature)
when it hits.
Your bite can be affected by
powers, spells, and effects
that enhance or improve weapons.

If you are not a Medium creature,
your bite attack's base damage varies as follows:
Tiny 1d4; Small 1d6; Large 2d6; Huge 2d8;
Gargantuan 4d6.

Based on your psi knight level,
your bite increases in ferocity as noted here:
at 5th level your bite deals an extra 1d8 points of damage,
at 10th level an extra 2d8,
at 15th level an extra 3d8,
and at 20th level an extra 4d8 points.

\DndPowerHeader%
  {Blow Up Object With Mind}
  {1st-level Psychokinesis}
  {1 action}
  {60 feet}
  {MB 1, PD \lvlone}
  {Instantaneous}
Your terrifying mental ability can cause matter to
wrench itself apart. 
Choose an object within range that you can see.
The object can be carried or worn by a creature.
Make a psionics attack roll against the object.
The AC is determined by the size of object, shown below.
\begin{table}[htbp]%
  \begin{DndTable}[width=\columnwidth,
                   header=Object AC and Damage]{X c c}
      Size & AC & Damage Die \\
      Tiny (bottle, pancakes)        & 11 & 2d6 \\
      Small (chest, lute)            & 13 & 2d8 \\
      Medium (barrel, chandelier)    & 15 & 2d10 \\
      Large (cart, 10x10 ft. window) & 17 & 2d12
  \end{DndTable}
\end{table}

If the attack hits, the object explodes.
Any creature within 10 feet of the object must make
a Dexterity saving throw.
If they fail,
they take bludgeoning damage equal to
the damage shown in the table below,
as dependent on the object's size.
If the object was being worn or carried by a creature,
that creature has disadvantage on the saving throw.
\augment{For every 2 additional psi dice you spend,
  roll one extra die using the same damage die as
  given for the object's size.}

% P2, PK2
\DndPowerHeader%
  {Body Equilibrium}
  {2nd-level Prana Bindu}
  {1 action}
  {Self}
  {MB 2, PD \lvltwo}
  {Concentration, up to 10 minutes}
You can adjust your body's equilibrium to correspond
with any solid or liquid that you stand on.
Thus, you can walk on water, quicksand,
or even a spider's web without sinking or breaking through
(this effect does not confer any resistance to particularly sticky webs).
You can move at your normal speed,
but you cannot take the Dash action
on an non-firm surface without sinking or breaking through.

If you fall from any height while using this power,
damage from the impact is halved.

\DndPowerHeader%
  {Bolt}
  {1st-level Metacreativity}
  {1 action}
  {Touch}
  {MB 1, PD \lvlone}
  {Concentration, up to 1 hour}
You create 2d4 crossbow bolts, arrows or sling bullets.
The ammunition dissipate upon hitting a target
or when the power ends.
The ammunition has a +1 bonus to attack and damage rolls.
\augment{For every 4 additional psi dice you spend,
  this power improves the ammunition's bonus to
  attack and damage rolls by +1,
  up to a maximum of three for each.}

% PK1
\DndPowerHeader%
  {Boots of the Demon King}
  {1st-level Prana Bindu}
  {1 action}
  {Self (5-foot radius)}
  {MB 1, PD \lvlone}
  {Instantaneous}
Your legs briefly gain incredible strength,
and your boots crackle with lightning.
You make a kick (unarmed attack) against an
opponent within your melee range.
If the attack hits,
the target takes damage from your unarmed strike
as normal plus 2d4 lightning damage.
If the target is above half their hit point maximum,
they must succeed on a Strength saving throw or be knocked prone.
If the target is below half their hit point maximum,
they must succeed on a Strength saving throw or immediately
be sent a distance into the air equal to your Strength score.
They then fall prone and take fall damage as normal.
\augment{For every additional psi die you spend,
  the lightning damage dealt by this power increases by 1d4.}

  % P6, PK6
\DndPowerHeader%
  {Breath of Numinex}
  {6th-level Prana Bindu}
  {1 action}
  {Self (30-foot cone)}
  {MB 5, PD \lvlsix}
  {Instantaneous}
Your alter your body's chemistry in an instant,
allowing you to secrete a vitriolic acid
which you spew forth in a 30-foot cone
centred on yourself.
All creatures inside the cone must make a
Dexterity saving throw,
taking 11d6 acid damage on a failed save
or half as much on a successful save. 
\augment{For every additional psi dice you spend,
this power's damage increases by 1d6.}

\DndPowerHeader%
  {Burst}
  {1st-level Spacefolding}
  {Free action, see text}
  {Self}
  {MB 1, PD \lvlone}
  {Until the end of your turn}
You subtly alter the space between matter
to gain a quick boost to your perceived speed.
At any point in your turn,
you may use a free action to manifest this power
and gain an extra 10 feet of base movement speed
for that turn only.

% PB2
\DndPowerHeader%
  {Chameleon}
  {1st-level Prana Bindu}
  {1 action}
  {Self}
  {MB 1, PD \lvlone}
  {Concentration, up to 1 hour}
Your skin and equipment take on the color and texture
of nearby objects,
including floors and walls.
You gain the ability to hide in dim light
while the power lasts.
You may also attempt to hide in normal levels
of illumination,
however your Dexterity (Stealth) check has disadvantage.

\DndPowerHeader%
  {Cloud Mind}
  {2nd-level Voice}
  {1 action}
  {30 feet}
  {MB 2, PD \lvltwo}
  {Concentration, up to 1 minute}
Your make yourself completely undetectable to a target
you can see
by erasing all awareness of your presence from its mind.
If the target fails a Wisdom saving throw,
this power has the following effects.

First, you are invisible and inaudible to the creature.
It cannot even detect your presence by means of blindsight,
scent, or tremorsense.
It cannot pinpoint your location by any means.
\textit{Exception:} Creatures with truesight are still
able to discern your location,
and therefore the power has no effect on them.
  
Second, the target remains unaware of your actions,
provided you do not make any attacks or cause any obvious
or directly threatening changes in the target's environment.\
If you attack the target creature, the power ends.
  
If you take an action that creates a sustained and obvious change
in the target's environment---for example, attacking a creature
aside from the target or moving a large or attended object
the target can see---the target immediately repeats the
saving throw against the power.
An ally of the target creature that is able to see or perceive you
can warn the target and thereby grant it another saving throw.

\DndPowerHeader%
  {Cloud Mind, Mass}
  {6th-level Voice}
  {1 action}
  {30 feet}
  {MB 6, PD \lvlsix}
  {Concentration, up to 1 minute}
As the \power{cloud mind} power,
except as noted here.
Each creature within a 30-foot radius
is affected individually.

\DndPowerHeader%
  {Compel}
  {4th-level Voice}
  {1 action}
  {60 feet}
  {MB 4, PD 7}
  {Concentration, up to 1 minute}
Your Voice compels a creature to act according to your wishes.
Choose a creature within range that can hear and understand you.
You then suggest a course of activity
limited to no more than five words
that that creature is to take.
The course of action can be harmful to the creature
or contrary to its intentions.

The target must make a Wisdom saving throw.
The target has advantage on the saving throw
if the course of action is plainly harmful
or against the creature's wishes.
If the target fails,
it pursues the course of action to the best of its ability.
The compelled course of action can continue for the
entire duration of the power.
If the compelled activity can be completed in a shorter time,
the power ends when the target finishes what it was asked to do.
  
You can also specify conditions that will trigger
a special activity during the duration.
For example, you might suggest that a knight
give her warhorse to the first beggar she meets.
If the condition isn't met before the power expires,
the activity isn't performed.

% P2, PK 2
\DndPowerHeader%
  {Concealing Membrane}
  {2nd-level Metacreativity}
  {1 action}
  {Self}
  {MB 2, PD \lvltwo}
  {Concentration, up to 1 minute}
You weave a quasi-real membrane around yourself,
which makes you lightly obscured.
You remain visible within the translucent, amorphous enclosure.
This distortion also grants you half cover
thanks to the rippling membrane encasing your form.
You can pick up or drop objects, easily reaching through the film.
Anything you hold is enveloped by the membrane.
Likewise, you can engage in melee, make ranged attacks,
and manifest powers without hindrance.

% P1, PK1
\DndPowerHeader%
  {Conceal Thoughts}
  {1st-level Voice}
  {1 action}
  {30 feet}
  {MB 1, PD \lvlone}
  {Concentration, up to 1 hour}
Choose a target within range that you can see.
You protect the target's thoughts from analysis.
While the duration lasts, the target gains advantage on
Deception checks against those attempting to discern their
true intentions with Wisdom (Insight).
They also gain advantage on saving throws
against any power or spell
used to read their mind (such as detect thoughts).

\DndPowerHeader%
  {Confusion, Psionic}
  {4th-level Voice}
  {1 action}
  {90 feet}
  {MB 3, PD \lvlfour}
  {Concentration, up to 1 minute}
As the \spell{confusion} spell, except as noted here.
Your Voice targets one creature you can see within range.
\augment{For every 2 additional psi dice you spend,
  this power's save DC increases by 1,
  and the power can affect an additional target.
  Any additional target canot be more than 15 feet from another
  target of the power.}

\DndPowerHeader%
  {Control Air}
  {2nd-level Psychokinesis}
  {1 action}
  {120 feet}
  {MB 2, PD \lvltwo}
  {Concentration, up to 1 minute}
You have some control over wind speed and direction.
The speed of the wind within 120-feet of you
can be increased or decreased by up to 10 knots.
This power also gives you the ability to alter the direction
of the wind by as much as 90 degrees.

Powerful enough winds can cause creatures to be knocked prone.
Creatures of size Small or smaller that are subject
to winds between 25--45 knots
must succeed on a DC 10 Strength saving throw or be knocked prone.
If they wind speed is greater than 45 knots,
this applies to Medium creatures as well.
  \augment{For every additional psi die you spend,
      you can modify the wind speed
      by an additional 10 miles per hour,
      to a maximum change in wind speed of
      50 knots.}

\DndPowerHeader%
  {Control Body}
  {4th-level Psychokinesis}
  {1 action}
  {60 feet}
  {MB 4, PD \lvlfour}
  {Concentration, up to 1 minute}
You psychokinetically control the actions of any humanoid
(including undead or outsiders with a humanoid physiology)
that you can see within range.
The target immediately makes a Strength saving throw,
with the power ending if they succeed
and continuing if they fail.
Control body doesn't require mental contact with the target,
since you are actually forcing limb movements independent
of the target's mind.
You can force the target to stand up,
sit down,
walk,
turn around,
and so on,
but operating the vocal cords is too difficult.
You can also hold the target immobile,
rendering it incapacitated.
You cannot force the target to manifest powers,
cast spells,
or use any special ability that is not a function
of just its body movements.
If you lose sight of the target,
the effect of this power ends.

If you force the target to engage in combat,
its attack bonus is equal to your
psionics attack bonus,
and its bonus on damage rolls is equal
to your Intelligence bonus.
A creature subject to this power cannot take reactions.
The target gains no benefit to its AC from its Dexterity,
but it does gain a bonus to its AC
equal to your Intelligence bonus.

Although the subject's body is under your control,
the subject's mind is not.
Creatures capable of taking purely mental actions
(such as manifesting powers)
can do so.
\augment{For every 2 additional psi dice you spend,
  this power can affect a target one size category larger.}

\DndPowerHeader%
  {Control Sound}
  {2nd-level Psychokinesis}
  {1 action}
  {60 feet}
  {MB 2, PD \lvltwo}
  {Concentration, up to 1 minute}
You shape and alter existing sounds.
You can target one sound,
such as a person speaking or singing,
or a group of related sounds,
such as the patter of many raindrops
or the tramp of soldiers passing by.
A sound as quiet as a snapping finger can be controlled.
You can substitute any sound you have heard
for the target sound.
If you attempt to exactly duplicate the voice
of a specific individual,
or an inherently terrifying sound (such as a dragon's roar),
you must succeed on a
Charisma (Deception) check with advantage
opposed by the intended listener's Wisdom (Insight) check
to avoid arousing suspicion.

You can entirely muffle a noise or magnify a sound
to such loudness that it drowns out
all other conversation in the immediate area.

Alternatively, you can use up the power in an instant.
You do this by modulating a sound into a one-time
destructive impetus that shatters nonmagical/nonpsionic,
unattended objects of crystal, glass, ceramics, or porcelain
(vials, bottles, flasks, jugs, mirrors, and so forth)
in the area.

\DndPowerHeader%
  {Correspond}
  {4th-level Voice}
  {10 minutes}
  {Unlimited, see text}
  {MB 4, PD \lvlfour}
  {Concentration, up to 1 minute}
You forge a mental link with a creature with which you have
previously had physical or mental contact
and that has an Intelligence score of 5 or greater.
You need not have line of sight towards the target,
nor need them to even be on the same
plane of existence as you are.
The subject recognizes you, and you can mentally communicate
with it for the duration
(though nothing forces the subject to respond to you),
exchanging messages of twenty-five words or less once per round.

\DndPowerHeader%
  {Create Object With Mind}
  {5th-level Metacreativity}
  {1 minute}
  {30 feet}
  {MB 4, PD \lvlfive}
  {See text}
As the \spell{creation} spell,
except as noted here.
\augment{For every 3 additional psi dice you spend,
the cube's size increases by 5 feet.}

\DndPowerHeader%
  {Create Sound}
  {1st-level Metacreativity}
  {1 action}
  {30 feet}
  {MB 1, PD \lvlone}
  {Concentration, up to 1 minute}
You create a volume of sound that rises, recedes, approaches,
or remains at a fixed place centred at a point within range.
You choose what type of sound
the power creates when manifesting it
and cannot thereafter change its basic character.
The volume of sound created,
however, depends on your psionic threshold.
You can produce as much noise as
four normal humans per psionic threshold (maximum twenty humans).
Thus, talking, singing, shouting, walking,
marching, or running sounds can be created.
The noise produced can be virtually any type of sound
within the volume limit.
A horde of rats running and squeaking
is about the same volume as eight humans running and shouting.
A roaring lion is equal to the noise from sixteen humans,
while a roaring dire lion is equal to the noise from twenty humans.

If you wish to create a specific message,
up to twenty-five words can be created,
and those words repeat over and over
until the duration expires or the power is dismissed.
If you attempt to exactly duplicate the voice
of a specific individual or an inherently terrifying sound
(such as a dragon's roar),
you must succeed on a Deception check opposed
by the listener's Wisdom (Insight) check
to avoid arousing suspicion.

Create sound can be used to bring sounds into existence
that you later manipulate by manifesting \power{control sound}.

\DndPowerHeader%
  {Create Weapon With Mind}
  {1st-level Metacreativity}
  {1 action}
  {Touch}
  {MB 1, PD \lvlone}
  {Concentration, up to 1 minute}
You touch any mundane matter in front of you,
morphing it into a weapon of your choosing
from the weapons table (phb 146).
The weapon is made of ordinary materials
as appropriate for its kind.
If you choose a projectile weapon,
it comes with 3d6 pieces of ammunition
of the applicable type.
If you relinquish your grip on the weapon you called
for 2 or more consecutive rounds,
it loses its coherence and returns to the substance(s)
it originated as.
\augment{You can augment this power in one or both of the follwing ways.
  \begin{itemize}
    \item For every 4 additional psi dice you spend,
      this power adds a +1 bonus to the weapon's attack and damage rolls
      (to a maximum of +3).
    \item If you spend 3 psi dice, the weapon overcomes a
      creature's resistance to non-magical weapons.
  \end{itemize}}

\DndPowerHeader%
  {Crisis of Breath}
  {2nd-level Voice}
  {1 action}
  {120 feet}
  {MB 4, PD \lvlthree}
  {Concentration, up to 1 minute}
Your Voice compels a humanoid creature you can see in range
to purge its entire store of air
in one explosive exhalation,
and thereby disrupt the target's autonomic breathing cycle.
The target's lungs do not automatically function
again while the powers's duration lasts.

If the target succeeds on a Wisdom save when
crisis of breath is manifested,
it is unaffected by this power.
If it fails its Wisdom save,
it can still continue to breathe by taking a standard action
in each round to gasp for breath.

An affected creature can attempt to take actions normally
(instead of consciously controlling its breathing),
but each round it does so,
beginning in the round when it failed its Wisdom save,
the target risks blacking out from lack of oxygen.
It must succeed on a Constitution save
at the end of any of its turns in which
it did not consciously take a breath.
The DC of this save increases by 1 in every consecutive
round after the first one that goes by without a breath;
the DC drops back to its original value
if the target spends an action to take a breath.

If a target fails its Constitution save,
it becomes unconscious.
At the end of its next turn,
it drops to 0 hit points
but must take death saving throws.
Curing powers or spells can revive a dying subject normally,
so long as this power's duration has expired; 
f the power is still in effect,
a revived creature is still subject to Constitution saves
in each round when it does not consciously breathe.
\augment{You can augment this power in one or more of the following ways.
\begin{itemize}
  \item If you spend 6 additional power points,
        this power can affect up to four creatures
        all within a 20-foot radius.
  \item If you spend 4 additional power points,
        this power can also affect creatures other than a humanoid,
        assuming they have a aerobic respiratory system to disrupt.
\end{itemize}}

\DndPowerHeader%
  {Cushion Fall}
  {1st-level Spacefolding}
  {1 reaction, which you take when you suffer any fall damage}
  {Self}
  {MB 1, PD \lvlone}
  {Instantaneous}
You recover instantly from a fall
and can absorb some damage from falling.
You land on your feet no matter how far you fall,
and you take damage as if the fall were
10 feet shorter than it actually is.
This power affects you and anything you carry or hold
(up to your maximum load).
\augment{For every additional psi die you spend, the power reduces
your damage as if the fall were an additional 10 feet damage.}

% PK3, P3
\DndPowerHeader%
  {Danger Sense}
  {3rd-level Prescience}
  {1 action}
  {Self}
  {MB 3, PD \lvlthree}
  {Concentration, up to 1 hour}
You can sense the presence of danger
before your senses would normally allow it.
Your intuitive sense alerts you to danger from traps,
giving you advantage on Dexterity saving throws to avoid traps
and a +4 bonus to your AC when subject to any attack roll
made by a trap.

\DndPowerHeader%
  {Deceleration}
  {1st-level Spacefolding}
  {1 action}
  {60 feet}
  {MB 1, PD \lvlone}
  {Concentration, up to 1 minute}
You warp space around a creature in range that you can see,
hindering the target's ability to move.
The target must succeed on a Wisdom saving throw,
otherwise their speed
(in every movement mode they possess) is halved.

A subsequent manifestation of deceleration on the target
does not further decrease its speed.
\augment{For every 2 additional psi dice you spend,
  this power can affect a target one size category larger.}

\DndPowerHeader%
  {Déjà Vu}
  {1st-level Voice}
  {1 action}
  {60 feet}
  {MB 1, PD \lvlone}
  {1 round}
The power of your Voice commands a target in range
that you can see to repeat their previous decisions.
The target must succeed on a Wisdom saving throw
or be forced to repeat the actions they took on their previous turn.
If the situation has changed in such a way that the target
can't take the same actions again
(if its foe is dead, or the target has run out of psi dice, and so on),
the target stands still and takes no actions for 1 round.
In any event, the target can still defend itself.
\augment{For every 2 additional psi dice you spend, this power's
save DC increases by 1.}

\DndPowerHeader%
  {Destiny Dissonance}
  {1st-level Prescience}
  {1 action}
  {Touch}
  {MB 1, PD \lvlone}
  {1 round per psionic threshold}
Your mere touch grants a creature
a vision of the limitless potential
of the not-yet-determined future,
possibilities undulating like a cloth blowing in the wind.
Unaccustomed to such a revelation,
the creature's mind is overwhelmed.
Make a melee psionics attack against a creature.
If the attack hits,
the creature takes 1d8 psychic damage
and becomes poisoned for a number of rounds
equal to your psionic threshold.
  \augment{For every two additional psi dice you spend,
    this power's damage increases by 1d8.}

\DndPowerHeader%
  {Detect Hostile Intent}
  {2nd-level Voice}
  {1 action}
  {30 feet}
  {MB 2, PD \lvltwo}
  {Concentration, up to 10 minutes}
While the duration of this power lasts,
you become aware of the presence of any creatures with hostile intent
within 30 feet of you,
and their direction from you
(but not their specific location).
The power detects active aggression,
as opposed to vigilance.
In addition, while this power is active you cannot be surprised
by creatures that are susceptible to mind-affecting powers.

While under the effect of this power,
you can make Wisdom (Insight) checks as a free action
against anyone you can see within 30 feet of you.

The power can penetrate barriers,
but 3 feet of stone,
3 inches of common metal,
1 inch of lead,
or 6 feet of wood or dirt blocks it.

\DndPowerHeader%
  {Detect Psionics}
  {1st-level Prescience}
  {1 action}
  {Self}
  {MB 1, PD \lvlone}
  {Concentration, up to 10 minutes}
For the power's duration,
you sense the presence of psionics within 30 feet of you.
If you sense psionics in this way,
you can use your action to see a faint aura around
any visible creature or object in the area that bears psionic effects,
and you learn the discipline of the effect, if any.

The power can penetrate most barriers, but is blocked by
1 foot of stone,
1 inch of common metal,
a thin sheet of lead,
or 3 feet of wood or dirt.

\DndPowerHeader%
  {Detect Prying Eyes}
  {4th-level Prescience}
  {1 action}
  {40 feet}
  {MB 3, PD \lvlfour}
  {One day}
You immediately become aware of any attempt to observe you
by means of a prescience power or divination spell.
The power's effect radiates in a 40-foot radius
from you and moves as you move.
You know the location of every psionic or magical sensor
within the power's area.

If the viewing attempt originates within the area,
you also know the viewer's location.
Otherwise, you and the remote viewer immediately make opposed checks.
You make an ability check using your psionics ability modifier,
and the remove viewer makes a check using their modifier
for manifesting powers or casting spells if a power or spell
respectively gave rise to the remote viewing.
If you at least match the remote viewer's result,
you get a visual image of the remote viewer
and a roughly accurate sense of the remote viewer's
direction and distance from you.

\DndPowerHeader%
  {Demoralise}
  {1st-level Voice}
  {1 action}
  {30 feet}
  {MB 1, PD \lvlone}
  {Concentration, up to 1 minute}
Choose 3 hostile creatures that you can see within range.
Each creature must make a Wisdom saving throw.
If they fail, they target takes a 1d4 penalty to attack rolls and
saving throws until the power ends.
\augment{For every two additional psi dice you spend,
        you may target one additional creature.}

\DndPowerHeader%
  {Detect Thoughts, Psionic}
  {2nd-level Voice}
  {1 action}
  {Self}
  {MB 2, PD \lvltwo}
  {Concentration, up to 1 minute}
As the \spell{detect thoughts} spell, except as noted here.

\DndPowerHeader%
  {Diamond Bullet}
  {1st-level Metacreativity}
  {1 action}
  {30 feet}
  {MB 1, PD \lvlone}
  {Instantaneous}
Upon manifesting this power,
you propel a diamond bullet at a target within range
that you can see and that you have line of effect to.
Make a ranged psionics attack against the target.
On a hit, the bullet deals 1d6 points of piercing damage.
\augment{For every 2 additional psi dice you spend,
  this power's damage increases by 2d6.}

  % N2, PK 2
\DndPowerHeader%
  {Dimension Swap}
  {2nd-level Spacefolding}
  {1 action}
  {60 feet}
  {MB 2, PD \lvltwo}
  {Instantaneous}
You instantly swap positions between your current position
and that of a designated ally in range that you can see.
Alternatively, you can swap the positions of
any two allies you can see that are both in range.
This power affects creatures of Large or smaller size.
You can bring along objects, but not other creatures.
\augment{For every 2 additional psi dice you spend, 
  this power can affect a target one size category larger.}

\DndPowerHeader%
  {Dispel Psionics}
  {3rd-level Psychokinesis}
  {1 action}
  {120 feet}
  {MB 3, PD \lvlthree}
  {Instantaneous}
Choose one creature, object, or psionic effect within range.
Any power of 3rd level or lower on the target ends.
For each power of 4th level or higher on the target,
make an ability check using your psionics ability.
The DC equals 10 + the power's level.
On a successful check, the power ends.
\augment{For every 2 additional psi dice you spend,
  you automatically end the effects of a power on the target
  that is one level higher.}

  % P1, PK1
\DndPowerHeader%
  {Dissipating Touch}
  {1st-level Spacefolding}
  {1 action}
  {Touch}
  {MB 1, PD \lvlone}
  {Instantaneous}
Your mere touch can disperse the surface material of a foe or object,
sending a tiny portion of it far away.
This effect is disruptive; thus,
your successful melee psionics attack deals 1d6 points of force damage.
\augment{For every additional psi dice you spend,
  this power's damage increases by 1d6.}

\DndPowerHeader%
  {Dissolving Touch}
  {2nd-level Prana Bindu}
  {1 action}
  {Touch}
  {MB 2, PD \lvltwo}
  {Instantaneous}
Your touch, claw, or bite is corrosive,
and sizzling moisture visibly oozes from your
natural weapon or hand.
You deal 4d6 acid damage to any creature or object
you hit with your next melee attack using the coated
natural weapon or unarmed attack.
You are immune to your own acid.
If you do not land a hit within a minute
of manifesting this power,
the effect wears off.
\augment{For every 2 additional psi dice you spend,
  this power's damage increases by 1d6.}

  \DndPowerHeader%
  {Distant Gaze}
  {2nd-level Prescience}
  {1 action}
  {See text}
  {MB 2, PD \lvltwo}
  {Concentration, up to 1 minute}
You can see and hear a distant location
almost as if you were there.
You don't need to see the location,
but the locale must be known;
that is, it must be
a place familiar to you or an obvious one
proximate to you,
such as behind a nearby door,
around a nearby corner,
or in a grove of local trees.
Once you have selected the locale,
the focus of your clairvoyant sense doesn't move,
but you can rotate it in all directions
to view the area as desired.
Unlike other scrying powers,
this power does not allow psionically or supernaturally
enhanced senses to work through it.

If the chosen locale is magically or psionically dark,
you see nothing.
If it is naturally pitch black,
you can see in a 10-foot radius around the center
of the power's effect or out to the extent
of your natural darkvision.
The power does not work across planes.

  % P1, PK1
\DndPowerHeader%
  {Distract}
  {1st-level Voice}
  {1 action}
  {30 feet}
  {MB 1, PD \lvlone}
  {Concentration, up to 1 minute}
Your Voice causes the mind of a creature you can see
in range to wander, distracting them.
If the target fails a Wisdom saving throw,
they have disadvantage on all Wisdom (Perception)
and Wisdom (Insight) checks until the power ends.

\DndPowerHeader%
  {Divert Teleport}
  {7th-level Spacefolding}
  {1 reaction, when you take whenever another creature teleports}
  {120 feet}
  {MB 6, PD \lvlseven}
  {Concentration, up to 10 minutes}
You can bend space and time to
divert the final destination of any teleportation attempt
made by any creature within 120 feet of you.
This power is manifested as a reaction to that attempt.

You can divert the destination of both incoming and outgoing
teleportations,
psionic and magical.
Whenever you use this power,
the teleporting creature makes a Wisdom saving throw.
If they succeed, the diversion is foiled.

For the purpose of this power,
`divert' means you choose the actual destination
of any teleportation attempt you can affect,
as if you yourself were teleporting to that location,
regardless of the teleportation range
of the effect you are diverting.
The destination you choose must be a location
with which you are very familiar by experience
or that you have studied carefully.
  
\DndPowerHeader%
  {Drain Power}
  {4th-level Voice}
  {1 action}
  {30 feet}
  {MB 4, PD \lvlfour}
  {Concentration, up to 1 minute}
Your Voice commands a target to relinquish their psionic power
reserves to you.
Choose a target that you can see within range.
The target must succeed on a Wisdom saving throw
or immediately lose 1d6 psi dice, one of which is added
to your pool of psi dice.
If the target has innate talents,
you may instead nominate to expend one use of
any of their innate talents with a psi dice cost of up to 6.

The target continues to lose 1d6 psi dice each round you
maintain concentration on the power and the subject remains
in range (or, alternatively, continues to expend uses of
innate talents that you choose).
In addition, you must continually use your action on your turn
to use the power.
If the subject is left with 0 psi dice
or has totally expended their uses of innate talents,
the power ends.

\DndPowerHeader%
  {Empath's Understanding}
  {1st-level Voice}
  {1 action}
  {30 feet}
  {MB 1, PD \lvlone}
  {Concentration, up to 1 minute}
You detect the surface emotions of a creature you can see in range.
You can sense basic needs, drives, and emotions.
Thirst, hunger, fear, fatigue, pain, rage, hatred, uncertainty,
curiosity, friendliness, and many other kinds of sensations and moods
can all be perceived.

You gain a +2 bonus to any social skill check involving the target
that you make in the round when you cease concentrating on this power.
\augment{
  You can augment this power in one or both of the following ways.
  \begin{itemize}
    \item For every additional psi die you spend,
          this power's range and the radius of its area increases by 5 feet.
    \item If you spend 2 additional psi dice,
          this power's maximum duration increases to 1 hour.
  \end{itemize}
}

\DndPowerHeader%
  {Empathic Feedback}
  {4th-level Voice}
  {1 action}
  {Self}
  {MB 3, PD \lvlfour}
  {Concentration, up to 10 minutes}
Your Voice berates your attacker,
causing it to empathically share your pain and suffering.
Each time a creature hits you,
it takes damage equal to the amount it dealt to you or 5 points,
whichever is less.
This damage is empathic in nature,
so powers and abilities the attacker may have such as damage reduction
and regeneration do not lessen or change this damage.
The damage from empathic feedback has no type,
so even if you took fire damage from a creature
that has immunity to fire,
empathic feedback will damage your attacker.
\augment{For every additional psi die you spend,
  this power's damage floor (base of 5) increases by 1.}

\DndPowerHeader%
  {Energy Bolt}
  {3rd-level Psychokinesis}
  {1 action}
  {Self (120-foot line)}
  {MB 3, PD \lvlthree}
  {Instantaneous}
Upon manifesting this power, you choose cold, lightning,
fire, or thunder.
You release a powerful stroke of energy of the chosen type
that emanates from your finger tips,
forming a line 120 feet long and 5 feet wide.
Each creature in the line must make a Dexterity saving throw.
A creature takes 6d6 points of damage of the chosen type
on a failed save,
or half as much on a successful save.

\subparagraph{Cold}
A bolt of this energy type deals +1 point of damage per die.
The saving throw to reduce damage from a cold bolt
is a Constitution save instead of a Dexterity save.

\subparagraph{Lightning}
Manifesting a bolt of this energy type provides
a +2 bonus to the save DC if the target is wearing metal armour.

\subparagraph{Fire}
A bolt of this energy type deals +1 point of damage per die.

\paragraph{Thunder}
A bolt of this energy type deals -1 point of damage per die
but provides a +1 bonus to the save DC.
\augment{For every additional psi die you spend,
  this power's damage increases by one die (d6).
  For each extra two dice of damage,
  this power's save DC increases by 1.}

\DndPowerHeader%
  {Energy Missile}
  {2nd-level Psychokinesis}
  {1 action}
  {120 feet}
  {MB 2, PD \lvltwo}
  {Instantaneous}
Upon manifesting this power,
you choose cold,
lightning,
fire,
or thunder.
You release a powerful missile of energy
of the chosen type at up to five creatures you can see in range
and have line of effect to.
Make a psionics attack against each target.
On a hit, the missile deals 3d6 points of damage of the chosen type.
You cannot hit the same target multiple times
with the same manifestation of this power.

\subparagraph{Cold}
A missile of this energy type deals +1 point of damage per die.
The saving throw to reduce damage from a cold missile
is a Constitution save instead of a Dexterity save.

\subparagraph{Lightning}
Manifesting a missile of this energy type provides
a +2 bonus to the attack roll against creatures wearing metal armour.

\subparagraph{Fire}
A missile of this energy type deals +1 point of damage per die.

\subparagraph{Thunder}
A missile of this energy type deals -1 point of damage per die
but reduces an opponent's AC by 1 for this attack only.
\augment{For every additional psi die you spend,
  this power's damage increases by 1d6.}

\DndPowerHeader%
  {Energy Push}
  {2nd-level Psychokinesis}
  {1 action}
  {60 feet}
  {MB 2, PD \lvltwo}
  {Instantaneous}
Upon manifesting this power, you choose cold, lightning, fire, or thunder.
You project a solid blast of energy of the chosen type at a target in range
that you can see and have line of effect to.
The target must make a Dexterity saving throw,
suffering from 4d6 points of damage of the chosen type if they fail.

In addition, if a subject of up to one size category larger than you
and fails a Strength saving throw,
the driving force of the energy blast pushes it back 5 feet
plus another 5 feet for every 5 points of damage it takes.

If a wall or other solid object prevents the subject from being pushed back,
the subject instead slams into the object and takes an extra
2d6 points of damage from the impact (no save).
\subparagraph{Cold}
  A blast of this energy type deals +1 point of damage per die.
  This does not apply to the slam damage.
  The saving throw to reduce damage from a cold push
  is a Constitution save instead of a Dexterity save.
\subparagraph{Lightning}
  Manifesting a blast of this energy type provides a +2 bonus
  to the save DC if the target is wearing metal armour.
\subparagraph{Fire}
  A blast of this energy type deals +1 point of damage per die
  (damage from impact remains at 2d6 points).
\subparagraph{Thunder}
  A blast of this energy type deals -1 point of damage per die
  (damage from impact remains at 2d6 points)
  but provides a +1 bonus to the save DC.
\augment{For every 2 additional psi dice you spend, this power's damage
  increases by one die (d6) and its save DC increases by 1.
  The damage increase applies to both the initial blast and any damage
  from impact with an object.}

\DndPowerHeader%
  {Energy Burst}
  {3rd-level Psychokinesis}
  {1 action}
  {Self (30-foot sphere)}
  {MB 3, PD \lvlthree}
  {Instantaneous}
Upon manifesting this power, you choose cold, lightning,
fire, or thunder.
You create an explosion of unstable energy of the chosen type
that radiates over a 30-foot sphere
centred on yourself.
Every creature or object within range that fails a Dexterity saving throw
takes 6d6 damage of the chosen type,
or half as much damage on a successful save.
The explosion creates almost no pressure.
Since this power extends outward from you,
you are not affected by the damage.

\subparagraph{Cold}
A burst of this energy type deals +1 point of damage per die.
The saving throw to reduce damage from a cold burst
is a Constitution save instead of a Dexterity save.

\subparagraph{Lightning}
Manifesting a burst of this energy type provides a +2 bonus
to the save DC to creatures wearing metal armour.

\subparagraph{Fire}
A burst of this energy type deals +1 point of damage per die.

\subparagraph{Thunder}
A burst of this energy type deals -1 point of damage per die
but provides a +1 bonus to the save DC.
\augment{For every additional psi die you spend,
  this power's damage increases by one die (d6).
  For each extra two dice of damage,
  this power's save DC increases by 1.}

\DndPowerHeader%
  {Energy Ray}
  {1st-level Psychokinesis}
  {1 action}
  {60 feet}
  {MB 1, PD \lvlone}
  {Instantaneous}
Upon manifesting this power,
you choose cold, lightning, fire, or thunder.
You create a ray of energy of the chosen type that
shoots forth from your fingertip and strikes a target within range
that you can see and have line of effect to,
dealing 2d6 points of damage if you succeed on a
ranged psionics attack with the ray.
The type of damage depends on
the type of energy of the ray you manifest. 
\subparagraph{Cold}
A ray of this energy deals +1 point of cold damage per die.
\subparagraph{Lightning}
Manifesting a ray of this energy type provides a
+3 bonus on your attack roll if the target is wearing metal armour.
\subparagraph{Fire}
A ray of this energy type deals +1 point of fire damage per die.
\subparagraph{Thunder}
A ray of this energy type deals -1 point of thunder damage per die
but reduces an opponent's AC by 1.
\augment{For every additional psi die you spend, this power's damage
  increases by 1d6.}

\DndPowerHeader%
  {Energy Transformation}
  {7th-level Prana Bindu}
  {1 action}
  {Self (30-foot line), see text}
  {MB 5, PD \lvlseven}
  {Concentration, up to 10 minutes}
Your bodily assimilates some of the effects of an
energy attack and converts it into a jet of plasma.
When you manifest this power,
the total amount of cold, fire,
lightning or thunder damage that you take
from separate sources is reduced by 10,
to a minimum of 0.
The amount of damage that has been reduced is `absorbed'.
You also have advantage on saving throws to
maintain concentration on powers if the
saving throw was prompted by damage of the
aforementioned types.
\emph{Note:} if a separate source of damage is
reduced to zero, you do not need to make a Constitution
saving throw to maintain concentration on any powers.

Whenever you absorb damage,
you store up the energy and can later discharge it
as a jet of plasma.
To discharge a ray requires a standard action.
The ray emerges from you as a 30-foot line with
a width of 5 feet.
Each creature caught in the path of the ray
takes damage equal to the amount of energy damage you have stored
on a failed save,
or half as much on a successful save.
However,
the total damage is capped at five times your psionic threshold level.
The ray you fire does fire or lightning damage
(you may decide on the type).

You can choose to fire any number of rays
during the power's duration.
As long as this power remains in effect,
you can continue to absorb energy damage and
fire additional rays using the stored damage.

\DndPowerHeader%
  {Energy Wave}
  {7th-level Psychokinesis}
  {1 action}
  {Self (60-foot cone)}
  {MB 6, PD \lvlseven}
  {Instantaneous}
Upon manifesting this power,
you choose cold, lightning, fire, or thunder.
You create a flood of energy of the chosen type
that deals 13d6 points of damage
to each creature and object in a 60-foot cone
centred on yourself
that fails a Dexterity saving throw,
or half as much on a successful save.
This power originates at your hand
and extends outward in a 60-foot cone.
\subparagraph{Cold}
A wave of this energy type deals +1 point
of damage per die.
The saving throw to reduce damage from a cold wave
is a Constitution save instead of a Dexterity save.  
\subparagraph{Lightning}
Manifesting a wave of this energy type
provides a +2 bonus to the save DC if the target is wearing
metal armour.
\subparagraph{Fire}
A wave of this energy type deals +1 point of damage per die.
\subparagraph{Thunder}
A wave of this energy type deals -1 point of damage per die
but grants a +1 bonus to the save DC.
\augment{For every additional psi dice you spend,
  this power's damage increases by 1d6.
  For each extra two dice of damage,
  this power's save DC increases by 1.}

\DndPowerHeader%
  {Envenom}
  {1st-level Prana Bindu}
  {1 bonus action}
  {Touch}
  {MB 1, PD \lvlone}
  {Concentration, up to 1 minute}
Your saliva briefly turns into a caustic bile
which does no damage to you,
but can damage your enemies.
You may use your bonus action to coat one weapon
you are holding with the bile.
The next time you hit with that weapon,
the bile deals 2d4 acid damage.

After the hit, roll 1d10.
On a roll of 10,
the bile totally corrodes the weapon you coated,
and it is destroyed
(unless it is a magic weapon, resistant to acid damage
or otherwise impervious to physical damage).

\DndPowerHeader%
  {Flame Choke}
  {1st-level Prana Bindu}
  {1 action}
  {Self}
  {MB 1, PD 1}
  {Instantaneous}
Your vice-like grip burns those that oppose you.
As part of the power's manifesting time,
take the grapple action against a creature within melee range.
If the grapple is successful,
the target immediately takes 2d6 fire damage.
On every turn you start while the target is still grappled,
the target takes an additional 2d6 fire damage.
Whenever the grapple ends,
the target must succeed on a Strength saving throw
or fall prone.

\DndPowerHeader%
  {Float}
  {1st-level Spacefolding}
  {1 action}
  {Self}
  {MB 1, PD \lvlone}
  {Concentration, up to 1 minute}
You mentally support yourself in water or similar liquid.
You gain a swim speed equal to your base movement speed for
the power's duration,
or use the power to boost your swim speed by 10 feet
if you already have one.

\DndPowerHeader%
  {Fist of Fury}
  {3rd-level Prana Bindu}
  {1 action}
  {Self}
  {MB 3, PD \lvlthree}
  {Concentration, up to 10 minutes}
Your fists become fuelled with terrifying power.
While the power lasts,
any unarmed attack you make with your fist
scores a critical hit on a roll of
18, 19 or 20.

\DndPowerHeader%
  {Force Screen}
  {1st-level Psychokinesis}
  {1 action}
  {Self}
  {MB 2, PD \lvlone}
  {Concentration, up to 1 minute}
You create an invisible mobile disk of force that hovers in front of you.
The force screen provides a +2 bonus to your AC until the power ends.
\augment{For every 4 additional psi dice you spend,
  the bonus to your AC from the force screen improves by 1.}

\DndPowerHeader%
  {Fusion}
  {8th-level Prana Bindu}
  {1 action}
  {Touch}
  {MB 8, PD \lvleight}
  {Concentration, up to 1 minute}
You and another willing, corporeal, living creature
of the same or smaller size fuse into one being.
As the manifester,
you control the actions of the fused being.
However, you can give up this control to the other creature.
Once you give up control,
you cannot regain it unless the other creature relinquishes it.

The fused being has your current hit points
plus the other creature's current hit points.
The fused being knows all the powers you
and the other creature know,
has the sum of your and the other creature's psi dice,
has the sum of your and the other creature's spell slots,
and knows or has prepared any spells you or the other creature possesses
(if any).
Likewise, all feats, racial abilities, and class features
are pooled
(if both creatures have the same ability,
the fused being gains it only once).
For each of the six ability scores,
the fused being's score is the higher of yours
and the other creature's,
and the fused being also has the higher hit dice
and psionic threshold;
this effectively means the fused being uses
the better saving throws,
attack bonus,
and skill modifiers of either member,
and it manifests powers up to the higher of the psionic thresholds
that you or the other creature possessed before becoming fused.

You decide what equipment is absorbed into the fused being
and what equipment remains available for use.
These fused items are restored once the power ends.

When the power ends,
the fused being separates.
The other creature appears in an area adjacent to you that you determine.
If separation occurs in a cramped space,
the other creature is expelled through the Astral Plane,
finally coming to rest materially in the nearest empty space
and taking 1d6 force damage
for each 10 feet of solid material passed through.

Damage taken by the fused being is split evenly
between you and the other creature when the power ends.
You do not leave the fusion with more hit points
than you entered it with,
unless you were damaged prior to the fusion
and the fused being was subsequently healed.
In a like manner,
the fused being's remaining power points are split
between you and the other creature
(you can leave with more points than you entered with,
as long as you don't exceed the maximum power points
for your level and ability score).

If a fused being is killed,
it separates into its constituent creatures,
both of which are also dead.

\DndPowerHeader%
  {Grip of Iron}
  {1st-level Prana Bindu}
  {1 bonus action or reaction, see text}
  {Self}
  {MB 1, PD \lvlone}
  {Until the end of your current turn}
You can improve your chances in a grapple as a bonus action,
gaining a +3 bonus on a grapple checks until
the end of your current turn.

In addition, you can also manifest this power
as a reaction to being grappled; in this case, you shrug
off your opponent's grip with great strength.
You gain a +3 bonus to the check to avoid being grappled.
If you do so,
your check must use Strength (Athletics) and not
Dexterity (Acrobatics).
\augment{For every 3 additional psi dice you spend,
  the bonus increases by 1.}

\DndPowerHeader%
  {Hammer}
  {1st-level Prana Bindu}
  {1 bonus action}
  {Self}
  {MB 1, PD \lvlone}
  {1 round}
This power charges your touch with the force of a sledgehammer.
Until the end of your next turn,
if you make a successful unarmed attack against any creature or object,
you use 1d12 for your damage dice.
\augment{For every additional psi dice you spend, the power's duration
increases by 1 round.}

\DndPowerHeader%
  {Identify Psionics}
  {1nd-level Prescience}
  {1 minute}
  {Touch}
  {MB 1, PD \lvlone}
  {Instantaneous}
As the \spell{identify} spell,
except for items imbued with psionics
and creatures manifesting powers.

\DndPowerHeader%
  {Immovability}
  {4th-level Prana Bindu}
  {1 action}
  {Self}
  {MB 4, PD \lvlfour}
  {Concentration, up to 1 hour}
  You are almost impossible to move.
Your weight does not vary;
instead, your body adjusts itself to withstand
immense amounts of force.
You could conceivably do this in midair.
You have a +10 bonus on any check or saving throw
against being moved.
You can't voluntarily move to a new location
unless you stop concentrating, which ends the power.

While under the effects of this power,
you cannot add your Dexterity bonus to your AC;
however, your anchored body gains damage threshold (10).

You cannot make physical attacks
or perform any other large-scale movements
(you can make small-scale movements,
such as breathing,
turning your head,
moving your eyes,
talking, and so on).
\augment{If you spend 6 psi dice you spend,
  you can manifest this power as a reaction to being
  involuntarily moved.}

\DndPowerHeader%
  {Inertial Armour}
  {1st-level Psychokinesis}
  {1 action}
  {Touch}
  {MB 1, PD \lvlone}
  {8 hours}
As for the \spell{mage armour} spell, except as noted.

\DndPowerHeader%
  {Inertial Dampener}
  {4th-level Psychokinesis}
  {1 action}
  {Self}
  {MB 4, PD \lvlfour}
  {Concentration, up to 10 minutes}
You create a skin-tight psychokinetic barrier around yourself
that resists blunt-force blows and explosions.
You become immune to bludgeoning damage for the power's duration.

\DndPowerHeader%
  {Inflict Pain}
  {2nd-level Voice}
  {1 action}
  {30 feet}
  {MB 2, PD \lvltwo}
  {Concentration, up to 1 minute}
You Voice stabs at the mind of your foe,
unleashing torrents of horrible agony.
Choose a creature in range that you can see.
They must succeed on a Wisdom saving throw or
suffer wracking pain that imposes disadvantage on
their attack rolls and skill checks.
\augment{For every 2 additional psi dice you spend,
  this power's save DC increases by 1,
  and the power can affect an additional target.
  Any additional target cannot be more than 15 feet
  from another target of the power.}

\DndPowerHeader%
  {Leap}
  {2nd-level Prana Bindu}
  {1 bonus action}
  {Self}
  {MB , PD \lvltwo}
  {Until the end of your current turn}
Your body prepares itself to undertake an incredible feat
of agility and strength.
As a bonus action,
you can propel yourself across vast distances.
You gain a flying speed equal to your base movement speed
until the end your current turn.
For example,
if you have taken the dash action on your turn to double
your base movement speed,
the fly speed you gain would be equal to that doubled speed.
\augment{If you spend 4 additional psi dice,
  any movement with this fly speed does not provoke
  opportunity attacks.}
 
\DndPowerHeader%
  {Levitate}
  {2nd-level Spacefolding}
  {1 action}
  {Self}
  {MB 2, PD \lvltwo}
  {Concentration, up to 10 minutes}
As the \spell{levitate} spell,
except when manifested the target can only be the manifester themselves,
not any other creature or a loose object.

\DndPowerHeader%
  {Matter Agitation}
  {1st-level Psychokinesis}
  {1 action}
  {30 feet}
  {MB 1, PD \lvlone}
  {Concentration, up to 1 minute}
You can excite the structure of a non-psionic,
non-magical object,
heating it to the point of combustion over time.
The agitation grows more intense in the second and third rounds
after you manifest the power, as described below.

\subparagraph{1st Round}
  Readily flammable material
  (paper, dry grass, tinder, torches) ignites.
\subparagraph{2nd Round}
  Wood smolders and smokes,
  metal becomes hot to the touch,
  paint shrivels, water boils.
\subparagraph{3rd and Subsequent Rounds}
  Wood ignites, metal scorches
  (1d4 points of damage for those holding metallic objects),
  lead melts.

\DndPowerHeader%
  {Manipulate Creature With Mind}
  {4th-level Psychokinesis}
  {1 action}
  {60 feet}
  {MB 3, PD \lvlfour}
  {Concentration, up to 1 minute}
You deal psychokinetic blows at a distance
against creatures you can see.
For the power's duration,
you can perform the shove, disarm or grapple action
without needing to be in melee range
(however, the target needs to be in range).
Resolve these attempts as normal.
Wherever you would make a Strength (Athletics) check or attack roll,
use your psionics attack modifier for the roll.
\augment{For every additional psi die you spend,
  this power grants a +1 bonus on all rolls you make with his power to
  shove, grapple and disarm.}
  
  \DndPowerHeader%
    {Manipulate Object With Mind}
    {3rd-level Psychokinesis}
    {1 action}
    {120 feet}
    {MB 2, PD \lvlthree}
    {Concentration, up to 1 minute}
By focusing your mind, you are able to manipulate objects as you desire.
Choose an object you can see within range
weighing no more than 250 pounds.
Each turn you spend concentrating on this power,
including the turn in which you manifest this power,
you may move the object up to 20 feet.
This power ends if the object is moved out of range
or you can no longer see the object,
and if you cease concentration,
the object falls or stops.

You may drop the object and pick up another during
the same turn as long as you don't stop concentrating
on the power.
You may also exert fine control on the objects you move.
The object does not move fast enough to cause any damage
when being rammed into other creatures or objects,
however it can cause damage to creatures and objects
if it is dropped onto them.
In this case, the object or creature deals as much
bludgeoning damage as it would take from the fall itself.
If the object falls onto a creature,
they must make a Dexterity saving throw, taking the full
damage on a failed save and no damage on a successful save.

If the object you are attempting to move is being being carried,
worn or is equipped to a creature in some way,
the creature can prevent you from moving the object.
If they do so, make an ability check with your
psionics ability modifier (not psionics attack modifier)
contested by their Strength check.
If they win the contest, you cannot move that and
any other object until your turn ends.
If the contest ends in a draw, you cannot move
that object, but you may attempt to move others.
  \augment{For every additional psi die you spend,
    the weight limit of the target increases by 25 pounds.}
  
\DndPowerHeader%
  {Mind Blank, Psionic}
  {6th-level Voice}
  {1 action}
  {Self}
  {MB 6, PD \lvlsix}
  {24 hours}
As the \spell{mind blank} spell, except as noted here.
You conceals your mind---or
the mind of one creature you touch---from
outside influence,
rendering it effectively invisible. 

\DndPowerHeader%
  {Mind Palace, Power$^*$}
  {4th-level Psychokinesis}
  {1 reaction, see text}
  {Self}
  {MB 4, PD \lvlfour}
  {1 round, see text}
As a reaction to when you or an ally within 20 feet of you
is targeted by an attack or subject to a harmful psionic effect,
you encase yourself and your allies in a
shimmering palace of psychokinetic force.
The palace is a sphere of radius 20 feet centred on you. 
All damage from powers and innate talents
taken by subjects inside the area of the mind palace
is halved.
This lowering takes place prior to the effects
of other powers or abilities that lessen damage.
The palace lasts until the start of your next turn.
$^*$Not to be confused with the \emph{mind palace} ability.
\augment{For every additional psi dice you spend,
  this power's duration increases by 1 round.}

\DndPowerHeader%
  {Mind Thrust}
  {1st-level Voice}
  {1 action}
  {30 feet}
  {MB 1, PD \lvlone}
  {Instantaneous}
You instantly deliver a massive assault on the thought pathways
of any one creature in range.
The target must succeed on an Intelligence saving throw
or take 1d12 points of psychic damage.
\augment{For every additional psi dice you spend, the power's damage
increases by 1d12.}

\DndPowerHeader%
  {Missive}
  {1st-level Voice}
  {1 action}
  {60 feet}
  {MB 1, PD \lvlone}
  {Instantaneous}
You send a telepathic message of up to ten words to
any living creature within range.
Missive is strictly a one-way exchange from you to the target.
If you do not share a common language,
the target `hears' meaningless mental syllables.
\augment{For every 2 additional psi dice you spend,
  the power's range increases by 5 feet.}

\DndPowerHeader%
  {Object Reading}
  {2nd-level Prescience}
  {1 minute}
  {Touch}
  {MB 2, PD \lvltwo}
  {Concentration, up to 10 minutes}
You can learn details of an inanimate object's previous owner.
Objects accumulate psychic impressions
left by their previous owners,
which can be read by use of this power.
The amount of information revealed depends
on how long you study a particular object.

\subparagraph{1st Minute}
  Last owner's race.
\subparagraph{2nd Minute}
  Last owner's gender.
\subparagraph{3rd Minute}
  Last owner's age.
\subparagraph{4th Minute}
  Last owner's alignment.
\subparagraph{5th Minute}
  How last owner gained and lost the object.
\subparagraph{6th+ Minute}
  Next-to-last owner's race, and so on.

The power always correctly identifies the last owner of the item,
and the original owner
(if you keep the power active long enough).

There is a 90\% chance
that this power will successfully identify all other former owners
in sequence,
but there is a 10\% chance that one former owner
will be skipped and thus not identified.

This power will not identify casual users as owners
(anyone who uses an object to attack someone
or something is not thereafter considered a casual user).

An object without any previous owners reveals no information.
You can continue to run through a list of previous owners
and learn details about them as long as the power's duration lasts.
If you use this power additional times on the same object,
the information yielded is the same
as if you were using the power on the object for the first time.
\augment{For every additional psi die you spend,
  this power's maximmum duration increases by 10 minutes.}

\DndPowerHeader%
  {Orient Self}
  {1st-level Prescience}
  {1 action}
  {Self}
  {MB 1, PD \lvlone}
  {Instantaneous}
You generally know where you are.
This power is useful to characters who end up at unfamiliar destinations
after teleporting,
using a gate,
or travelling to or from other planes of existence.
The power reveals general information about your location
as a feeling or presentiment.
The information is usually no more detailed than a summary
that locates you according to a prominent local or regional site.
Using this power also tells you what direction you are facing.

\DndPowerHeader%
  {Plane Shift, Psionic}
  {5th-level Spacefolding}
  {1 action}
  {Touch}
  {MB 4, PD \lvlfive}
  {Instantaneous}
Your mind warps space and time,
allowing you to move yourself
or some other creature to another plane of existence
or alternate dimension.
If several willing persons link hands in a circle,
as many as eight can be affected by the plane shift at the same time.
Unlike the \spell{plane shift} spell,
precise accuracy as to a particular arrival location
on the intended plane is nigh impossible,
unless you are attuned to a planar beacon
or have some other means of anchoring yourself in one plane.
Without these items,
when you travel to another plane of existence,
you appear 5 to 500 miles (5d100) from your intended destination.
\augment{For every 3 additional psi dice you spend,
  roll one less d100 when determining how close you arrive
  to your intended destination.}

\DndPowerHeader%
  {Possible Futures}
  {5th-level Prescience}
  {1 action}
  {Self}
  {MB 4, PD \lvlfive}
  {Concentration, up to 1 minute}
You take a hand in influencing the probable outcomes
of your immediate environment.
You see the many alternative branches that reality could take
in the next few seconds,
and with this foreknowledge you gain the ability to,
while this power lasts,
reroll one attack roll,
one saving throw,
one ability check,
or one skill check each round.
You must take the result of the reroll,
even if it's worse than the original roll.
You do not have to make another roll
if satisfied with your original roll.
The roll must be made by yourself.

\DndPowerHeader%
  {Power Pirate}
  {6th-level Voice}
  {1 action}
  {60 feet}
  {MB 5, PD \lvlsix}
  {Concentration, up to 1 round per psionic threshold}
You use your Voice to take over control of a power
requiring concentration that was manifested
by a creature you can see in range.
The target of this power is the manifesting creature.

Once you wrest control of the power from the target,
you have several options.
\begin{itemize}
  \item Allow the power to function as normal.
  \item Keep the power targeted on the manifesting creature
        (if a power with a target of self) but decide how the power
        fulfils its function each round.
  \item Retarget the power on yourself (if a power with a target of self).
  \item Choose not to concentrate on the usurped power
          in the next round, ending the power at that point.
\end{itemize}
When the duration of \power{power pirate} expires,
the power you took control of ends
(even if this would mean that the power ends earlier than normal).

\DndPowerHeader%
  {Precognition}
  {1st-level Prescience}
  {1 action}
  {Self}
  {MB 2, PD \lvlone}
  {Concentration, up to 1 minute}
Your awareness of your immediate circumstances extends
a fraction of a second into the future,
granting you an edge over your opponents and environment.
Choose a variation from the list below.
The nature of the benefit conferred to you
depends on the chosen variant.
\subparagraph{Defensive}
  You can a +2 bonus to your AC and on all saving throws.
\subparagraph{Anticipatory}
  You gain a +2 bonus to attack rolls.
\subparagraph{Offensive}
  You gain a +3 to bonus to your damage rolls.
\augment{For every 3 additional psi dice you spend, the bonus
  you have gained from a variant increaes by 1 to a maximum of 4.}

% P4, PK4
\DndPowerHeader%
  {Psi Door}
  {4th-level Spacefolding}
  {1 action}
  {500 feet}
  {MB 3, PD \lvlfour}
  {Instantaneous}
As the \spell{dimension door}, except as noted here.
\augment{If you spend 6 additional psi dice,
  you can manifest this power as a bonus action.}

\DndPowerHeader%
  {Psi Hand}
  {1st-level Psychokinesis}
  {1 action}
  {30 feet}
  {MB 1, PD \lvlone}
  {Concentration, up to 1 minute}
As the \spell{mage hand} spell.

\DndPowerHeader%
  {Psi Vision}
  {2rd-level Prescience}
  {1 action}
  {Self}
  {MB 2, PD \lvltwo}
  {8 hours}
As the \spell{darkvision} spell.

\DndPowerHeader%
  {Psi Wall}
  {4th-level Metacreativity}
  {1 action}
  {30 feet}
  {MB 4, PD \lvlfour}
  {Concentration, up to 1 minute}
You fashion a great wall of raw material which erupts from the earth
at a point you can see within range.
The wall cannot move once it is formed.
The origin of the wall must be within range,
but it can extend past this power's range.
It is 1 inch thick per psionic threshold
and has a side length of 10 feet per psionic threshold.
You can also choose to form the wall into a sphere or hemisphere
with a radius of up to 1 foot per psionic threshold.
The wall must be continuous and unbroken when manifested.
If its surface is interrupted by any object or creature,
the power fails.

Each 10-foot panel of the wall
(or section of surface area 100 ft.$^2$)
has 10 hit points per inch of thickness
and damage immunity (5).
A section of the wall whose hit points drop to 0 is breached
and can be passed through.
Otherwise, if a creature tries to break through the wall,
they can do so with a successful Strength check with
DC equal to 15 + the number of inches it is thick.

The wall is susceptible to the \power{dispel psionics} power,
but it gains a +4 bonus on any check
to determine whether the wall is negated in such a way.
Spells and breath weapons cannot pass through the wall
in either direction,
since the wall provides total cover
(though they could damage it).
Powers which ordinarily do not require line of effect
cannot target creatures or objects
on the opposite side of the wall.
However, the wall does not block travel by Spacefolding.
The wall is opaque, and thus a creature on one side does not have
sight of creatures or objects on the other side.

\DndPowerHeader%
  {Psionic Golem}
  {1st-level Metacreativity}
  {1 action}
  {30 feet}
  {MB 1, PD \lvlone}
  {Up to 1 minute}
\textit{Note: you cannot manifest this power unless
you have learnt the create golem Metacreativity tenet.}

This power creates one 1st-level psionic golem
(see \secref{sec:psionic_golem})
that attacks your enemies.
It appears where you designate in range and acts
whenever you end your turn
(it still technically acts on your turn, however).
It attacks your opponents to the best of its ability.
As a free action,
you can mentally direct it not to attack,
to attack particular enemies,
or to perform other actions.
The psionic golem acts normally on the last round
of the power's duration
and dissipates after you finish your turn.
\augment{For every 3 additional psi dice you spend,
  the level of the psionic golem increases by 1.}

\DndPowerHeader%
  {Psionic Golem, Repair}
  {2nd-level Metacreativity}
  {1 action}
  {Touch}
  {MB 2, PD \lvltwo}
  {Instantaneous}
When laying your hands upon a psionic golem
that has at least 1 hit point remaining,
you reknit its structure to repair damage it has taken.
The power repairs 3d8 points of damage
+1 point per psionic threshold.
Constructs that are immune to psionics or magic
cannot be repaired in this fashion.
\augment{For every 2 additional psi dice you spend,
  this power repairs an additional 1d8 hit points.}

\DndPowerHeader%
  {Psionic Step}
  {2nd-level Spacefolding}
  {1 bonus action}
  {Self}
  {MB 2, PD \lvltwo}
  {Instantaneous}
You create a brief, unstable link between two nearby locations,
allowing you to jump from one spot to another.
You teleport to an unoccupied space you can see
up to 30 feet from you.
\augment{For every 2 additional psi dice you spend,
  you can teleport an extra 10 feet.}

\DndPowerHeader%
  {Psychic Archaeology}
  {2nd-level Prescience}
  {1 hour}
  {60 feet}
  {MB 2, PD \lvltwo}
  {Concentration, up to 1 hour}
You gain historical vision in a given location.
Rooms, streets, tunnels, and other discrete locations
accumulate psychic impressions left by powerful emotions
experienced in a given area.
These impressions offer you a picture of the location's past.

The types of events most likely to leave
psychic impressions are those that elicited strong emotions:
battles and betrayals,
marriages and murders,
births and great pain,
or any other event where one emotion dominates.
Everyday occurrences leave no residue for a manifester to detect.

The vision of the event is dreamlike and shadowy.
You do not gain special knowledge of those involved in the vision,
though you might be able to read large banners
or other writing if they are in your language.

Beginning with the most recent significant event
at a location and working backward in time,
you can sense one distinct event for every 10 minutes
you maintain concentration,
if any such events exist to be sensed.
Your sensitivity extends into the past
a maximum number of years equal to
200 times your psionic threshold.

\DndPowerHeader%
  {Psychic Crush}
  {5th-level Voice}
  {1 action}
  {30 feet}
  {MB 4, PD \lvlfive}
  {Instantaneous}
Your Voice abruptly and brutally crushes the mental essence
of any one creature you can see, debilitating its acumen.
The target must succeed on a Wisdom saving throw
or take 7d12 psychic damage.
If they succeed, they instead take 4d6 psychic damage.
\augment{For every 2 additional psi dice you spend,
  the damage on a successful save increases by 1d6.}

\DndPowerHeader%
  {Psychofeedback}
  {5th-level Prana Bindu}
  {1 action}
  {Self}
  {MB 5, PD \lvlfive}
  {1 minute}
You can readjust your body to boost one
physical ability score at the expense of one
or more of the other scores.
Select one ability score you would like to boost,
and increase it by the same amount that you decrease
one or more other scores.

You can boost your
Strength, Dexterity or Constitution score by an
amount equal to your psionic threshold
(or any lesser amount),
assuming you can afford to burn your other ability
scores to such an extent.
You may boost your ability scores past 20 in this fashion.

When the duration of this power expires,
your ability boost ends,
but your the ability score decrease
remains until you finish a long rest.

  % PK 2, P3
\DndPowerHeader%
  {Rapid Healing}
  {3rd-level Prana Bindu}
  {1 bonus action}
  {30 feet}
  {MB 2, PD \lvlthree}
  {Instantaneous}
You mind takes control of your body's healing process,
rapidly speeding it up.
You gain hit points equal to 1d12 + your psionics ability modifier.
\augment{For every 2 additional psi dice you spend, this power
  heals an additional 1d8 points of damage.}

\DndPowerHeader%
  {Recall}
  {1st-level Prescience}
  {1 minute}
  {Self}
  {MB 1, PD \lvlone}
  {Instantaneous}
By meditating on a subject, you can recall
natural memories and knowledge otherwise inaccessible to you.

When you fail a History, Religion, Arcana or Nature check,
you can manifest this power to, at the end of the
power's manifesting time, repeat the check.
If successful, you instantly recall
what was previously buried in your subconscious.

\DndPowerHeader%
  {Remote Viewing Trap}
  {6th-level Prescience}
  {1 action}
  {Self}
  {MB 5, PD \lvlsix}
  {24 hours}
When others use any means of scrying you from afar,
your prepared trap gives them a nasty surprise.
If the scryer fails a Wisdom saving throw
when it attempts to scry you,
you are undetected.
Moreover, the would-be observer takes 8d6 lightning damage.
If the scryer succeeds its saving throw,
it takes only half as much damage
and is able to observe you normally.

Either way, you are aware of the attempt to view you,
but not of the viewer or the viewer's location.

\DndPowerHeader%
  {Restore Extremity}
  {5th-level Prana Bindu}
  {1 action}
  {Touch}
  {MB 5, PD \lvlfive}
  {Instantaneous}
You restore a severed extremity to a creature that has
lost a digit, hand, arm, leg,
or even its head.
This power does not restore life,
but it returns a lost extremity to a living or dead creature
if the creature is otherwise mostly intact.
The original extremity need not be present
when this power is manifested;
a new extremity is created by the power.
If a head is restored to a body,
the original head (if not already destroyed)
loses all spark of identity,
and can be considered as dead tissue.

\DndPowerHeader%
  {Resuscitation}
  {3rd-level Voice}
  {1 action}
  {30 feet}
  {MB 3, PD \lvlthree}
  {Instantaneous}
Your Voice commands a creature back to life.
If a creature you can see within range
has died in between your previous turn
and your current turn,
you may as an action return them to life
with 1 hit point.
This power can't return to life a creature
that has died of old age,
nor can it restore any missing body parts.

\DndPowerHeader%
  {Retrieve}
  {6th-level Spacefolding}
  {1 action}
  {60 feet}
  {MB 5, PD \lvlsix}
  {Instantaneous}
You automatically teleport an item you can see within range
directly to your hand.
The item can weigh up to
10 times your psionic threshold in pounds.
If the object is in the possession of an opponent,
it comes to your hand if your opponent
fails a Wisdom saving throw.
\augment{For every additional two psi dice you spend,
  the weight limit of the target increases by 10 pounds.}

\DndPowerHeader%
  {Reveal Agony}
  {2nd-level Prescience}
  {1 action}
  {60 feet}
  {MB 2, PD \lvltwo}
  {Instantaneous}
The fabric of time parts to your will,
revealing wounds a creature in range is yet to receive.
If the target fails a Wisdom saving throw,
the target takes 4d6 psychic damage as the future
impinges briefly on the present.
\augment{For every additional psi dice you spend, 
  this power's damage increases by 1d6 points.
  For each extra 2d6 points of damage,
  this power's save DC increases by 1.}

  % P7, PK3
\DndPowerHeader%
  {Sacrificial Shell}
  {3rd-level Metacreativity}
  {1 reaction, which you take when making a saving throw against certain damage types, see text}
  {Self}
  {MB 3, PD \lvlthree}
  {Instantaneous}
You instantly weave a protective shell around you
which absorbs damage and then falls of.
When you make a saving throw against an effect
which would deal fire, cold, lightning or thunder damage,
you may manifest this power as a reaction.
For this saving throw only,
if you fail, you take only half damage from the effect.
If you succeed, you take no damage.

\DndPowerHeader%
  {Scrying, Psionic}
  {5th-level Prescience}
  {10 minutes}
  {Self}
  {MB 5, PD \lvlfive}
  {Concentration, up to 10 minutes}
As the \spell{scrying} spell, except as noted here.
The power works across the planes of existence.
In addition,
instead of creating an invisible sensor which appears
as a luminous orb,
an ethereal form of yourself appears at the target's location,
scarcely detectable to the naked eye.
You,
however,
must succeed immediately on a Intelligence (Stealth) check
to avoid detection contested by the target's passive
Wisdom (Perception).
You have a +10 bonus to this check.

If you wish,
you may speak through your ethereal form,
however your voice is whispery and faint.

\DndPowerHeader%
  {Sense Link}
  {1st-level Voice}
  {1 action}
  {120 feet}
  {MB 1, PD \lvlone}
  {Concentration, to 1 minute}
You target one willing creature within sight.
You perceive what the target creature perceives using its sight,
hearing, taste, or smell.
Only one sense is linked,
and you cannot switch between senses with the same manifestation.

You make any skill checks involving senses,
such as Wisdom (Perception), as the target,
and only within the target's field of view.
You lose your Dexterity bonus to AC while
directly sensing what the target senses
if you are perceiving using the target's sight or hearing.

Once sense link is manifested, the link persists
even if the target moves out of the range
of the original manifestation as is normally the case
(but the link does not work across planes).
You do not control the target,
nor can you communicate with it by means of this power.

The strength of the target's linked sense could be enhanced
by other powers or items the target possesses,
allowing you the same enhanced sense.
If you are blinded or deafened, or suffer some other sensory deprivation,
the linked creature functions as an independent sensory organ,
and provides you the benefit of the linked sense from its perspective
while this power's duration lasts.
\augment{You can augment this power in one or
both of the following ways:
\begin{itemize}
  \item If you spend 2 additional psi dice,
        you can have the target perceive one of your senses
        instead of the other way around.
  \item If you spend 4 additional psi dice,
        you can link to a second sense of the same target.
\end{itemize}}

\DndPowerHeader%
  {Sense Link, Forced}
  {2nd-level Voice}
  {1 action}
  {60 feet}
  {MB 2, PD \lvltwo}
  {Concentration, up to 1 minute}
As the \power{sense link} power,
except you can use this power on any creature---willing or
unwilling---and this power can't be augmented.

\DndPowerHeader%
  {Share Pain}
  {3rd-level Voice}
  {1 action}
  {Touch}
  {MB 3, PD \lvlthree}
  {Concentration, up to 1 minute}
You use your Voice to convince a creature that your wounds are
in fact its own.
A creature you touch must make a Wisdom saving throw.
If they fail, a link is created between you and that creature.
You take half damage from all attacks that deal damage to you,
and the target takes the remainder as psychic damage.
If your hit point maximum is reduced by any means,
that reduction is not shared with the target.
If you are immune, resistant or have some reduction
to the type of damage dealt to you,
this also applies to the linked creature. 

When this power ends,
subsequent damage is no longer divided between the target and you,
but damage already shared is not reassigned.
\augment{For every 2 additional psi dice you spend,
  this power's save DC increases by 1.}

\DndPowerHeader%
  {Shatter Mind Blank}
  {5th-level Voice}
  {1 action}
  {30 feet}
  {MB 5, PD \lvlfive}
  {Instantaneous}
Your Voice is an illuminating light,
revealing minds which have been hidden from you.
This power can negate the \power{psionic mind blank} power
or the \spell{mind blank} spell,
affecting all creatures within 30 feet of you.
You can shatter each creature's mind blank by succeeding on a
contest between your roll of
1d20 + your psionic attack modifier
and each targets' roll of 
1d20 + their psionics attack modifier or
their spellcasting attack modifier,
depending on whether a power or spell respectively
gave rise to the mind blank effect.
If you succeed,
the psionic mind blank ends,
allowing you to affect the target thereafter
with Voice or any other powers and abilities.

\DndPowerHeader%
  {Shockwave}
  {2nd-level Psychokinesis}
  {1 action}
  {60 feet}
  {MB 2, PD \lvltwo}
  {Instantaneous}
A target in range you select is pummelled with telekinetic force
for 2d6 + 2 force damage as a shockwave radiates out from your palm.
The shockwave always damages a target within range that you can see.

Nonmagical, unattended objects
(including doors, walls, locks, and so on)
may also be damaged by this power.
\augment{You can augment this power in one or both of the following ways.
\begin{itemize}
  \item For every 2 additional psi dice you spend,
        this power's damage increases by 1d6 points.
  \item For every 2 additional psi dice you spend,
        this power can affect an additional target.
        Any additional target cannot be more than 15 feet
        from another target of the power.
\end{itemize}}

\DndPowerHeader%
  {Skate}
  {1st-level Spacefolding}
  {1 action}
  {Self or touch}
  {MB 1, PD \lvlone}
  {Concentration, up to 1 minute}
You, another willing creature, or an unattended object
can slide along solid ground as if on smooth ice.
If you manifest skate on yourself or another creature,
the target of the power retains equilibrium by mental desire alone,
allowing them to gracefully skate along the ground,
turn, or stop suddenly as desired.
The skater's base movement speed increases by 15 feet.
As with any effect that increases speed,
this power affects the target's maximum jumping distance.

The target can skate up or down any incline or decline
they could normally walk upon without mishap,
though skating up an incline reduces the target's speed to normal,
while skating down a decline increases their speed
by an additional 15 feet.

If you manifest skate on an object,
treat the object as having only one-tenth of its normal weight
for the purpose of dragging it along the ground.

  % Pk 4
\DndPowerHeader%
  {Steadfast Perception}
  {4th-level Prescience}
  {1 action}
  {Self}
  {MB 4, PD \lvlfour}
  {Concentration}
Your vision cannot be distracted or misled,
granting you immunity to all figments and visual deceptions
(such as invisibility),
but not mental deceptions.
Moreover, your Wisdom (Perception) checks
are made with advantage for the duration of this power.

  % P1, PK1
\DndPowerHeader%
  {Stillness of Mind}
  {1st-level Voice}
  {1 reaction, which you take when making a Wisdom saving throw}
  {Self}
  {MB 2, PD \lvlone}
  {Instantaneous}
You immediately empty your mind of all transitory and distracting thoughts,
improving your self-control.
You gain a +2 bonus on the Wisdom save which triggered this power.
\augment{For every 2 additional psi dice you spend, the bonus to your
  Wisdom saving throws increases by 1.}

\DndPowerHeader%
  {Stunning Wave}
  {2nd-level Voice}
  {1 action}
  {10 feet}
  {MB 2, PD \lvltwo}
  {Instantaneous}
You generate a stunning mental wave that instantly
sweeps out from your location.
All creatures you designate within 10 feet of you
(you can choose certain creatures to be unaffected)
must succeed on a Wisdom saving throw
or become dazed for until the start of their next turn.
\augment{You can augment this power in one or both of the following ways.
\begin{itemize}
  \item For every 2 additional psi dice you spend,
        this power's save DC increases by 1.
  \item For every 2 additional psi dice you spend,
        this power's range and the radius of its area both increase by 5 feet.
\end{itemize}}

\DndPowerHeader%
  {Sublime Insight}
  {9th-level Prescience}
  {1 hour}
  {Self}
  {MB 9, PD \lvlnine}
  {Instantaneous}
You elevate your mind to a near-universal consciousness,
cogitating countless impressions and predictions
involving any creature you have seen before,
whether personally or by means of another power
such as \power{psionic scrying}.

This process gives you an uncannily accurate vision
of the creature's nature,
activities,
and whereabouts.
When you manifest the power,
you learn the following facts about the creature.
\begin{itemize}
  \item Its name, race, alignment, and character class.
  \item A general estimate of its level and hit points.
  \item Its location (including place of residence, town,
        country, world, and plane of existence).
  \item Significant items currently in its possession.
  \item Any significant activities or actions the creature
        has undertaken in the previous 8 hours,
        including details such as locales travelled through,
        the names or races of those the creature fought,
        spells it cast,
        items it acquired,
        and items it left behind
        (including the location of those items).
  \item An instantaneous mental view of the creature's
        immediate state.
\end{itemize}

Sublime Insight can defeat spells,
powers,
and special abilities such as \spell{mind blank}
(or even a \spell{wish} spell)
that normally obscures Prescience powers.
You can attempt a check---adding your
Intelligence modifier to a d20 roll---against
a DC of 6 + the caster level of the
creator of the obscuring effect)
to defeat these sorts of otherwise impervious defences.

This power is not guaranteed to work against gods, divine beings
and other supremely powerful beings.

% PK6, P6
\DndPowerHeader%
  {Suspend Life}
  {6th-level Prana Bindu}
  {1 action}
  {Self}
  {MB 5, PD \lvlsix}
  {Permanent unless ended or dismissed, see text}
Your mastery over your body allows you to slow your
metabolism to a point
that you are almost in suspended animation.
Even powers that detect life or thought are
incapable of determining that you are alive.

While you are suspended,
you are aware of your surroundings.
You feel the passage of one day
for every year that actually passes.
Though on a slower schedule,
you grow hungry after a `day' without food
(though a year passes in actuality)
and begin to suffer the effects of
thirst and starvation as appropriate.

If you take any damage,
you come out of your trance 4 rounds later.
The trance can also be ended by a
successful use of dispel psionics.
If you choose to dismiss the power,
your trance ends 1 minute later.

\DndPowerHeader%
  {Swarm of Crystals}
  {2nd-level Metacreativity}
  {1 action}
  {Self (15-foot cone)}
  {MB 2, PD \lvltwo}
  {Instantaneous}
Thousands of tiny crystal shards spray forth in an arc from your hand.
These razor-like crystals slice everything in their path.
Any creature caught in the cone takes 4d4 points of slashing damage.
\augment{For every additional psi dice you spend,
  this power's damage increases by 1d4.}

\DndPowerHeader%
  {Swiftsteed}
  {2nd-level Prana Bindu}
  {1 bonus action}
  {Self}
  {MB 2, PD \lvltwo}
  {Until the end of your current turn}
Your body more efficiently converts its anaerobic
reserves of energy into a swift, immediate burst of speed.
When you manifest this power as a bonus action,
your base movement speed is doubled until the end
of your turn.

  \DndPowerHeader%
    {Teleport, Psionic}
    {7th-level Spacefolding}
    {1 action}
    {10 feet}
    {MB 7, PD 13}
    {Instantaneous}
  As the \spell{teleport} spell, except as noted here.
  Knowledge of a permanent teleportation circle and
  its sigil sequence confers no benefit,
  since your teleportation works by psionic 
  and not magic means.

  % PK1, PB1
\DndPowerHeader%
  {Thicken Skin}
  {1st-level Prana Bindu}
  {1 action}
  {Self}
  {MB 1, PD \lvlone}
  {Concentration, up to 10 minutes}
Your skin or natural armor thickens
and spreads across your body,
providing a +2 bonus to your AC.
\augment{You can augment this power in one
  or both of the following ways.
\begin{itemize}
  \item For every 4 additional power points you spend,
    the AC bonus increases by 1.
  \item If you spend 6 additional power points,
    you can manifest this power as a reaction to being hit.
\end{itemize}}

\DndPowerHeader%
  {Third Eye}
  {5th-level Prescience}
  {1 action}
  {Self}
  {MB 5, PD \lvlfive}
  {Concentration, up to 1 minute}
Your vision awakens to those that were hitherto
concealed from you.
You gain \emph{truesight} out to 120 feet.

\DndPowerHeader%
  {Time Warp}
  {3rd-level Spacefolding}
  {1 action}
  {Touch}
  {MB 3, PD \lvlthree}
  {1 round per psionic threshold}
A creature of Medium size or smaller
you touch when manifesting this power makes a Wisdom saving throw.
If they fail, they move forward in time for a number of rounds
equal to your psionic threshold.
In effect, the target seems to disappear in a shimmer
of silver energy,
then reappear after the duration of this power expires.
The target reappears in exactly the same
orientation and condition as before.
From the target's point of view,
no time has passed at all.

In each round of the power's duration,
on what would have been the target's turn,
it can attempt a Wisdom saving throw.
Success allows the target to return and the power ends.
When the target reappears---whether by succeeding on the save
or the power's duration expiring---the target
can act normally on their next turn.

If the space from which the target departed is occupied
upon their return to the time stream,
they appear in the closest unoccupied space,
still in their original orientation.
Determine the closest space randomly if necessary.
\augment{You can augment this power in one or both of the following ways.
\begin{itemize}
  \item For every 2 additional psi dice you spend,
      you can affect a creature of one size category larger.
  \item For every 2 additional psi dice you spend,
      this power can affect an additional target.
      Any additional target cannot be more than
      15 feet from another target of the power.
\end{itemize}}

\DndPowerHeader%
  {Timeless Body}
  {9th-level Spacefolding}
  {1 action}
  {Self}
  {MB 9, PD \lvlnine}
  {Concentration, up to 1 minute}
You carve out a isolated pocket in time,
rendering your body impervious to the outside world.
While this power lasts,
you are invulnerable to all damage and the effects
of all attacks, spells, powers
and environmental conditions
(whether harmful or helpful).

\DndPowerHeader%
  {Tornado}
  {9th-level Psychokinesis}
  {1 action}
  {120 feet}
  {MB 9, PD \lvlnine}
  {Instantaneous}
You induce the formation of a slender vortex
of fiercely swirling air.
When you manifest it,
a vortex of air visibly and audibly snakes out from your
outstretched hand.

If you want to aim the vortex at a specific creature
you can see in range,
you can make a psionics attack to strike the creature.
If you succeed,
direct contact with the vortex deals 8d6 bludgeoning damage
to the creature.
Otherwise,
you may pick any point you can see in range.

Regardless of whether your psionics attack hits
(and even if you forgo the attack),
all creatures
(including the one possibly damaged by direct contact)
within a cylinder of radius 40 feet
and height 100 feet
are picked up and violently dashed about,
dealing 17d6 bludgeoning damage to each one.
Creatures that make a successful Dexterity save
take half damage.

After being dashed about,
each creature that was affected finds itself situated
in a new space 1d4 times 10 feet
away from its original space in a random direction.
Walls and other barriers can restrict this relocation;
in such a case,
the creature ends up adjacent to the barrier.

\DndPowerHeader%
  {Touchsight}
  {3rd-level Prana Bindu}
  {1 action}
  {Self}
  {MB 3, PD \lvlthree}
  {Concentration, up to 1 minute}
You generate a subtle telekinetic field of mental contact,
allowing you to `feel' your surroundings even in total darkness
or when your sight would otherwise be obscured
by your physical environment.
Your touchsight field emanates from you out to 60 feet.
You ignore invisibility, darkness, and concealment,
though you must have line of sight to a creature or an object
to discern it.

You do not need to make Wisdom (Perception) checks
to notice creatures; you can detect and pinpoint all creatures
within 60 feet.
In many circumstances, comparing your regular senses
to what you learn with touchsight is enough to tell you
the difference between visible, invisible, hiding,
and concealed creatures.
\augment{For every 2 additional psi dice you spend,
  the radius of your touchsight field increases by 10 feet.}

\DndPowerHeader%
  {Tower of Power}
  {5th-level Voice}
  {1 reaction, which you take when you or an ally is targeted by a Voice or mind-affecting power}
  {15 feet}
  {MB 4, PD \lvlfive}
  {Concentration, until the start of your next turn}
You generate a bastion of thought so strong
that it offers protection to you and all creatures around you,
improving the self-control of all.
You and all creatures you choose within 15 feet of you
can add an extra 2d6 to saving throws against all Voice powers
and powers that affect one's mind.
The power lasts until the start of your next turn.
\augment{For every additional psi die you spend,
  this power's duration increases by 1 round.}

\DndPowerHeader%
  {Trace Teleport}
  {4th-level Prescience}
  {1 action}
  {Self}
  {MB 3, PD \lvlfour}
  {Instantaneous}
As for the \power{detect teleportation} power,
except you can trace the destination of any psionic
or magical teleportation made by others within
a 60-foot radius from you
and that occurred the last minute.

You know the direction and distance the individuals travelled
as if you had casually seen the location.
This power does not grant you any information on the conditions
at the other end of the trace
beyond the mental coordinates of the location.
\augment{If you spend 2 additional psi dice,
  this power's range increases to 120 feet.}

\DndPowerHeader%
  {Unrelenting Voice}
  {3rd-level Voice}
  {1 action}
  {Self (30-feet cone)}
  {MB 3, PD \lvlthree}
  {Instantaneous}
Your Voice bursts forth from you as tidal wave of raw force,
knocking creatures down.
Any creature that lies within a 30-foot cone centred on yourself
must make a Dexterity or Strength saving throw (their choice),
taking 8d4 force damage on a failed saving throw or half as much on
a successful roll.
In addition, each target that failed the save is pushed 10 feet back
and knocked prone.
\augment{For every additional psi die you spend,
  this power's damage increases by 1d4.}

\DndPowerHeader%
  {Vigour}
  {1st-level Prana Bindu}
  {1 action}
  {Self}
  {MB 1, PD \lvlone}
  {1 hour}
You suffuse yourself with power,
gaining 5 temporary hit points.
\augment{For every additional psi dice you spend, the number of
  hit points you gain increases by 5.}

  % PK2
\DndPowerHeader%
  {Wall Walker}
  {2nd-level Spacefolding}
  {1 action}
  {Self}
  {MB 2, PD \lvltwo}
  {Concentration, up to 1 minute}
You reverse the direction of gravity in your immediate vicinity,
allowing you to walk on vertical surfaces
or even traverse ceilings
(you need not make any ability checks to traverse these surfaces).
Because of the need to keep at least one foot in contact
with the wall or ceiling at all times,
you cannot jump or take the dash action,
and all walls and ceilings walked on in this way count as
difficult terrain.

\DndPowerHeader%
  {Weapon of Energy}
  {4th-level Psychokinesis}
  {1 action}
  {Touch}
  {MB 4, PD \lvlfour}
  {Until the end of your next turn}
You can manifest this power to
energise a weapon you are holding.
The weapon deals an extra 2d8 points of cold,
lightning, fire or thunder damage
(as chosen by you at the time of manifestation)
on a successful hit.

This power can be manifested on a weapon that
is magic or otherwise already deals damage the
aforementioned types of damage,
and if the weapon already deals the same type of damage
as the power,
the effects stack.
If this power is manifested on a weapon
already benefiting from the effect of the power,
the newer manifestation supersedes the older manifestation.

\chapter{Classes}
\label{chap:classes}

\section{Psion}

In the audience chamber of the Duke of Baradstone,
a half-elf pries into the pathways of Duke's mind---their
thoughts, feelings, desires revealed to her
with lucid precision.
All the while, the Duke's retinue remain totally oblivious
to the psionic intrusion.

In an vanishing instant of time,
the githzerai projects the movements of his
githyanki foes two seconds into the future,
giving himself the edge he needs desperately needs
to stop their advance.

The tiefling crouches imperceptibly behind the garrison,
recalling her Prana Bindu training regimen to force
her body to make no sound.
Then, she focuses on the sword carried by the guardsmen;
before he takes his next step,
the steel burns a brilliant red before exploding in his
very hand.

In the underbelly of the city,
the dwarf brushes his hand against the ancient stone.
Its feeling, its texture, ignite a subconscious
awareness within his mind,
and he is borne back to the memory of countless aeons
past.
He watches in striking detail as the genetic memory unravels,
showing his forebears laying the same stone---one piece
of a much larger city which was lost to the entropic
assault of time.

\subsection{Masters of the Mind}
While psions are diverse in their abilities and training,
all are unified by a common thread---their unparalleled
mastery over the latent powers that lie within their mind.
While some turn to the magic essence which suffuses the cosmos,
psions peer inwards,
unlocking powers which might have laid dormant over many years.
Only through strict training and profound mental discipline
can these powers be unlocked,
which all psions have been through in one way or another.

Psions manifest distinct effects known as powers, which,
depending on their discipline, achieve myriad effects.
Powers are integral to the effectiveness of a psion,
and the more experience they accrue,
the more powers they unlock from deep within.

\subsection{Creating a Psion}
Creating a character with the psion class prompts
important questions.
How did your character first become aware of the
powers of the mind?
Perhaps they found an old, forgotten tome
describing the \emph{Old Way} and the monks who adhered
to it.
Or they might have been awoken to it abruptly,
awakening a dormant power which manifested fleetingly.

One additional line of inquiry should be about the nature
of their training.
How did your character hone their psionic abilities?
Did they train with the githzerai in their floating citadels
on Limbo?
Did they live with a reclusive order who have knowledge
of psionic powers?
Or did they train by themselves,
experimenting with their mind through trial and error?

\subsubsection{Quick Build}
You can make a psion quickly by following these suggestions.
First, Intelligence should be your highest ability score,
followed by Constitution.
Second, choose the following three 1st-level psion powers:.

% \DndQuote%
%   {`Sometimes, what you need and what you}
%   {want turn out to be the same thing: An uplifting quote.'}
%   {The Adventurer}

  

\begin{figure*}[t]
    \begin{ornamentedtabular}{c c l c c c c}[title={The Psion}]
        \textbf{Level} & \textbf{P. Bonus} & \textbf{Features} & \textbf{Psi Dice} & \textbf{Mental Bandwidth} & \textbf{Powers Known} & \textbf{Psionic Threshold} \\
        1st  & +2 & Powers, Discipline        & 2   & 2  & 3  & 1 \\
        2nd  & +2 & Microsleep (one use)      & 5   & 2  & 5  & 1 \\
        3rd  & +2 & ---                       & 10  & 3  & 7  & 2 \\
        4th  & +2 & Ability Score Improvement & 13  & 3  & 9  & 2 \\
        5th  & +3 & ---                       & 23  & 4  & 11 & 3 \\
        6th  & +3 & Discipline Tenet          & 28  & 4  & 13 & 3 \\
        7th  & +3 & Mind Palace               & 35  & 4  & 15 & 4 \\
        8th  & +3 & Ability Score Improvement & 42  & 5  & 17 & 4 \\
        9th  & +4 & ---                       & 58  & 5  & 19 & 5 \\
        10th & +4 & Discipline Tenet          & 67  & 6  & 21 & 5 \\
        11th & +4 & Microsleep (two uses)     & 78  & 6  & 22 & 6 \\
        12th & +4 & Ability Score Improvement & 78  & 6  & 24 & 6 \\
        13th & +5 & ---                       & 91  & 7  & 25 & 7 \\
        14th & +5 & Discipline Tenet          & 91  & 7  & 27 & 7 \\
        15h  & +5 & ---                       & 106 & 7  & 28 & 8 \\
        16th & +5 & Ability Score Improvement & 106 & 8  & 30 & 9 \\
        17th & +6 & ---                       & 123 & 8  & 31 & 9 \\
        18th & +6 & ---                       & 132 & 8  & 33 & 9 \\
        19th & +6 & Ability Score Improvement & 143 & 9  & 34 & 9 \\
        20th & +6 & Ascendance                & 156 & 10 & 36 & 9 \\
    \end{ornamentedtabular}
\end{figure*}

\subsection{Class Features}
As a psion, you gain the following features.

\subsubsection{Hit Points}

\listparagraph[4pt]{Hit Dice}{
    1d6 per psion level
}
\listparagraph{Hit Points at 1st Level}{
    6 + your Constitution modifier
}
\listparagraph{Hit Points at Higher Levels}{
    1d6 (or 4) + your Constitution modifier per psion level
    after 1st
}

\subsubsection{Proficiencies}

\listparagraph[4pt]{Armour}{
    Light armour
}
\listparagraph{Weapons}{
    Clubs, daggers, heavy crossbows, light crossbows,
    quarterstaffs, spears
}
\listparagraph{Tools}{
    None
}
\vspace{6pt}
\listparagraph{Saving Throws}{
    Intelligence, Wisdom
}
\listparagraph{Skills}{
    Choose two from History, Religion, Nature,
    Insight and Perception
}

In addition, depending on the psion's chosen discipline,
they acquire an additional skill proficiency.
This is shown in the table below.
\begin{table}[htbp]%
    \begin{DndTable}[width=\columnwidth,
                     header=Additional Profciency]{
                     X X}
        Discipline & Proficiency \\
        Prescience & Investigation \\
        Prana Bindu & Athletics, Acrobatics or Stealth \\
        Voice & Persuasion \\
        Metacreativity & Survival \\
        Spacefolding & Arcana \\
        Psychokinesis & Intimidation
    \end{DndTable}
\end{table}

\subsubsection{Equipment}
You start with the following equipment,
in addition to the equipment granted by your background:
\begin{itemize}
    \item (a) a club or (b) a dagger
    \item a light crossbow and 20 bolts
    \item (a) a scholar's pack or (b) a dungeoneer's pack
    \item Leather armour
\end{itemize}

\subsection{Powers}
\subsubsection{Psi Dice}
The number of powers that you can manifest
is limited by the number of psi dice that
you have available.
This is the principal limit on the output of a psion.
For example, a 9-th level psion with 72 psi dice
can manifest a power costing 1 psi dice 72 times.

The number of available psi dice per level
is shown in the psion class table.
You regain all expended psi dice when you finish
a long rest.

\subsubsection{Mental Bandwidth}
The powers you can manifest without straining yourself
is reflected by your mental bandwidth.
The total number of mental bandwidth slots you have
is shown in the class table.

\subsubsection{Discipline}
At 1st level,
you must decide upon a discipline.
Your discipline grants you a specific sublist of powers
which cannot be accessed by other disciplines.
This is called the \define{discipline sublist}.
In addition, your discipline grants you different abilities
when you reach certain levels.
These are called \define{tenets}.

\subsubsection{Powers Known}
At 1st level,
you choose three psion powers of your choice
from either the psion class power list
or your chosen discipline sublist.
You cannot choose powers from discipline sublists
other than that of the discipline you have nominated.

\subparagraph{Psionic Threshold}
You cannot learn nor manifest powers with a level
greater than your psionic threshold,
shown in the class table.
As mentioned in \secref{sub:augmenting},
you also cannot augment powers to a level
beyond your psionic threshold.

\subsubsection{Manifesting Ability}
Your psionic ability modifier is your Intelligence modifier.
The save DC against your psionic powers and your
psionic attack modifier are therefore respectively:
\small\begin{equation*}
    \begin{gathered}
        \text{\textbf{Psionics save DC}}
            = 8 + \text{your proficiency bonus} + \\
                  \text{your Intelligence modifier} \\
        \text{\textbf{Psionics attack modifier}}
            = \text{your proficiency bonus} + \\
              \text{your Intelligence modifier}
    \end{gathered}
\end{equation*}\normalsize

\subsection{Microsleep}
Starting at 2nd level,
as a bonus action on your turn,
you may recover a number of psi dice equal to
your Intelligence modifier multiplied by your psionic threshold.

Once you use this feature,
you must finish a long rest before you can use it again.
Starting at 11th level,
you can use it twice before a long rest.

\subsection{Ability Score Improvement}
When you reach 4th level,
and again at 8th, 12th, 16th and 19th level,
you can increase one ability score of your choice by 2,
or you can increase two of your ability scores by 1.
As normal,
you can't increase an ability score above 20 using this feature.

\subsection{Mind Palace}
Starting at 7th level,
your mind becomes your sanctuary---a palace constructed
from pure thought.
Whenever you are asleep,
you return to this sanctuary,
which exists in a perfect equilibrium between
external and internal influences.
You are fully alert to your surroundings
while you are asleep.
In addition, you can communicate telepathically to other
creatures who are also sleeping within 30 feet of you,
in which case you appear as a character in their dreams.

\subsection{Ascendance}
Starting at 20th level,
your body becomes only a vessel for the vast potential
of your mind.
Your mind alters your biological makeup to better resist damage,
granting you resistance to bludgeoning, piercing and slashing damage.
In addition,
you no longer age,
but you can still be killed otherwise.

\subsection{Prescience Tenets}

\subsubsection{Precognitive Step}
% phb3(4e) 89
% prescience
Starting at 6th level,
you can peer through the mists obscuring the future,
learning how one event unfolds.
At the start of every encounter, roll a d20.
Once during the same encounter,
you can at any time replace one of your attack rolls,
saving throws or skill checks with that roll,
adding modifiers on top of it as usual.
You can also replace an enemy's attack roll against you.

\subsubsection{Epitaph}
Starting at 10th level,
your hone your ability to see through time.
Once during an encounter,
if a hostile creature takes a turn immediately following yours,
you may use a free action at the start of your turn
to ask the DM what the creature will do on their turn,
as conditioned by your explicit choices.
For example,
you may ask the DM what the creature will do
if you move 30 feet in a given direction and manifest a power.
If you deviate at all from your stated conditions,
or if you leave any ambiguity or uncertainty as to those choices,
the prediction may not necessarily come true.

If multiple hostile creatures follow your turn,
you may use this ability for each of those creatures
until the turn order reaches a friendly creature
or the round ends.

\subsubsection{Other Memory}
Starting at 14th level,
your unlock the hereditary memories of your forebears,
allowing you to use their knowledge in your time of need.
Once per day, as an action you may call upon a particular ancestor
with a certain background as shown in the table below.
The nature of the benefit conferred to you depends on the background.
The benefits conferred to you end when you finish a long rest.

\begin{table}[htbp]%
    \begin{DndTable}[width=\columnwidth,
                     header=Other Memory Benefit]{
                     l X}
        Background          & Benefit                                                                   \\
        Warrior             & Gain proficiency in all armours,
                                martial weapons and Strength (Athletics) \\
        Battlemage          & You learn three 3rd-level or lower
                                evocation spells of your choice
                                which can be cast without expending a spell slot;
                                once you cast any of these spells,
                                you can't cast the same spell again;
                                the spellcasting ability for these
                                spells is Intelligence \\
        Commander           & You learn three Uberium faction abilities \\
        Thief               & You gain an extra bonus action \\ 
        Priest              & You learn three 3rd-level or lower
                                spells from the cleric spell list
                                of your choice;
                                these can be cast without expending a spell slot;
                                once you cast any of these spells,
                                you can't cast the same spell again;
                                the spellcasting ability for these
                                spells is Intelligence \\\
        Musician            & You gain proficiency in a musical instrument
                                and Charisma (Performance), or expertise if
                                you are already proficient in that skill
    \end{DndTable}
\end{table}

If you spend subsequent days relying on the same ancestor
and their abilities,
you run the risk of having your personality usurped by them.
For each consecutive day you use the same ancestor,
make a Wisdom saving throw with DC equal to 10 plus the number
of consecutive days.
If you fail the save,
hand your character sheet to the DM.

\subsection{Prana Bindu Tenets}
\label{sub:prana_bindu_tenets}

\subsubsection{Lightning Warfare}
Starting at 6th level,
your mastery of nerve and muscle
allows you to move at incredible speeds.
Your movement speed increases by 20 ft.
In addition, your movement no longer provokes
opportunity attacks.

\subsubsection{Optimised Reflexes}
Starting at 10th level,
your reaction times become almost negligible.
You gain an additional reaction on top of the one
you already have, which can be used in the same way
as the latter.

\subsubsection{Alter Metabolism}
Starting at 14th level,
your mastery over body now extends to your metabolism,
and you are able to by thought alone
alter the chemical makeup of malign substances.
You become immune to the effects of all poisons,
poison damage, the poisoned condition, and any toxic venoms
or secretions.
You also gain resistance to acid damage,
and you become immune to disease.

Additionally, you become immune to the maluses of extreme heat,
extreme cold and high altitude,
unless the conditions are so severe
that the DM decides otherwise.

\subsection{Voice Tenets}

\subsubsection{Inner Voice}
Starting at 6th level,
you gain telepathy up to 120 feet.
Once per day,
you may also send a message up to 10 words long
to a creature you have seen before and that is
presently within 10 miles of you.

\subsubsection{Subtle Voice}
Starting at 10th level,
your mastery of the Voice allows you to manifest Voice
powers without arousing suspicion.
Any non-telepathic Voice power---which ordinarily alerts
creatures that can hear you to your Voice---now is
indistinguishable from normal speech.
\emph{Exception:} Psions who have chosen the Voice discipline
cannot be affected by Subtle Voice.

\subsubsection{Enthrall Creature}
Starting at 14th level,
you gain the ability to dominate other creatures and have them
carry out your bidding.
Twice per day,
you may manifest the \spell{geas} spell as a power,
which counts as a 5th-level Voice power with no PD or MB cost.
Subtle Voice also applies for this power.

\subsection{Metacreativity Tenets}
\label{sub:metacreativity_tenets}
\subsubsection{Create Golem}
\label{subs:menu_a}
Starting at 6th level,
you learn the \power{psionic golem} power,
which allows you to manifest a psionic golem
from any raw material.
You are mentally linked to the psionic golem,
and they answer to your bidding as described in the power.
The statistics of the golem depends on the level
at which \power{psionic golem} is cast.
For example,
casting the power at 3rd-level creates a 3rd-level psionic golem.
The statblocks by golem level are given in \secref{sec:psionic_golem}.

Each manifested golem has a certain number of
\define{metacreative points} (MP).
This is called the \define{MP potential} of the golem.
Metacreative points allow you to augment the golem
with particular abilities.
At 6th level,
you have access to the abilities given in Menu A (see the table below),
all of which have an MP cost of 1.
You cannot add abilities to the golem such that the total MP cost
is above the MP potential of the golem.
For example,
a 4th-level golem with an MP potential of 2
can have two Menu A abilities.

\begin{table}[htbp]%
    \begin{DndTable}[width=\columnwidth,
                     header=Psionic Golem Abilities (Menu A)]{
                     X X}
        Name                & Benefit   \\
        Resilient           & The golem gains an extra 10 hit points \\
        Sturdy              & The golem gains a +2 bonus to its AC \\
        Vertical Propulsion & The golem gains a fly speed equal to 15 feet \\ 
        Celerity            & The golem gains the mobile feat \\
        Resistance          & The golem gains resistance to fire, lightning, cold
                                or thunder damage \\
        Seaworthy           & The golem gains a swim speed of 30 feet \\
        Aggressive          & If the golem hits a creature with a melee attack,
                                it can use its reaction to attempt to shove
                                that creature prone
    \end{DndTable}
\end{table}

\subsubsection{Improved Psionic Golem}
\label{subs:menu_b}
Starting at 10th level,
you gain access to psionic golem abilities on Menu B,
as shown in the table below.
The abilities on Menu B cost 2 MP.

\begin{table}[htbp]%
    \begin{DndTable}[width=\columnwidth,
                     header=Psionic Golem Abilities (Menu B)]{
                     X X}
        Name                & Benefit   \\
        Energy Touch        & The golem deals an extra 1d4 points of
                                fire, cold, electricity or thunder damage when it
                                hits with a melee attack \\
        Extra attack        & Whenever the golem takes the \define{Multiattack} action,
                                it can make one additional slam attack \\
        Auto-reassembly     & The golem heals 5 hit points each round \\ 
        Heavy Deflection    & The golem gains a +3 bonus to its AC  \\
        Highly Resilient    & The golem gains an extra 15 hit points \\
        Improved Critical   & The golem lands a critical hit on a roll of either
                                19 or 20 \\
        Muscular            & The golem gains a +4 bonus to its Strength score \\
    \end{DndTable}
\end{table}

\subsubsection{Heart of Ascension}
\label{subs:menu_c}
Starting at 14th level,
any golem you manifest with \power{psionic golem}
becomes Ascendant.
You gain access to psionic golem abilities on Menu C,
as shown in the table below.
The abilities on Menu C cost 4 MP.

\begin{table}[htbp]%
    \begin{DndTable}[width=\columnwidth,
                     header=Psionic Golem Abilities (Menu C)]{
                     X X}
        Name                & Benefit   \\
        WTF is Truesight!?  & The golem gains truesight out to 60 feet \\
        Supreme Resilience  & The golem gains an extra 30 hit points \\
        Metal Carapace      & All incoming damage from separate sources
                                is reduced by 5 \\ 
        Shield!             & The golem gains a +5 bonus to its AC \\
        Active Camouflage   & The golem becomes naturally invisible \\
        Energy Bolt         & The golem can manifest the \power{energy bolt} power
                                as an action once per round; the energy type is
                                chosen when the golem is initially manifested
    \end{DndTable}
\end{table}

\subsection{Spacefolding Tenets}
\subsubsection{Delete Space}
Starting at 6th level,
you become capable of removing the space between you
and another creature with your mind alone. 
Choose a creature you can see within 60 feet and spend 5 psi dice.
That creature is pulled up to 30 feet in your direction as the space
between you and them is eliminated.

\subsubsection{Wormhole}
Starting at 10th level,
you alter the fabric of space and time such that
you can link two different locations together.
Once per day as an action,
you may create a wormhole mouth at a point you are standing on.
The mouth is a 10-foot radius sphere and is presently closed.
At a later point in time,
you may create another mouth at a different location.
The location need not be on the same plane of existence
as the mouth.
When you do so, the two mouths become adjacent to each other.
Creatures on either side can see to the other side of the wormhole
through the 10-foot sphere,
and can step through using their movement.

You may choose to collapse the wormhole at any time.
However, it is not stable.
For every hour that has elapsed since you placed the second mouth,
roll a d10.
On a roll of 1--5,
the wormhole collapses and the bridge between the two locations
is lost.
You must then create a new wormhole.

\subsubsection{Master of Spacetime}
Starting at 14th level,
you can effortlessly warp space and time locally to
move yourself across the battlefield.
The \power{psionic step} power costs no psi dice and MB
whenever you manifest it.
The power still costs 4 psi dice to augment,
however if you do so,
you can manifest the power as a free action
instead of a bonus action
(limit of once per turn).

\subsection{Psychokinesis Tenets}
\subsubsection{Sculpt Psioncs}
Starting at 6th level,
akin to the evocation wizard's ability to sculpt spells,
you can alter the direction that hazardous effects
created by your psychokinesis powers propagate.
When you manifest a psychokinesis power that affects
other creatures that you can see,
you can choose a number of them equal to your
psionic threshold.
The chosen creatures automatically succeed on any saving
throws they make to resist the power,
taking no damage on a successful save if they would
otherwise take half.
If the power does not prompt a saving throw
(e.g. if it automatically hits),
the chosen creatures also take no damage.

\subsubsection{Signature Damage}
Starting at 10th level,
you become highly proficient in dealing damage
with a specific energy manifested by your psionic powers.
Choose a damage type from fire, cold, lightning and thunder.
Each die of damage you deal with the chosen type
become \define{exploding die}.
If you roll the maximum result on an exploding die,
you can reroll that die again and add the result to the
total damage.
If the result is again the maximum,
you may `explode' the die again.
This continues indefinitely or until you have
no longer rolled the maximum over all dice.

\subsubsection{Signature Power}
Starting at 14th level,
one psychokinesis power become so familiar to you that you
manifest it without straining yourself.
Choose one 3rd-level psychokinesis power.
The MB cost for that power is reduced to 0 whenever you manifest it.
In addition, you can manifest this power without expending
any psi die.
Once you do so,
you can't do so again until you finish a short or long rest.

\clearpage\section{Psi Knight}

With a thought alone, the silver-armoured half-orc effortlessly
shunts aside the first gnoll before plunging his longsword
into the abdomen of the next.
Then, with near-imperceptible speed, he appears behind the third gnoll,
who scarcely processes the gruesome end he meets.

In the silver-purple depths of the Astral Plane,
the githyanki astral skiff careens into the illithid nautiloid.
Before long, the erstwhile slaves of shattered Astromundi Nova
engage the mind flayers,
battling each other with thoughts alone.
Then the githyanki vanguard charges,
their silver swords slicing into the exposed flesh of their former
masters.

In the subterranean halls of a long-forgotten city,
the elf uses her mind to propel herself across a seemingly
bottomless crevasse.
Then, landing on the other side,
she shapes the very earth itself into a glowing handaxe,
which she picks up and throws at the awakened undead
bearing down upon her.

\subsection{The Body and the Mind}
Psi Knights supplement their martial abilities
with the latent power of their mind to
achieve victory over their enemies.
Formidable opponents,
psi knights are rare and highly accomplished warriors
who achieve an almost unparalleled balance between mind
and body.
Adventuring parties often readily accept a noble psi knight
by their side,
and the psi knight's enemies on the battlefield tremble
as they approach.

Like psions, psi knights manifest powers with myriad effects.
While they cannot manifest as many powers as psions,
they have substantial martial capacity
and have a greater ability to weather the tide of battle. 

\subsubsection{Creating a Psi Knight}
When creating a psi knight,
ask yourself why your character has chosen the path of psionics.
Perhaps they belonged to a small community of like-minded warriors
who understood the untapped potential of psionic powers.
Or perhaps they ventured down a solitary path,
testing the powers they learnt piecemeal in the din of battle.

Also ask yourself how they interact with the broader community.
Do they see themselves as an ordinary person with a fortuitous
talent?
Or do they feel that they rise above the masses?

\subsubsection{Quick Build}
You can make a psi knight quickly by following these suggestions.
First, Strength or Dexterity should be your highest ability score,
followed by Wisdom then Constitution.
Second, choose the soldier background.
Third, pick the \power{x} power from the
psi knight class power list.

\label{sec:psi_knight}
\begin{figure*}[t]
    \begin{ornamentedtabular}{c c l c c c c}[title={The Psi Knight}]
        \textbf{Level} & \textbf{P. Bonus} & \textbf{Features} & \textbf{Psi Dice} & \textbf{Mental Bandwidth} & \textbf{Powers Known} & \textbf{Psionic Threshold} \\
        1st  & +2 & Powers, Fighting Style    & 0   & 1  & 1  & 1 \\
        2nd  & +2 & Prana Bindu Initiate      & 1   & 1  & 2  & 1 \\
        3rd  & +2 & Psionic Tradition         & 3   & 1  & 3  & 1 \\
        4th  & +2 & Ability Score Improvement & 4   & 2  & 4  & 2 \\
        5th  & +3 & Extra attack              & 6   & 2  & 5  & 2 \\
        6th  & +3 & Psionic Tradition feature & 9   & 2  & 6  & 2 \\
        7th  & +3 & Iron Will                 & 12  & 3  & 7  & 3 \\
        8th  & +3 & Ability Score Improvement & 15  & 3  & 8  & 3 \\
        9th  & +4 & ---                       & 18  & 3  & 9  & 3 \\
        10th & +4 & Psionic Tradition feature & 21  & 3  & 10 & 4 \\
        11th & +4 & Calmness of Self          & 25  & 4  & 11 & 4 \\
        12th & +4 & Ability Score Improvement & 29  & 4  & 12 & 4 \\
        13th & +5 & ---                       & 33  & 4  & 13 & 5 \\
        14th & +5 & ---                       & 37  & 4  & 14 & 5 \\
        15th & +5 & Psionic Tradition feature & 42  & 4  & 15 & 5 \\
        16th & +5 & Ability Score Improvement & 47  & 5  & 16 & 6 \\
        17th & +6 & ---                       & 52  & 5  & 17 & 6 \\
        18th & +6 & Psionic Tradition feature & 58  & 5  & 18 & 6 \\
        19th & +6 & Ability Score Improvement & 64  & 5  & 19 & 6 \\
        20th & +6 & Psionic Reserve           & 70  & 5  & 20 & 6 \\
    \end{ornamentedtabular}
\end{figure*}

\subsection{Class Features}
As a psi knight, you gain the following features.

\subsubsection{Hit Points}

\listparagraph[4pt]{Hit Dice}{
    1d10 per psi knight level
}
\listparagraph{Hit Points at 1st Level}{
    10 + your Constitution modifier
}
\listparagraph{Hit Points at Higher Levels}{
    1d10 (or 6) + your Constitution modifier per psion level
    after 1st
}

\subsubsection{Proficiencies}

\listparagraph[4pt]{Armour}{
    All armour, shields
}
\listparagraph{Weapons}{
    Simple weapons, martial weapons
}
\listparagraph{Tools}{
    None
}
\vspace{6pt}
\listparagraph{Saving Throws}{
    Constitution, Wisdom
}
\listparagraph{Skills}{
    Choose two from Athletics, Acrobatics, Insight,
    Perception, Animal Handling, Intimidation
    and Survival
}

\subsubsection{Equipment}
You start with the following equipment,
in addition to the equipment granted by your background:
\begin{itemize}
    \item (a) chain mail or
          (b) leather armour, longbow
            and 20 arrows
    \item a martial weapon and a shield or
          (b) two martial weapons
    \item (a) a light crossbow and 20 bolts or
          (b) two handaxes
    \item (a) a dungeoneer's pack or
          (b) an explorer's pack
\end{itemize}

\subsection{Fighting Style}
You adopt a particular fighting style as your specialty.
Choose one of the following options.
You can't take a Fighting Style option more than once,
even if you later get to choose again.

\subsubsection{Blind Fighting}
You have blindsight with a range of 10 feet.
Within that range, you can effectively see anything that
isn't behind total cover,
even if you're blinded or in darkness.
Moreover, you can see an invisible creature within that range,
unless the creature successfully hides from you.

\subsubsection{Defence}
While you are wearing armour, you gain a +1 bonus to AC.

\subsubsection{Dueling}
When you are wielding a melee weapon in one hand
and no other weapons,
you gain a +2 bonus to damage rolls with that weapon.

\subsubsection{Great Weapon Fighting}
When you roll a 1 or 2 on a damage die for an attack you make
with a melee weapon that you are wielding with two hands,
you can reroll the die and must use the new roll,
even if the new roll is a 1 or a 2.
The weapon must have the two-handed or versatile property
for you to gain this benefit.

\subsubsection{Interception}
When a creature you can see hits a target,
other than you,
within 5 feet of you with an attack,
you can use your reaction to reduce the damage the target takes
by 1d10 + your proficiency bonus (to a minimum of 0 damage).
You must be wielding a shield or a simple or martial weapon
to use this reaction.

\subsubsection{Protection}
When a creature you can see attacks a target other than you
that is within 5 feet of you,
you can use your reaction to impose disadvantage on the attack roll.
You must be wielding a shield.

\subsection{Powers}
\subsubsection{Psi Dice}
The number of powers that you can manifest
is limited by the number of psi dice that
you have available.
This is the principal limit on the output of a psion.
For example, a 9-th level psi knight with 18 psi dice
can manifest a power costing 1 psi dice 18 times.

The number of available psi dice per level
is shown in the psion class table.
You regain all expended psi dice when you finish
a long rest.

\subsubsection{Mental Bandwidth}
The powers you can manifest without straining yourself
is reflected by your mental bandwidth.
The total number of mental bandwidth slots you have
is shown in the class table.

\subsubsection{Powers Known}
At 1st level,
you choose one psion power of your choice
from the psi knight class power list.

\subparagraph{Psionic Threshold}
You cannot learn nor manifest powers with a level
greater than your psionic threshold,
shown in the class table.
As mentioned in \secref{sub:augmenting},
you also cannot augment powers to a level
beyond your psionic threshold.

\subsubsection{Manifesting Ability}
Your psionic ability modifier is your Wisdom modifier.
The save DC against your psionic powers and your
psionic attack modifier are therefore respectively:
\small\begin{equation*}
    \begin{gathered}
        \text{\textbf{Psionics save DC}}
            = 8 + \text{your proficiency bonus} + \\
                  \text{your Wisdom modifier} \\
        \text{\textbf{Psionics attack modifier}}
            = \text{your proficiency bonus} + \\
              \text{your Wisdom modifier}
    \end{gathered}
\end{equation*}\normalsize

\subsection{Prana Bindu Initiate}
Starting at 2nd level,
your mind-body training makes manifesting powers
belonging to the Prana Bindu discipline
vastly easier.
Prana Bindu powers require one less mental bandwidth slot
than mentioned in their description.

\subsection{Psionic Tradition}
When you reach 3rd level,
you embark on a journey down a particular
psionic tradition.
The tradition represents the aspects of
your mind-body mastery that you have decided
to focus on.
At this stage, two traditions are offered:
the Weirding Way, which focuses on your mastery
of devastatingly precise and supremely fast
Prana Bindu techniques,
and the Commander,
who couples their experience in battle
with profound psionic powers to meticulously
plan and execute combat stratagems.
Your tradition grants you features at 3rd level,
and again at 6th, 10th, 15th and 18th level.

\subsection{Ability Score Improvement}
When you reach 4th level,
and again at 8th, 12th, 16th and 19th level,
you can increase one ability score of your choice by 2,
or you can increase two of your ability scores by 1.
As normal,
you can't increase an ability score above 20 using this feature.

\subsection{Extra Attack}
Beginning at 5th level, you can attack twice,
instead of once,
whenever you take the Attack action on your turn.

\subsection{Iron Will}
Starting at 7th level,
your mind is able to override your body at times
when the untrained would falter.
Whenever you are reduced to hit points,
but not killed outright,
you can spend 5 psi dice to immediately regain hit points
equal to your Wisdom modifier plus the number of levels
you have in the psi knight class.

\subsection{Calmness of Self}
Starting at 11th level,
your unfettered desire to improve mind and body
has granted you a sense of inner calm.
You have resistance to psychic damage.
Moreover, if you start your turn charmed or frightened,
you can expend a 3 psi dice and end every effect on yourself
subjecting you to those conditions.
In addition, you have advantage on saving throws to resist
effects from Voice psionic powers.

\subsection{Psionic Reserve}
Starting at 20th level,
when you roll for initiative and have no psi dice remaining,
you regain 10 psi dice.

\subsection{Weirding Way}
The Weirding Way champions incredible control over one's nerves
and reaction times---granted by Prana Bindu techniques---to
manoeuvre and strike at opponents before they can retaliate.
The name `Weirding Way' comes from the fact that,
to untrained opponents, the movements of those versed in
the Weirding Way appear unnatural and superhuman,
almost as if they are teleporting across the battlefield.

\subsubsection{Prana Bindu Prodigy}
Starting at 3rd level,
you can choose to learn any Prana Bindu tenet from
the psion discipline at \secref{sub:prana_bindu_tenets}.
The level requirement stated there does not apply to you.

\subsubsection{Seize the Initiative}
Starting at 6th level,
whenever a fight breaks out,
you can reliably act before your enemies.
If you spend two psi dice,
you can grant yourself advantage on an initiative roll.

\subsubsection{CQC Dominance}
Starting at 10th level,
you make quick use of off-hand and blunt strikes to
achieve dominance in CQC (`close quarters combat').
You have advantage on all checks when taking the
grapple, shove, shove aside, or overrun action.

In addition,
whenever a creature within 5 feet of you ends their turn,
you can use your reaction and spend 2 psi die
to make a single attack against that creature.
The damage type is replaced with bludgeoning damage,
since you use your off-hand,
a kick, a punch or the blunt end of your weapon
to make the attack.
Whether or not the attack is made with a weapon
or is an unarmed attack is your choice.

\subsubsection{Slippery Movement}
Starting at 15th level,
your incredible agility makes its difficult for your
enemies to pin you down.
You have advantage on all saving throws against effects
which would slow your movement or reduce it to 0.
In addition,
if you become restrained or grappled,
you can use a bonus action on your turn
to attempt to repeat the save
or contest respectively which caused that condition.
You do not have advantage from Slippery Movement
on either of these rolls,
but can have advantage from other sources as usual.

\subsubsection{Lisan al Gaib}
Starting at 18th level,
your perfection of Prana Bindu methods becomes legendary,
and in the eyes of many your are considered a
sublime master of the body and mind.
You may pick an additional Prana Bindu tenet from
\secref{sub:prana_bindu_tenets}.

\subsection{Commander}
The Commander uses their experience and psionic abilities
to control and manipulate the battlefield as they desire.
Enemies fear a psi knight commander,
for they know that they will struggle to ever
gain tactical advantage over them and their allies.

\subsubsection{Wise Strikes}
Starting at 3rd level,
your battlefield experience ensures that your attacks
are as damaging as they can be for your enemy.
You may add your Wisdom modifier to all damage
rolls for melee attacks. 

\subsubsection{Knowledge is Power}
Starting at 6th level,
once per day as an action,
you choose a hostile creature you can see within
30 feet of you.
Your mind links with them,
and you gain a deep understanding of their
strengths, weaknesses and peculiarities.
The DM then hands you the statblock of that creature.
When the encounter ends,
or after 1 in-game minute if an encounter is not occurring,
you must return the statblock.

\subsubsection{Order of Battle}
Starting at 10th level,
your mind's eye expands,
and you begin to see the battlefield from a top-down perspective.
Enemies and allies are pieces on a board, which you can manipulate
with thought alone.
Once per day,
at the start of an encounter but before initiative is rolled,
you may forego initiative rolls and instead
declare the initiative order for all combatants.

This ability does not apply when you are surprised.
You also cannot alter the initiative order of creatures
in the encounter that you have no knowledge of,
those that join later in the fight (such as those
that are summoned),
and environmental effects or lair actions
which act on a set initiative count.

\subsubsection{Gambit}
Starting at 15th level,
your see fighters on a battlefield as pieces on a chessboard
that can be effortlessly manipulated---you simply turn your
mind to them, and they bend to your will.
You have a pool of 120 feet of movement.
On your turn,
you may use your action to attempt to move any number of creatures
cumulatively up to 120 feet in any direction,
including vertically.
Each creature you wish to move can resist this movement by
succeeding a Wisdom saving throw with DC equal to your
psionics save DC.

Any unspent movement remaining in your pool persists
and can be carried over to subsequent turns.
Your pool of movement replenishes whenever you finish a long rest.

\subsubsection{Only a Simulation}
Sometimes the tide of battle can be unpredictable.
Fortunately for you,
starting at 18th level,
your supreme mental abilities confer you the ability to
run combat simulations viewed internally by your mind's eye.
These simulations allow you to decide the best course of action.

Once per day, on your turn,
you can use an action to undo the previous round of combat,
allowing you to `replay' that round.
This ability reverses time to the point in the past
just prior to your previous turn,
undoing the effects of everyone else's actions in the meantime.
Once you have used Only a Simulation,
only you retain knowledge of what happened during the round
that is being replayed; after all, it was just a simulation.
However, you can communicate that knowledge verbally to your companions,
if desired,
and you can act on that privileged knowledge
in any way you desire.

\chapter{Miscellany}
\section{Psionic Golems}
\label{sec:psionic_golem}
The statblocks for the psionic golems described in
\secref{sub:metacreativity_tenets}
are given here, beginning on the following page.

\clearpage\begin{DndMonster}[float*=b,width=\textwidth + 8pt]{1st-Level Psionic Golem}
\begin{multicols}{2}
    \DndMonsterType{Medium construct, unaligned}
  
    \DndMonsterBasics[
        armor-class = {18 (natural armour)},
        hit-points  = {\DndDice{3d10 + 5}},
        speed       = {25 ft.},
      ]
  
    \DndMonsterAbilityScores[
        str = 14,
        dex = 11,
        con = 13,
        int = 1,
        wis = 3,
        cha = 1,
      ]
  
    \DndMonsterDetails[
        damage-immunities = {poison},
        condition-immunities = {blinded, charmed, deafened, exhaustion,
                                frightened, paralyzed, petrified, poisoned},
        senses = {darkvision 30 ft., passive Perception 6},
        languages = {---},
        challenge = 1,
        proficiency = +2
        ]
    % Traits
    \DndMonsterAction{Golem}
    The psionic golem cannot be put to sleep, is immune to disease
    and cannot be raised by necromancy or similar spells.
    The golem also cannot be healed by magic or other means,
    but can be repaired by manifested powers.

    \DndMonsterAction{MP Potential}
    The psionic golem has 1 MP, which may be spent on abilities
    in Menus you have access to.
    
    % Actions
    \DndMonsterSection{Actions}
    \DndMonsterAction{Multiattack}
    The psionic golem makes two slam attacks.
  
    \DndMonsterMelee[
      name=Slam,
      mod=+4,
      reach=5,
      targets=one target,
      dmg=\DndDice{1d6+2},
      dmg-type=bludegoning,
    ]
\end{multicols}
\end{DndMonster}

\begin{DndMonster}[float*=b,width=\textwidth + 8pt]{2nd-Level Psionic Golem}
\begin{multicols}{2}
    \DndMonsterType{Medium construct, unaligned}
  
    \DndMonsterBasics[
        armor-class = {18 (natural armour)},
        hit-points  = {\DndDice{4d10 + 6}},
        speed       = {30 ft.},
      ]
  
    \DndMonsterAbilityScores[
        str = 16,
        dex = 11,
        con = 15,
        int = 1,
        wis = 3,
        cha = 1,
      ]
  
    \DndMonsterDetails[
        damage-immunities = {poison},
        condition-immunities = {blinded, charmed, deafened, exhaustion,
                                frightened, paralyzed, petrified, poisoned},
        senses = {darkvision 30 ft., passive Perception 6},
        languages = {---},
        challenge = 1,
        proficiency = +2
      ]
    % Traits
    \DndMonsterAction{Golem}
    The psionic golem cannot be put to sleep, is immune to disease
    and cannot be raised by necromancy or similar spells.
    The golem also cannot be healed by magic or other means,
    but can be repaired by manifested powers.

    \DndMonsterAction{MP Potential}
    The psionic golem has 1 MP, which may be spent on abilities
    in Menus you have access to.
    
    % Actions
    \DndMonsterSection{Actions}
    \DndMonsterAction{Multiattack}
    The psionic golem makes two slam attacks.
  
    \DndMonsterMelee[
      name=Slam,
      mod=+5,
      reach=5,
      targets=one target,
      dmg=\DndDice{1d6+3},
      dmg-type=bludegoning,
    ]
\end{multicols}
\end{DndMonster}

\begin{DndMonster}[float*=b,width=\textwidth + 8pt]{3rd-Level Psionic Golem}
\begin{multicols}{2}
    \DndMonsterType{Medium construct, unaligned}
  
    \DndMonsterBasics[
        armor-class = {18 (natural armour)},
        hit-points  = {\DndDice{5d10 + 6}},
        speed       = {30 ft.},
      ]
  
    \DndMonsterAbilityScores[
        str = 16,
        dex = 11,
        con = 15,
        int = 1,
        wis = 3,
        cha = 1,
      ]
  
    \DndMonsterDetails[
        damage-immunities = {poison},
        condition-immunities = {blinded, charmed, deafened, exhaustion,
                                frightened, paralyzed, petrified, poisoned},
        senses = {darkvision 30 ft., passive Perception 6},
        languages = {---},
        challenge = 1,
        proficiency = +2
      ]
    % Traits
    \DndMonsterAction{Golem}
    The psionic golem cannot be put to sleep, is immune to disease
    and cannot be raised by necromancy or similar spells.
    The golem also cannot be healed by magic or other means,
    but can be repaired by manifested powers.

    \DndMonsterAction{MP Potential}
    The psionic golem has 1 MP, which may be spent on abilities
    in Menus you have access to.
    
    % Actions
    \DndMonsterSection{Actions}
    \DndMonsterAction{Multiattack}
    The psionic golem makes two slam attacks.
  
    \DndMonsterMelee[
      name=Slam,
      mod=+5,
      reach=5,
      targets=one target,
      dmg=\DndDice{2d6+3},
      dmg-type=bludegoning,
    ]
\end{multicols}  
\end{DndMonster}

\begin{DndMonster}[float*=b,width=\textwidth + 8pt]{4th-Level Psionic Golem}
\begin{multicols}{2}
    \DndMonsterType{Medium construct, unaligned}
  
    \DndMonsterBasics[
        armor-class = {18 (natural armour)},
        hit-points  = {\DndDice{6d10 + 12}},
        speed       = {30 ft.},
      ]
  
    \DndMonsterAbilityScores[
        str = 18,
        dex = 13,
        con = 17,
        int = 1,
        wis = 3,
        cha = 1,
      ]
  
    \DndMonsterDetails[
        damage-immunities = {poison},
        condition-immunities = {blinded, charmed, deafened, exhaustion,
                                frightened, paralyzed, petrified, poisoned},
        senses = {darkvision 30 ft., passive Perception 6},
        languages = {---},
        challenge = 4,
        proficiency = +2
      ]
    % Traits
    \DndMonsterAction{Golem}
    The psionic golem cannot be put to sleep, is immune to disease
    and cannot be raised by necromancy or similar spells.
    The golem also cannot be healed by magic or other means,
    but can be repaired by manifested powers.

    \DndMonsterAction{MP Potential}
    The psionic golem has 2 MP, which may be spent on abilities
    in Menus you have access to.
    
    % Actions
    \DndMonsterSection{Actions}
    \DndMonsterAction{Multiattack}
    The psionic golem makes two slam attacks.
  
    \DndMonsterMelee[
      name=Slam,
      mod=+6,
      reach=5,
      targets=one target,
      dmg=\DndDice{2d6+4},
      dmg-type=bludegoning,
    ]
\end{multicols}  
\end{DndMonster}

\begin{DndMonster}[float*=b,width=\textwidth + 8pt]{5th-Level Psionic Golem}
\begin{multicols}{2}
    \DndMonsterType{Large construct, unaligned}
  
    \DndMonsterBasics[
        armor-class = {19 (natural armour)},
        hit-points  = {\DndDice{7d10 + 12}},
        speed       = {35 ft.},
      ]
  
    \DndMonsterAbilityScores[
        str = 18,
        dex = 15,
        con = 17,
        int = 1,
        wis = 3,
        cha = 1,
      ]
  
    \DndMonsterDetails[
        damage-immunities = {poison},
        condition-immunities = {blinded, charmed, deafened, exhaustion,
                                frightened, paralyzed, petrified, poisoned},
        senses = {darkvision 30 ft., passive Perception 6},
        languages = {---},
        challenge = 5,
        proficiency = +3
      ]
    % Traits
    \DndMonsterAction{Golem}
    The psionic golem cannot be put to sleep, is immune to disease
    and cannot be raised by necromancy or similar spells.
    The golem also cannot be healed by magic or other means,
    but can be repaired by manifested powers.

    \DndMonsterAction{MP Potential}
    The psionic golem has 2 MP, which may be spent on abilities
    in Menus you have access to.

    \DndMonsterAction{Damage Threshold (5)}
    The psionic golem has immunity to all damage unless it takes
    an amount of damage from a single attack or effect equal to
    or greater than its damage threshold,
    in which case it takes damage as normal.
    
    % Actions
    \DndMonsterSection{Actions}
    \DndMonsterAction{Multiattack}
    The psionic golem makes two slam attacks.
  
    \DndMonsterMelee[
      name=Slam,
      mod=+7,
      reach=5,
      targets=one target,
      dmg=\DndDice{3d6+4},
      dmg-type=bludegoning,
    ]
\end{multicols}  
\end{DndMonster}

\begin{DndMonster}[float*=b,width=\textwidth + 8pt]{6th-Level Psionic Golem}
\begin{multicols}{2}
    \DndMonsterType{Large construct, unaligned}
  
    \DndMonsterBasics[
        armor-class = {19 (natural armour)},
        hit-points  = {\DndDice{9d10 + 12}},
        speed       = {40 ft.},
      ]
  
    \DndMonsterAbilityScores[
        str = 18,
        dex = 15,
        con = 19,
        int = 1,
        wis = 3,
        cha = 1,
      ]
  
    \DndMonsterDetails[
        damage-immunities = {poison},
        condition-immunities = {blinded, charmed, deafened, exhaustion,
                                frightened, paralyzed, petrified, poisoned},
        senses = {darkvision 30 ft., passive Perception 6},
        languages = {---},
        challenge = 6,
        proficiency = +3
      ]
    % Traits
    \DndMonsterAction{Golem}
    The psionic golem cannot be put to sleep, is immune to disease
    and cannot be raised by necromancy or similar spells.
    The golem also cannot be healed by magic or other means,
    but can be repaired by manifested powers.

    \DndMonsterAction{MP Potential}
    The psionic golem has 2 MP, which may be spent on abilities
    in Menus you have access to.

    \DndMonsterAction{Damage Threshold (8)}
    The psionic golem has immunity to all damage unless it takes
    an amount of damage from a single attack or effect equal to
    or greater than its damage threshold,
    in which case it takes damage as normal.
    
    % Actions
    \DndMonsterSection{Actions}
    \DndMonsterAction{Multiattack}
    The psionic golem makes two slam attacks.
  
    \DndMonsterMelee[
      name=Slam,
      mod=+7,
      reach=5,
      targets=one target,
      dmg=\DndDice{4d6+4},
      dmg-type=bludegoning,
    ]
\end{multicols}  
\end{DndMonster}

\begin{DndMonster}[float*=b,width=\textwidth + 8pt]{7th-Level Psionic Golem}
\begin{multicols}{2}
    \DndMonsterType{Large construct, unaligned}
  
    \DndMonsterBasics[
        armor-class = {19 (natural armour)},
        hit-points  = {\DndDice{11d10 + 12}},
        speed       = {40 ft.},
      ]
  
    \DndMonsterAbilityScores[
        str = 19,
        dex = 15,
        con = 19,
        int = 1,
        wis = 3,
        cha = 1,
      ]
  
    \DndMonsterDetails[
        damage-immunities = {poison},
        condition-immunities = {blinded, charmed, deafened, exhaustion,
                                frightened, paralyzed, petrified, poisoned},
        senses = {darkvision 30 ft., passive Perception 6},
        languages = {---},
        challenge = 7,
        proficiency = +3
      ]
    % Traits
    \DndMonsterAction{Golem}
    The psionic golem cannot be put to sleep, is immune to disease
    and cannot be raised by necromancy or similar spells.
    The golem also cannot be healed by magic or other means,
    but can be repaired by manifested powers.

    \DndMonsterAction{MP Potential}
    The psionic golem has 4 MP, which may be spent on abilities
    in Menus you have access to.

    \DndMonsterAction{Damage Threshold (12)}
    The psionic golem has immunity to all damage unless it takes
    an amount of damage from a single attack or effect equal to
    or greater than its damage threshold,
    in which case it takes damage as normal.
    
    % Actions
    \DndMonsterSection{Actions}
    \DndMonsterAction{Multiattack}
    The psionic golem makes two slam attacks.
  
    \DndMonsterMelee[
      name=Slam,
      mod=+7,
      reach=5,
      targets=one target,
      dmg=\DndDice{4d6+4},
      dmg-type=bludegoning,
    ]
\end{multicols}  
\end{DndMonster}

\begin{DndMonster}[float*=b,width=\textwidth + 8pt]{8th-Level Psionic Golem}
\begin{multicols}{2}
    \DndMonsterType{Large construct, unaligned}
  
    \DndMonsterBasics[
        armor-class = {19 (natural armour)},
        hit-points  = {\DndDice{15d10 + 12}},
        speed       = {40 ft.},
      ]
  
    \DndMonsterAbilityScores[
        str = 20,
        dex = 15,
        con = 20,
        int = 1,
        wis = 3,
        cha = 1,
      ]
  
    \DndMonsterDetails[
        damage-immunities = {poison},
        condition-immunities = {blinded, charmed, deafened, exhaustion,
                                frightened, paralyzed, petrified, poisoned},
        senses = {darkvision 30 ft., passive Perception 6},
        languages = {---},
        challenge = 8,
        proficiency = +3
      ]
    % Traits
    \DndMonsterAction{Golem}
    The psionic golem cannot be put to sleep, is immune to disease
    and cannot be raised by necromancy or similar spells.
    The golem also cannot be healed by magic or other means,
    but can be repaired by manifested powers.

    \DndMonsterAction{MP Potential}
    The psionic golem has 4 MP, which may be spent on abilities
    in Menus you have access to.

    \DndMonsterAction{Damage Threshold (15)}
    The psionic golem has immunity to all damage unless it takes
    an amount of damage from a single attack or effect equal to
    or greater than its damage threshold,
    in which case it takes damage as normal.
    
    % Actions
    \DndMonsterSection{Actions}
    \DndMonsterAction{Multiattack}
    The psionic golem makes two slam attacks.
  
    \DndMonsterMelee[
      name=Slam,
      mod=+8,
      reach=5,
      targets=one target,
      dmg=\DndDice{4d6+5},
      dmg-type=bludegoning,
    ]
\end{multicols}  
\end{DndMonster}

\begin{DndMonster}[float*=b,width=\textwidth + 8pt]{9th-Level Psionic Golem}
\begin{multicols}{2}
    \DndMonsterType{Huge construct, unaligned}
  
    \DndMonsterBasics[
        armor-class = {20 (natural armour)},
        hit-points  = {\DndDice{18d10 + 12}},
        speed       = {40 ft.},
      ]
  
    \DndMonsterAbilityScores[
        str = 20,
        dex = 15,
        con = 20,
        int = 1,
        wis = 3,
        cha = 1,
      ]
  
    \DndMonsterDetails[
        damage-immunities = {poison},
        condition-immunities = {blinded, charmed, deafened, exhaustion,
                                frightened, paralyzed, petrified, poisoned},
        senses = {darkvision 30 ft., passive Perception 6},
        languages = {---},
        challenge = 9,
        proficiency = +4
      ]
    % Traits
    \DndMonsterAction{Golem}
    The psionic golem cannot be put to sleep, is immune to disease
    and cannot be raised by necromancy or similar spells.
    The golem also cannot be healed by magic or other means,
    but can be repaired by manifested powers.

    \DndMonsterAction{MP Potential}
    The psionic golem has 8 MP, which may be spent on abilities
    in Menus you have access to.

    \DndMonsterAction{Damage Threshold (20)}
    The psionic golem has immunity to all damage unless it takes
    an amount of damage from a single attack or effect equal to
    or greater than its damage threshold,
    in which case it takes damage as normal.
    
    % Actions
    \DndMonsterSection{Actions}
    \DndMonsterAction{Multiattack}
    The psionic golem makes two slam attacks.
  
    \DndMonsterMelee[
      name=Slam,
      mod=+9,
      reach=5,
      targets=one target,
      dmg=\DndDice{4d6+5},
      dmg-type=bludegoning,
    ]
\end{multicols}
\end{DndMonster}

\end{document}