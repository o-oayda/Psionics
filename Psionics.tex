\documentclass[
    10pt,
    twoside,
    twocolumn,
    openany,
    nodeprecatedcode,
    bg=full,
    justified%
    ]{dndbook}

\usepackage[english]{babel}
\usepackage[utf8]{inputenc}
\usepackage{hyperref}
\usepackage{rotating}
\newcommand{\secref}[1]{\ref{#1}. \nameref{#1}}
\newcommand{\define}[1]{\textbf{#1}}
\newcommand{\augment}[1]{\newline\indent\textbf{Augment.} #1}
\newcommand{\upcast}[1]{\newline\indent\textbf{At Higher Levels.} #1}
\newcommand{\spell}[1]{\textit{#1}}
\newcommand{\power}[1]{\textit{#1}}
\usepackage{lipsum}
\usepackage{adjustbox}
\usepackage{rotating}
\usepackage{amsmath}
\usepackage{epigraph}
\usetikzlibrary{intersections}
\renewcommand{\textflush}{flushepinormal}

\hypersetup{%
	colorlinks = true,%
	linkcolor =blue,%
	anchorcolor = red,%
	citecolor = blue,%
	urlcolor = blue%
}

\title{\Huge\scshape\fontsize{50}{40}\selectfont
    The Old Way \\
    \medskip
    \normalfont\scshape
    \Huge
    A Psionics Supplement\\
    \normalfont\Large\vspace{5mm}\texttt{Version 1.3.2}}
\author{DM Oayda}
% \usepackage[capitalise,nameinlink,noabbrev]{cleveref}
\setcounter{tocdepth}{2}
\setcounter{secnumdepth}{2}

\cftsetindents{subsection}{1em}{14mm} % toc spacing

% power costs variables
\newcommand\lvlone{1}
\newcommand\lvltwo{3}
\newcommand\lvlthree{5}
\newcommand\lvlfour{7}
\newcommand\lvlfive{9}
\newcommand\lvlsix{11}
\newcommand\lvlseven{13}
\newcommand\lvleight{15}
\newcommand\lvlnine{17}

\newcommand\pklvlnine{11}

\begin{document}

\frontmatter
\maketitle
\tableofcontents

\mainmatter%

\chapter{Introduction}


\chapter{Using Psionics}

\DndDropCapLine{T}{his chapter explains} the nature of psionics,
as well as the key rules behind
manifesting and maintaining psionic powers.

\section{What are Psionics?}
Psionics tap into the latent ability of the mind
to master itself, its body and its environment.
Whereas spellcasters manipulate the magical essence
innate to the cosmological order,
users of psionics rely on their mind's power alone.
For many creatures,
manifesting any psionic power,
let alone mastering a psionic discipline,
takes years, even decades of rigorous mental training.
That being said,
some creatures have a natural propensity for psionics,
for example the children of Gith, illithids and aboleths.

Many refer to psionics as \emph{the Old Way}.
This is because there is a school of thought that
the seeds of all creatures---the progenitor beings
which gave rise to existence itself---were
masters of the mind.
According to this doctrine,
magic and spellcraft were created by these
beings when the reality was moulded to as it is today.
To do this, they used psionics.

\section{Types of Psionic Abilities}
There are two types of psionic abilities.
The main type are \textbf{powers},
which are what most psionics users will be
interacting with the most.
The rules around powers are described in this chapter.

A second type is an \textbf{innate talent}.

\subsection{Interaction with Magic}
Since psionics and magic are disparate entities,
how do they interact together?
A creature that manifests a psionic power is not
casting a spell,
and therefore any ability which negates the effects
of magic cannot negate the power.
For example, psionic powers cannot be affected by
\spell{counterspell} or \spell{dispel magic}.

However, certain psionic powers are explicitly stated
to interact with magic.
In addition, this supplement adds spells to the existing
grimoire which explicitly interact with psioics.

\section{Manifesting Powers}
In order to manifest a psionic power,
the \define{power level} must be equal to or below
the manifesting character's \define{psionic threshold}.
The psionic threshold is determined by the number of levels
a character has in classes with psionics.
If the power satisfies this test,
it can be manifested if they have sufficient \define{psi dice}
to expend.
The manifesting character will also need to note their
current \define{mental bandwidth} and whether or not manifesting
the power will exceed this bandwidth.
Each of these concepts are explained below.

\subsection{Psionic Threshold}
The psionic threshold represents the strongest powers
that a psionics user can learn and manifest.
The psionic threshold is determined by a character's class
and level,
as shown in the class tables of \secref{chap:classes}.
When levelling,
a character can only learn powers with level up
their psionic threshold.
This is explained in more detail in the same chapter.

Each power has a level written in its description.
If a character attempts to manifest a power with level
greater than their psionic threshold,
the power fails to manifest,
however no cost is incurred in doing so,
see \secref{sub:psi_dice}.

\subsection{Mental Bandwidth}
Manifesting powers strains the mind and the body.
Mental bandwidth represents the amount of psionic strain that
a character can endure while using psionics,
and is determined by the character's psionic class and level.
Mental bandwidth is divided into slots,
and each power takes up a different number of slots.

When a character manifests a power,
they note their total mental bandwidth
as determined by their class and level.
This is called the
\define{maximum mental bandwidth}.
They then subtract from the maximum mental bandwidth
any slots which are taken up by powers they are already concentrating on.
The number determined after this subtraction is called the
\define{effective mental bandwidth}.
If the chosen power has a mental bandwidth
greater than the effective mental bandwidth,
the character must make a Constitution saving throw
with DC equal to 10
plus twice the number of slots in excess of their effective bandwidth.

For example,
suppose Kadoth is a 10th level psion
and therefore has a maximum mental bandwidth of 6.
They are already concentrating on adapt body,
which takes up 4 mental bandwidth slots,
so their effective mental bandwidth is 2.
They then attempt to manifest the mind slam power,
which takes up 3 slots.
Since this puts them in excess of their effective bandwidth by 1,
they make a DC 12 Constitution saving throw ($10 + 2 \times 1)$.

If a character fails this constitution saving throw,
the power fails to manifest.
In addition, the manifester gains the \textbf{strained} condition.
The severity of the condition depends on the number of slots
that were in excess of the available bandwidth when the attempt
to manifest the power was made.
This is shown in the table below.
Note that the maluses are cumulative.
For example,
with two slots in excess,
a character incurs the malus from being in excess by 1
as well as the malus for being in excess by 2.

\begin{table}[htbp]%
    \begin{DndTable}[width=\columnwidth,
                     header=Strained Condition]{
                     X X}
        Slots in Excess & Cumulative Maluses \\
        1 &  \\
        2 & Gain level of exhaustion \\
        3 & \\
        4 & Gain level of exhaustion \\
        5+ & 
    \end{DndTable}
\end{table}

\subsection{Psi Dice}
\label{sub:psi_dice}
When manifesting a power,
a character must expend a number of psi dice
equal to the number stated in the power's description.
The number of psi dice a character has is determined
by their class and level.
If a character has less psi dice than the total psi dice cost
of the power, the power fails (but no dice are expended).

The total psi dice cost of a power is determined by the its level.
The cost is equal to one less than double the power's level.
This is shown in the table below.
For example, a 1st level power incurs a cost of 1 psi die,
whereas a 5-th level power incurs a cost of 9 psi dice.
\begin{table*}[htbp]%
    \begin{DndTable}[width=\textwidth,
                     header=Psi Dice Cost by Level]{
                     X X X X X X X X X X}
         Level         & 1 & 2 & 3 & 4 & 5 & 6  & 7  & 8  & 9 \\
        \textbf{Cost}  & 1 & 3 & 5 & 7 & 9 & 11 & 13 & 15 & 17
    \end{DndTable}
\end{table*}

\subsection{Concentration}
If stipulated in the power description,
a power requires uninterrupted concentration
in order to be continually used.
There is no limit to the number of powers that may be
simultaneously concentrated on,
however each power concentrated on will take away from the
available mental bandwidth slots, as mentioned above.

Whenever a manifester takes damage while
concentrating on any number of powers,
they must make a Constitution saving throw.
The DC of the save is either equal to 10 or half the damage taken,
whichever is higher.
Each separate source of damage prompts a separate save.
If the save fails,
the manifester loses concentration on all the powers they were
concentrating on.
In addition, the DM may call the manifester to make a
save with any DC in a circumstance that would
jeopardise the manifester's concentration,
like being thrown of a flying broom or
being battered by an intense blizzard.

A manifester may at any time stop concentrating on any power.
However, a manifester will automatically stop concentrating on a power
if they become incapacitated or if they die.

\section{Power Attributes}
In \secref{sec:list_of_powers},
the available powers are given.
In this section,
the terminology and structure of the power descriptions are explained.

\subsection{Level and Discipline}
Beneath the name of the power,
the power level and discipline are given.
Each power is associated with a discipline,
of which there are six.
These are described below.

\subsubsection{Prescience}
Strictly, the discipline of prescience relates to psionic powers
which allow the manifester to peer through time and see future events.
Prescient manifesters describe the future as
an undulating cloth blowing in a swift breeze,
with the hills and valleys of its surface representing
future possibilities. 

However, in more recent times,
the discipline of prescience has incorporated additional tenets,
and its adherents are also characterised by an uncanny ability to
understand the past and know things not normally known to
the untrained mind.

\subsubsection{Prana Bindu}
The discipline of Prana Bindu stresses that the mind and body
must work in close concert with each other.
Adherents have a profound mastery of nerve and muscle,
and can react to situations with speeds nearly imperceptible
to the untrained.
The discipline also allows one to alter their internal equilibrium
and metabolism at will,
granting the body the ability to heal and adapt to hostile environments.

\subsubsection{Voice}
The Voice is a dangerous but supremely powerful tool
used to dominate the minds and wills of others.
Adherents hone their speech in such a way that,
only by subtle changes in pitch, volume and speed,
their Voice can control the behaviour of others.
In addition, adherents to the discipline can telepathically
communicate with other creatures.

\subsubsection{Metacreativity}
The discipline of metacreativity allows the manifester to mould
matter as they desire,
turning the mundane into useful tools or deadly weapons.
A metacreator can also in this manner craft constructs
which bend to their mind,
allowing myriad servants to carry out their bidding.

\subsubsection{Spacefolding}
A spacefolder warps the fabric of space and time to
transport themselves and others.
Many powerful spacefolders can cross the gap between
the planes themselves.

\subsubsection{Psychokinesis}
The discipline of psychokinesis allows one to transform
their raw mental energy into destructive power,
unleashing havoc on the battlefield.
Psychokineticists can deal terrifying amounts of damage
and leave their enemies trembling in awe and fear.

\subsection{Manifesting Time}
Each power has a \define{manifesting time},
which may be a bonus action, action, or reaction.
Unlike spellcasting,
there is no level restriction to manifesting powers
on the same turn where one has been manifested
with a bonus action. 

In addition, some powers may have longer casting times,
in which case on each round the manifester must use their action
to manifest the power while maintaining concentration.
While the manifester is maintaining concentration in this way,
the power will take up the specified number
of mental bandwidth slots given in the power's description.
If their concentration is broken, the power fails,
but no psi dice are expended,
and the mental bandwidth slots are freed up.

\subsection{Range}
A power's \define{range} indicates the extent
to which its effects can reach when manifested.
The \define{target} of the power must be within range.

There are three types of ranges: self, touch and a range
specified in feet.
Powers with a range of self will always target the manifester.

\subsubsection{Targets}
When selecting a target for a power,
in addition to the range requirement mentioned above,
the manifester must have \define{line of sight}
towards the target.
The meaning line of sight has two prongs:
\define{sight} and \define{uninterruption}.
Respectively, these mean that the manifester must be able to
(a) see the target
and (b) draw a line to the target
without that line passing through an obstacle,
even if the target can be seen through that obstacle.

Powers with a range of self (powers that target the manifester)
are not bound by this requirement.
In addition, a power may mention that the manifester does not need
line of sight towards the target,
or that the manifester does not need either of the
two prongs of line of sight. 

\subsection{Cost}
The \define{cost} specifies the mental bandwidth and psi die
that must be expended to manifest the power.
For example,
`MB 1, PD 1'
means that the power takes up 1 slot of mental bandwidth
and costs one psi die to manifest.

\subsection{Duration}
The \define{duration} of a power is how long the effects
of a power remain active.
Powers can be instantaneous,
while others can last for any length of time
as specified in the powers description.

If the power requires concentration for throughout its duration,
this will be indicated alongside the duration.

\subsection{Areas of Effect}
Some powers specify in their description an area of effect
over which the effects manifest.
In this case, the rules governing the area of effect
are identical to those of spellcasting beginning at
phb 202.

\subsection{Affecting Targets}
Some powers specify that the manifester makes an attack
roll against each target,
whereas some specify that all targets need to make
a type of saving throw.

If the power specifies an attack roll,
then the manifester rolls a d20
and adds their \define{psionic attack modifier},
which is their \define{psionic ability modifier} (determined by class)
plus their proficiency bonus.

If the power specifies a saving throw,
then each target must succeed on a saving throw
with DC equal to 8 + the manifester's psionic attack modifier.
The type of saving throw
is specified in the power's description.

\subsection{Augmenting Powers}
\label{sub:augmenting}
Some powers will indicate that they can be \define{augmented},
which is analogous to upcasting for a spellcaster.
When augmenting a power,
the manifester spends additional psi dice on top of the cost
incurred from the level of the power itself.
To augment a power to one additional level,
the manifester spends two psi dice,
since for every single increase in level,
the psi dice cost increases by two,
see \secref{sub:psi_dice}. 

A power cannot be augmented to a level beyond the manifester's
psionic threshold.
For example,
if a manifester has a psionic threshold of 3,
they may spend four psi dice to augment a power from
1st to 3rd level,
but may not go beyond 3rd level.

\section{List of Powers}
\label{sec:list_of_powers}
\DndPowerHeader%
  {Demoralise}
  {1st-level Voice}
  {1 action}
  {30 feet}
  {MB 1, PD 1}
  {Concentration, up to 1 minute}
Choose 3 hostile creatures that you can see within range.
Each creature must make a Wisdom saving throw.
If they fail, they target takes a 1d4 penalty to attack rolls and
saving throws until the power ends.
\higher{You may target one additional creature for
        each additional level.}

\chapter{Classes}
\label{chap:classes}

\section{Psion}
\begin{figure*}[t]
    \begin{ornamentedtabular}{c c l c c c c}[title={The Psion}]
        \textbf{Level} & \textbf{P. Bonus} & \textbf{Features} & \textbf{Psi Dice} & \textbf{Mental Bandwidth} & \textbf{Powers Known} & \textbf{Psionic Threshold} \\
        1st  & +2 & Powers, Discipline        & 2   & 2  & 3  & 1 \\
        2nd  & +2 & Microsleep (one use)      & 5   & 2  & 5  & 1 \\
        3rd  & +2 &                           & 10  & 3  & 7  & 2 \\
        4th  & +2 & Ability Score Improvement & 13  & 3  & 9  & 2 \\
        5th  & +3 &                           & 23  & 4  & 11 & 3 \\
        6th  & +3 &                           & 28  & 4  & 13 & 3 \\
        7th  & +3 &                           & 35  & 4  & 15 & 4 \\
        8th  & +3 & Ability Score Improvement & 42  & 5  & 17 & 4 \\
        9th  & +4 &                           & 58  & 5  & 19 & 5 \\
        10th & +4 &                           & 67  & 6  & 21 & 5 \\
        11th & +4 & Microsleep (two uses)     & 78  & 6  & 22 & 6 \\
        12th & +4 & Ability Score Improvement & 78  & 6  & 24 & 6 \\
        13th & +5 &                           & 91  & 7  & 25 & 7 \\
        14th & +5 &                           & 91  & 7  & 27 & 7 \\
        15h  & +5 &                           & 106 & 7  & 28 & 8 \\
        16th & +5 & Ability Score Improvement & 106 & 8  & 30 & 9 \\
        17th & +6 &                           & 123 & 8  & 31 & 9 \\
        18th & +6 &                           & 132 & 8  & 33 & 9 \\
        19th & +6 & Ability Score Improvement & 143 & 9  & 34 & 9 \\
        20th & +6 &                           & 156 & 10 & 36 & 9 \\
    \end{ornamentedtabular}
\end{figure*}

\subsection{Class Features}
As a psion, you gain the following features.

\subsubsection{Hit Points}

\listparagraph[4pt]{Hit Dice}{
    1d6 per psion level
}
\listparagraph{Hit Points at 1st Level}{
    6 + your Constitution modifier
}
\listparagraph{Hit Points at Higher Levels}{
    1d6 (or 4) + yuour Constitution modifier per psion level
    after 1st
}

\subsubsection{Proficiencies}

\listparagraph[4pt]{Armour}{
    Light armour
}
\listparagraph{Weapons}{
    Clubs, daggers, heavy crossbows, light crossbows,
    quarterstaffs, spears
}
\listparagraph{Tools}{
    None
}
\vspace{6pt}
\listparagraph{Saving Throws}{
    Intelligence, Wisdom
}
\listparagraph{Skills}{
    Choose two from History, Religion, Nature,
    Insight and Perception
}

In addition, depending on the psion's chosen discipline,
they acquire an additional skill proficiency.
This is shown in the table below.
\begin{table}[htbp]%
    \begin{DndTable}[width=\columnwidth,
                     header=Additional Profciency]{
                     X X}
        Discipline & Proficiency \\
        Prescience & Investigation \\
        Prana Bindu & Athletics, Acrobatics or Stealth \\
        Voice & Persuasion \\
        Metacreativity & Survival \\
        Spacefolding & Arcana \\
        Psychokinesis & Intimidation
    \end{DndTable}
\end{table}

\subsubsection{Equipment}
You start with the following equipment,
in addition to the equipment granted by your background:
\begin{itemize}
    \item (a) a club or (b) a dagger
    \item a light crossbow and 20 bolts
    \item (a) a scholar's pack or (b) a dungeoneer's pack
    \item Leather armour
\end{itemize}

\subsection{Powers}
\subsubsection{Psi Dice}
The number of powers that you can manifest
is limited by the number of psi dice that
you have available.
This is the principal limit on the output of a psion.
For example, a 9-th level psion with 72 psi dice
can manifest a power costing 1 psi dice 72 times.

The number of available psi dice per level
is shown in the psion class table.
You regain all expended psi dice when you finish
a long rest.

\subsubsection{Mental Bandwidth}
The powers you can manifest without straining yourself
is reflected by your mental bandwidth.
The total number of mental bandwidth slots you have
is shown in the class table.

\subsubsection{Discipline}
At 1st level,
you must decide upon a discipline.
Your discipline grants you a specific sublist of powers
which cannot be accessed by other disciplines.

\subsubsection{Powers Known}
At 1st level,
you choose three psion powers of your choice
from either the psion class power list
or your chosen discipline sublist.
You cannot choose powers from discipline sublists
other than that of the discipline you have nominated.

\subparagraph{Psionic Threshold}
You cannot learn nor manifest powers with a level
greater than your psionic threshold,
shown in the class table.
As mentioned in \secref{sub:augmenting},
you also cannot augment powers to a level
beyond your psionic threshold.

\subsubsection{Manifesting Ability}
Your psionic ability modifier is your Intelligence modifier.
The save DC against your psionic powers and your
psionic attack modifier respectively are therefore respectively:
\small\begin{equation*}
    \begin{gathered}
        \text{\textbf{Psionics save DC}}
            = 8 + \text{your proficiency bonus} + \\
                  \text{your Intelligence modifier} \\
        \text{\textbf{Psionics attack modifier}}
            = \text{your proficiency bonus} + \\
              \text{your Intelligence modifier}
    \end{gathered}
\end{equation*}\normalsize

\subsection{Microsleep}
Starting at 2nd level,
as a bonus action on your turn,
you may recover a number of psi dice equal to
your Intelligence modifier multiplied by your psionc threshold.

Once you use this feature,
you must finish a long rest before you can use it again.
Starting at 11th level,
you can use it twice before a long rest.

\subsection{Ability Score Improvement}
When you reach 4th level,
and again at 8th, 12th, 16th and 19th level,
you can increase one ability score of your choice by 2,
or you can increase two of your ability scores by 1.
As normal,
you can't increase an ability score above 20 using this feature.

\newpage\section{Psi Knight}

\end{document}