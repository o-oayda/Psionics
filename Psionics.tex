\documentclass[
    10pt,
    twoside,
    twocolumn,
    openany,
    nodeprecatedcode,
    bg=full,
    justified%
    ]{dndbook}

\usepackage[english]{babel}
\usepackage[utf8]{inputenc}
\usepackage{hyperref}
\usepackage{rotating}
\newcommand{\secref}[1]{\ref{#1}. \nameref{#1}}
\newcommand{\define}[1]{\textbf{#1}}
\newcommand{\augment}[1]{\newline\indent\textbf{Augment.} #1}
\newcommand{\upcast}[1]{\newline\indent\textbf{At Higher Levels.} #1}
\newcommand{\spell}[1]{\textit{#1}}
\newcommand{\power}[1]{\textit{#1}}
\usepackage{lipsum}
\usepackage{adjustbox}
\usepackage{rotating}
\usepackage{amsmath}
\usepackage{epigraph}
\usetikzlibrary{intersections}
\renewcommand{\textflush}{flushepinormal}

\hypersetup{%
	colorlinks = true,%
	linkcolor =blue,%
	anchorcolor = red,%
	citecolor = blue,%
	urlcolor = blue%
}

\title{\Huge\scshape\fontsize{50}{40}\selectfont
    The Old Way \\
    \medskip
    \normalfont\scshape
    \Huge
    A Psionics Supplement\\
    \normalfont\Large\vspace{5mm}\texttt{Version 1.3.2}}
\author{DM Oayda}
% \usepackage[capitalise,nameinlink,noabbrev]{cleveref}
\setcounter{tocdepth}{2}
\setcounter{secnumdepth}{2}

\cftsetindents{subsection}{1em}{14mm} % toc spacing

% power costs variables
\newcommand\lvlone{1}
\newcommand\lvltwo{3}
\newcommand\lvlthree{5}
\newcommand\lvlfour{7}
\newcommand\lvlfive{9}
\newcommand\lvlsix{11}
\newcommand\lvlseven{13}
\newcommand\lvleight{15}
\newcommand\lvlnine{17}

\newcommand\pklvlnine{11}

\begin{document}

\frontmatter
\maketitle
\tableofcontents

\mainmatter%

\chapter{Introduction}

\chapter{Using Psionics}
This chapter explains the key rules behind manifesting and
maintaining psionic powers.

\section{Manifesting Powers}
In order to manifest a psionic power,
the power must be equal to or below the manifesting character's
\define{psionic threshold}.
The psionic threshold is determined by the character's class level
in a psionic class.

If the power satisfies this test,
it can be manifested if they have sufficient \define{psi dice}
to expend.
The manifesting character will also need to note their
current \define{mental bandwidth} and whether or not manifesting
the power will exceed this bandwidth.
These are both explained below.

\subsection{Mental Bandwidth}
Manifesting powers strains the mind and the body.
Mental bandwidth represents the amount of psionic strain that
a character can endure while using psionics,
and is determined by the character's psionic class and level.
Mental bandwidth is divided into slots,
and each power takes up a different number of slots.
When a character manifests a power,
they note their current mental bandwidth and subtract from that
any slots which are taken up by powers they are already concentrating on.
If the chosen power has a mental bandwidth
greater than the number of slots remaining,
the character must make a Constitution saving throw
with DC equal to 10
plus twice the number of slots in excess of their remaining bandwidth.

For example,
suppose Kadoth is a 10th level psion
and therefore has a mental bandwidth of 6.
They are already concentrating on adapt body,
which takes up 4 mental bandwidth slots.
They then attempt to manifest the mind slam power,
which takes up 3 slots.
Since this puts them in excess of their bandwidth maximum by 1,
they make a DC 12 Constitution saving throw ($10 + 2 \times 1)$.

If a character fails this constitution saving throw,
the power fails to manifest.
In addition, the manifester gains the \textbf{strained} condition.
The severity of the condition depends on the number of slots
that were in excess of the available bandwidth when the attempt
to manifest the power was made.
This is shown in the table below.
Note that the maluses are cumulative.
For example,
with two slots in excess,
a character incurs the malus from being in excess by 1
as well as the malus for being in excess by 2.

\begin{table}[htbp]%
    \begin{DndTable}[width=\columnwidth,
                     header=Strained Condition]{
                     X X}
        Slots in Excess & Cumulative Maluses \\
        1 &  \\
        2 & Gain level of exhaustion \\
        3 & \\
        4 & Gain level of exhaustion \\
        5+ & 
    \end{DndTable}
\end{table}

\subsection{Psi Die}
When manifesting a power,
a character must expend a number of psi dice
equal to the number stated in the power's description.
The number of psi dice a character has is determined
by their class and level.
If a character has less psi dice than the total psi dice cost
of the power, the power fails (but no dice are expended).

\subsection{Concentration}
If stipulated in the power description,
a power requires uninterrupted concentration
in order to be continually used.
There is no limit to the number of powers that may be
simultaneously concentrated on,
however each power concentrated on will take away from the
available mental bandwidth slots, as mentioned above.

Whenever a manifester takes damage while
concentrating on any number of powers,
they must make a Constitution saving throw.
The DC of the save is either equal to 10 or half the damage taken,
whichever is higher.
Each separate source of damage prompts an separate save.
If the save fails,
the manifester loses concentration on all the powers they were
concentrating on.

A manifester may at any time stop concentrating on any power.
However, a manifester will automatically stop concentrating on a power
if they become incapacitated or if they die.

\section{List of Powers}

\DndPowerHeader%
  {Demoralise}
  {1st-level telepathy}
  {1 action}
  {30 feet}
  {MB 1, PD 1}
  {Concentration, up to 1 minute}
Choose 3 hostile creatures that you can see within range.
Each creature must make a Wisdom saving throw.
If they fail, they target takes a 1d4 penalty to attack rolls and
saving throws until the power ends.
\higher{You may target one additional creature for
        each additional level.}

\chapter{Classes}

\section{Psion}


\end{document}