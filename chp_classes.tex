\chapter{Classes}
\label{chap:classes}
\DndDropCapLine{H}{ere, two psionics users are}
presented: the psion at \secref{sec:psion} and the psi knight at
\secref{sec:psi_knight}.
The psion is a full user of psionics,
focusing nearly exclusively on their mental prowess
to manifest devastatingly effective powers.
They have access to the largest number of powers,
and can hone their talent further by specialising
in a particular discipline.
The psi knight is a martial warrior
who supplements their fighting ability with
psionics.
This makes them a highly versatile and formidable
opponent on the battlefield.

\section{Psion}
\label{sec:psion}
\DndSetThemeColor[DmgLavender]
In the audience chamber of the Duke of Baradstone,
a half-elf pries into the pathways of Duke's mind---their
thoughts, feelings, desires revealed to her
with lucid precision.
All the while, the Duke's retinue remain totally oblivious
to the psionic intrusion.

In an vanishing instant of time,
the githzerai projects the movements of his
githyanki foes two seconds into the future,
giving himself the edge he needs to lure them
into a carefully-placed trap.

The tiefling crouches imperceptibly behind the garrison,
recalling her Prana Bindu training regimen to force
her body to make no sound.
Then, she focuses on the sword carried by the guardsman;
before he takes his next step,
the steel burns a brilliant red before exploding in his
very hand.

In the underbelly of the city,
the dwarf brushes his hand against the ancient stone.
Its feeling, its texture, ignite a subconscious
awareness within his mind,
and he is borne back to the memory of countless aeons
past.
He watches in striking detail as the genetic memory unravels,
showing his forebears laying the same stone---one piece
of a much larger city which has been lost to the entropic
assault of time.

\subsection{Masters of the Mind}
While psions are diverse in their abilities and training,
all are unified by a common thread---their unparalleled
mastery over the latent powers that lie within their mind.
While some turn to the magic essence which suffuses the cosmos,
psions peer inwards,
unlocking powers which might have laid dormant over many years.
Only through strict training and profound mental discipline
can these powers be unlocked,
which all psions have been through in one way or another.

Psions manifest distinct effects known as powers, which,
depending on their discipline, achieve myriad effects.
Powers are integral to the effectiveness of a psion,
and the more experience they accrue,
the more powers they unlock from deep within.

\subsection{Creating a Psion}
Creating a character with the psion class prompts
important questions.
How did your character first become aware of the
powers of the mind?
Perhaps they found an old, forgotten tome
describing the \emph{Old Way} and the monks who adhered
to it.
Or they might have been awoken to it abruptly,
awakening a dormant power which manifested fleetingly.

One additional line of inquiry should be about the nature
of their training.
How did your character hone their psionic abilities?
Did they train with the githzerai in their floating citadels
on Limbo?
Did they live with a reclusive order who have knowledge
of psionic powers?
Or did they train by themselves,
experimenting with their mind through trial and error?

\subsubsection{Quick Build}
You can make a psion quickly by following these suggestions.
First, Intelligence should be your highest ability score,
followed by Constitution.
Second, choose the following three 1st-level psion powers:
\nameref{pwr:blow-up-object-with-mind},
\nameref{pwr:mind-thrust}
and \nameref{pwr:precognition}.
If you have access to a 1st level power on your discipline
sublist (see \secref{subs:psion_discipline}),
choose that over one of the powers given above.

\begin{DndSidebar}[float=htbp]{Multiclassing with the Psion}
    \tocside{Multiclassing with the Psion}In order to
    multiclass into the psion class,
    a character must have a minimum Intelligence score of 13
    (in addition to the usual score requirement for their base class).
    When you multiclass into the psion class,
    you gain proficiency with
    light armour, quarterstaffs and daggers.
\end{DndSidebar}

\begin{figure*}[t]
    \begin{ornamentedtabular}{c c p{4.3cm} c c c c}[title={The Psion}]
        \textbf{Level} & \textbf{P. Bonus} & \textbf{Features} & \textbf{Psi Dice} & \textbf{Mental Bandwidth} & \textbf{Powers Known} & \textbf{Psionic Threshold} \\
        1st  & +2 & Powers, Discipline        & 1   & 2  & 3  & 1 \\
        2nd  & +2 & Microsleep (one use)      & 3   & 2  & 5  & 1 \\
        3rd  & +2 & ---                       & 6  & 3  & 7  & 2 \\
        4th  & +2 & Ability Score Improvement & 9  & 3  & 9  & 2 \\
        5th  & +3 & ---                       & 12  & 4  & 11 & 3 \\
        6th  & +3 & Discipline Tenet          & 15  & 4  & 13 & 3 \\
        7th  & +3 & Mind Palace               & 18  & 4  & 15 & 4 \\
        8th  & +3 & Ability Score Improvement & 21  & 5  & 17 & 4 \\
        9th  & +4 & ---                       & 24  & 5  & 19 & 5 \\
        10th & +4 & Discipline Tenet          & 28  & 6  & 21 & 5 \\
        11th & +4 & Microsleep (two uses)     & 32  & 6  & 22 & 6 \\
        12th & +4 & Ability Score Improvement & 36  & 6  & 24 & 6 \\
        13th & +5 & ---                       & 40  & 7  & 25 & 7 \\
        14th & +5 & Discipline Tenet          & 44  & 7  & 27 & 7 \\
        15h  & +5 & ---                       & 48 & 7  & 28 & 8 \\
        16th & +5 & Ability Score Improvement & 52 & 8  & 30 & 8 \\
        17th & +6 & ---                       & 57 & 8  & 31 & 9 \\
        18th & +6 & ---                       & 61 & 8  & 33 & 9 \\
        19th & +6 & Ability Score Improvement & 66 & 9  & 34 & 9 \\
        20th & +6 & Ascendance                & 71 & 10 & 36 & 9 \\
    \end{ornamentedtabular}
\end{figure*}

\subsection{Class Features}
As a psion, you gain the following features.

\subsubsection{Hit Points}

\begin{description}
    \item[Hit Dice:] 1d6 per psion level
    \item[Hit Points at 1st Level:] 6 + your Constitution modifier
    \item[Hit Points at Higher Levels:] 1d6 (or 4) +
        your Constitution modifier per psion level after 1st
\end{description}

\subsubsection{Proficiencies}

\begin{description}
    \item[Armour:] Light armour
    \item[Weapons:] Clubs, daggers, heavy crossbows, light crossbows,
        quarterstaffs, spears
    \item[Tools:] None \vspace{4pt}
    \item[Saving Throws:] Intelligence, Wisdom
    \item[Skills:] Choose two from History, Religion, Nature,
        Insight and Perception
\end{description}

In addition, depending on the psion's chosen discipline,
they acquire an additional skill proficiency.
This is shown in the table below.
\begin{table}[htbp]%
    \begin{DndTable}[width=\columnwidth,
                     header=Additional Proficiency]{
                     X X}
        Discipline & Proficiency \\
        Prescience & Investigation \\
        Prana Bindu & Athletics, Acrobatics or Stealth \\
        Voice & Persuasion \\
        Metacreativity & Survival \\
        Spacefolding & Arcana \\
        Psychokinesis & Intimidation
    \end{DndTable}
\end{table}

\subsubsection{Equipment}
You start with the following equipment,
in addition to the equipment granted by your background:
\begin{itemize}
    \item (a) a club or (b) a dagger
    \item a light crossbow and 20 bolts
    \item (a) a scholar's pack or (b) a dungeoneer's pack
    \item Leather armour
\end{itemize}

\subsection{Powers}
\subsubsection{Psi Dice}
The number of powers that you can manifest
is limited by the number of psi dice that
you have available.
This is the principal limit on the output of a psion.
For example, a 9-th level psion with 24 psi dice
can manifest a power costing 1 psi dice 24 times.

The number of available psi dice per level
is shown in the psion class table.
You regain all expended psi dice when you finish
a long rest.

\subsubsection{Mental Bandwidth}
The powers you can manifest without straining yourself
is reflected by your mental bandwidth.
The total number of mental bandwidth slots you have
is shown in the class table.

\subsubsection{Discipline}
\label{subs:psion_discipline}
At 1st level,
you must decide upon a discipline.
Your discipline grants you a specific sublist of powers
which cannot be accessed by other disciplines.
This is called the \define{discipline sublist}\index{discipline sublist}.
These are given at \secref{sub:discipline_sublists}.
In addition, your discipline grants you different abilities
when you reach certain levels.
These are called \define{tenets}\index{tenet},
and are given later on in this section.

\subsubsection{Powers Known}
At 1st level,
you choose three psion powers of your choice
from either the psion class power list
or your chosen discipline sublist.
You cannot choose powers from discipline sublists
other than that of the discipline you have nominated.
When you gain a psion level,
you can, subject to the same restrictions,
learn additional powers on top of the ones you already know.
The total numbers of powers you know cannot
exceed your powers known value,
as shown in the class table.

The powers you have selected in this fashion have been committed
to memory, and you need not prepare a selection of them each long rest.
Also,
whenever you take a long rest,
you may substitute any number of powers you already know
for ones you do not know
on the psion class list or your discipline sublist.
The power that has been replaced counts as no longer being known.

\subparagraph{Psionic Threshold}
You cannot learn nor manifest powers with a level
greater than your psionic threshold,
shown in the class table.
As mentioned in \secref{sub:augmenting},
you also cannot augment powers to a level
beyond your psionic threshold.

\subsubsection{Manifesting Ability}
Your psionic ability modifier is your Intelligence modifier.
The save DC against your psionic powers and your
psionic attack modifier are therefore respectively:
\small\begin{equation*}
    \begin{gathered}
        \text{\textbf{Psionics save DC}}
            = 8 + \text{your proficiency bonus} + \\
                  \text{your Intelligence modifier} \\
        \text{\textbf{Psionics attack modifier}}
            = \text{your proficiency bonus} + \\
              \text{your Intelligence modifier}
    \end{gathered}
\end{equation*}\normalsize

\subsection{Microsleep}
Starting at 2nd level,
as a bonus action on your turn,
you may recover a number of psi dice equal to
your Intelligence modifier multiplied by your psionic threshold.

Once you use this feature,
you must finish a long rest before you can use it again.
Starting at 11th level,
you can use it twice before a long rest.

\subsection{Ability Score Improvement}
When you reach 4th level,
and again at 8th, 12th, 16th and 19th level,
you can increase one ability score of your choice by 2,
or you can increase two of your ability scores by 1.
As normal,
you can't increase an ability score above 20 using this feature.

\subsection{Mind Palace}
Starting at 7th level,
your mind becomes your sanctuary---a palace constructed
from pure thought.
Whenever you are asleep,
you return to this sanctuary,
which exists in a perfect equilibrium between
external and internal influences.
You are fully alert to your surroundings
while you are asleep.
In addition, you can communicate telepathically to other
creatures who are also sleeping within 30 feet of you,
in which case you appear as a character in their dreams.

\subsection{Ascendance}
Starting at 20th level,
your body becomes only a vessel for the vast potential
of your mind.
Your mind alters your biological makeup to better resist damage,
granting you resistance to bludgeoning, piercing and slashing damage.
In addition,
you no longer age,
but you can still be killed otherwise.

\subsection{Prescience Tenets}

\subsubsection{Precognitive Step}
% phb3(4e) 89
% prescience
Starting at 6th level,
you can peer through the mists obscuring the future,
learning how one event unfolds.
At the start of every encounter, roll a d20.
Once during the same encounter,
you can at any time replace one of your attack rolls,
saving throws or skill checks with that roll,
adding modifiers on top of it as usual.
You can also replace an enemy's attack roll against you.

\subsubsection{Epitaph}
Starting at 10th level,
your hone your ability to see through time.
Once during an encounter,
if a hostile creature takes a turn immediately following yours,
you may use a free action at the start of your turn
to ask the DM once what the creature will do on their turn,
as conditioned by your explicit choices.
For example,
you may ask the DM what the creature will do
if you move 30 feet in a given direction and manifest a power.
If you deviate at all from your stated conditions,
or if you leave any ambiguity or uncertainty as to those choices,
the prediction may not necessarily come true.
If you do exactly as you say,
the creature is bound to act in exactly the way the DM described.

If multiple hostile creatures follow your turn,
you may use this ability for each of those creatures
until the turn order reaches a friendly creature
or the round ends.

\subsubsection{Other Memory}
Starting at 14th level,
your unlock the hereditary memories of your forebears,
allowing you to use their knowledge in your time of need.
Once per day, as an action you may call upon a particular ancestor
with a certain background as shown in the table below.
The nature of the benefit conferred to you depends on the background.
The benefits conferred to you end when you finish a long rest.

\begin{table}[htbp]%
    \begin{DndTable}[width=\columnwidth,
                     header=Other Memory Benefit]{
                     l X}
        Background          & Benefit                                                                   \\
        Warrior             & Gain proficiency in all armours,
                                martial weapons and Strength (Athletics) \\
        Battlemage          & You learn three 5th-level or lower
                                evocation spells of your choice
                                which can be cast without expending a spell slot;
                                once you cast any of these spells,
                                you can't cast the same spell again;
                                the spellcasting ability for these
                                spells is Intelligence \\
        Commander           & You learn three Uberium faction abilities \\
        Thief               & You gain an extra bonus action \\ 
        Priest              & You learn three 5th-level or lower
                                spells from the cleric spell list
                                of your choice;
                                these can be cast without expending a spell slot;
                                once you cast any of these spells,
                                you can't cast the same spell again;
                                the spellcasting ability for these
                                spells is Intelligence \\\
        Musician            & You gain proficiency in a musical instrument
                                and Charisma (Performance), or expertise if
                                you are already proficient in that skill
    \end{DndTable}
\end{table}

If you spend subsequent days relying on the same ancestor
and their abilities,
you run the risk of them usurping your personality.
For each consecutive day you use the same ancestor,
make a Wisdom saving throw with DC equal to 10 plus the number
of consecutive days.
This roll cannot be influenced by any Prescience powers
or other abilities which allow you to replace rolls.
If you fail the save,
hand your character sheet to the DM and roll a new character.

\subsection{Prana Bindu Tenets}
\label{sub:prana_bindu_tenets}

\subsubsection{Lightning Warfare}
Starting at 6th level,
your mastery of nerve and muscle
allows you to move at incredible speeds.
Your movement speed increases by 20 ft.
In addition, your movement no longer provokes
opportunity attacks.

\subsubsection{Optimised Reflexes}
Starting at 10th level,
your reaction times become almost negligible.
You gain an additional reaction on top of the one
you already have, which can be used in the same way
as the latter.

\subsubsection{Alter Metabolism}
Starting at 14th level,
your mastery over body now extends to your metabolism,
and you are able to by thought alone
alter the chemical makeup of malign substances.
You become immune to the effects of all poisons,
poison damage, the poisoned condition, and any toxic venoms
or secretions.
You also gain resistance to acid damage,
and you become immune to disease.

Additionally, you become immune to the maluses of extreme heat,
extreme cold and high altitude,
unless the conditions are so severe
that the DM decides otherwise.

\subsection{Voice Tenets}

\subsubsection{Inner Voice}
Starting at 6th level,
you gain telepathy up to 120 feet.
Once per day,
you may also send a message up to 10 words long
to a creature you have seen before and that is
presently within 24 miles of you.

Additionally,
after you have communicated telepathically with a creature
and that creature is the target of a Voice power with the Speech
requirement that you manifest,
the creature no longer needs to be able to understand you.

\subsubsection{Subtle Voice}
Starting at 10th level,
your mastery of the Voice allows you to manifest Voice
powers without arousing suspicion.
Any Voice power with the Speech requirement,
therefore producing an audible effect
(see \secref{subs:voice_powers}),
now is indistinguishable from normal speech.
Namely,
creatures which would ordinarily be alerted
to your Voice by distinguishing it from normal speech
now no longer are.
\emph{Exception:} Psions who have chosen the Voice discipline
cannot be affected by Subtle Voice.

\subsubsection{Enthrall Creature}
Starting at 14th level,
you gain the ability to dominate other creatures and have them
carry out your bidding.
Twice per day,
you may manifest the \spell{geas} spell as a power,
which counts as a 5th-level Voice power with no PD or MB cost.
Subtle Voice also applies for this power.

\subsection{Metacreativity Tenets}
\label{sub:metacreativity_tenets}
\subsubsection{Create Golem}
\label{subs:menu_a}
Starting at 6th level,
you learn the \nameref{pwr:psionic-golem} power,
which allows you to manifest a psionic golem
from mundane matter.
You are mentally linked to the psionic golem,
and they answer to your bidding as described in the power.
The statistics of the golem depend on the level
at which \nameref{pwr:psionic-golem} is cast.
For example,
casting the power at 3rd-level creates a 3rd-level psionic golem.
The statblocks by golem level are given in \secref{sec:psionic_golem}.

\begin{table}[htbp]%
    \begin{DndTable}[width=\columnwidth,
                     header=Psionic Golem Abilities (Menu A)]{
                     l X}
        Name                & Benefit   \\
        Resilient           & The golem gains an extra 10 hit points \\
        Sturdy              & The golem gains a +2 bonus to its AC \\
        Vertical Propulsion & The golem gains a fly speed equal to 15 feet \\ 
        Celerity            & The golem gains the mobile feat \\
        Resistance          & The golem gains resistance to fire, lightning, cold
                                or thunder damage \\
        Seaworthy           & The golem gains a swim speed of 30 feet \\
        Aggressive          & If the golem hits a creature with a melee attack,
                                it can use its reaction to attempt to shove
                                that creature prone
    \end{DndTable}
\end{table}

Each manifested golem has a certain number of
\define{metacreative points} (MP)\index{metacreative point (MP)}.
This is called the \define{MP potential}\index{MP potential} of the golem.
Metacreative points allow you to augment the golem
with particular abilities.
At 6th level,
you have access to the abilities given in Menu A
(see the corresponding table),
all of which have an MP cost of 1.
You cannot add abilities to the golem such that the total MP cost
is above the MP potential of the golem.
For example,
a 4th-level golem with an MP potential of 2
can have two Menu A abilities.
You may only choose each ability once.

\subsubsection{Improved Psionic Golem}
\label{subs:menu_b}
Starting at 10th level,
you gain access to psionic golem abilities on Menu B,
as shown in the table below.
The abilities on Menu B cost 2 MP.

\begin{table}[htbp]%
    \begin{DndTable}[width=\columnwidth,
                     header=Psionic Golem Abilities (Menu B)]{
                     l X}
        Name                & Benefit   \\
        Energy Touch        & The golem deals an extra 1d4 points of
                                fire, cold, electricity or thunder damage when it
                                hits with a melee attack \\
        Extra Attack        & Whenever the golem takes the \define{Multiattack} action,
                                it can make one additional slam attack \\
        Auto-reassembly     & The golem heals 5 hit points each round \\ 
        Heavy Deflection    & The golem gains a +4 bonus to its AC  \\
        Highly Resilient    & The golem gains an extra 20 hit points \\
        Improved Critical   & The golem lands a critical hit on a roll of either
                                19 or 20 \\
        Muscular            & The golem gains a +4 bonus to its Strength score \\
    \end{DndTable}
\end{table}

\subsubsection{Heart of Ascension}
\label{subs:menu_c}
Starting at 14th level,
any golem you manifest with \nameref{pwr:psionic-golem}
becomes Ascendant.
You gain access to psionic golem abilities on Menu C,
as shown in the table below.
The abilities on Menu C cost 4 MP.

\begin{table}[htbp]%
    \begin{DndTable}[width=\columnwidth,
                     header=Psionic Golem Abilities (Menu C)]{
                     l X}
        Name                & Benefit   \\
        WTF is Truesight!?  & The golem gains truesight out to 60 feet \\
        Supreme Resilience  & The golem gains an extra 30 hit points \\
        Metal Carapace      & All incoming damage from separate sources
                                is reduced by 5 \\ 
        Shield!             & The golem gains a +6 bonus to its AC \\
        Active Camouflage   & The golem becomes naturally invisible \\
        Energy Bolt         & The golem can manifest the \nameref{pwr:energy-bolt} power
                                as an action once per round; the energy type is
                                chosen when the golem is initially manifested
    \end{DndTable}
\end{table}

\subsection{Spacefolding Tenets}
\subsubsection{Length Contraction}
Starting at 6th level,
you become capable of dramatically contracting the space between you
and another creature with your mind alone. 
As an action,
choose a creature you can see within 60 feet and spend 3 psi dice.
That creature is pulled up to 30 feet in your direction as the space
between you and them is eliminated.

\subsubsection{Wormhole}
Starting at 10th level,
you alter the fabric of space and time such that
you can link two different locations together.
Once per day as an action,
you may create a wormhole mouth at a point you are standing on.
The mouth is a 10-foot radius sphere and is presently closed.
At a later point in time,
as an action you may create another mouth at a different location
(no per day restriction).
The location need not be on the same plane of existence
as the mouth.
When you do so, the two mouths become adjacent to each other.
Creatures on either side can see to the other side of the wormhole
through the 10-foot sphere,
and can step through using their movement.

You may choose to collapse the wormhole at any time using an action.
However, it is not stable.
For every hour that has elapsed since you placed the second mouth,
roll a d10.
On a roll of 1--5,
the wormhole collapses and the bridge between the two locations
is lost.
You must then create a new wormhole.

\subsubsection{Master of Spacetime}
Starting at 14th level,
you can effortlessly warp space and time locally to
move yourself across the battlefield.
The \nameref{pwr:psionic-step} power costs no psi dice
and occupies no mental bandwidth slots
whenever you manifest it.
The power still costs 4 psi dice to augment,
however if you do so,
you can manifest the power as a free action
instead of a bonus action
(limit of once per turn).

\subsection{Psychokinesis Tenets}
\subsubsection{Sculpt Psioncs}
Starting at 6th level,
akin to the evocation wizard's ability to sculpt spells,
you can alter the direction that hazardous effects
created by your psychokinesis powers propagate.
When you manifest a psychokinesis power that affects
other creatures that you can see,
you can choose a number of them equal to your
psionic threshold.
The chosen creatures automatically succeed on any saving
throws they make to resist the power,
taking no damage on a successful save if they would
otherwise take half.
If the power does not prompt a saving throw
(e.g. if it automatically hits),
the chosen creatures also take no damage.

\subsubsection{Signature Damage}
Starting at 10th level,
you become highly proficient in dealing damage
with a specific energy manifested by your psionic powers.
Choose a damage type from fire, cold, lightning and thunder.
Each die of damage you deal with the chosen type
become \define{exploding die}\index{exploding die}.
If you roll the maximum result on an exploding die,
you can reroll that die again and add the result to the
total damage.
If the result is again the maximum,
you may `explode' the die again.
This continues until you no
longer rolled the maximum value over all dice.

\subsubsection{Signature Power}
Starting at 14th level,
one psychokinesis power becomes so familiar to you that you
manifest it without straining yourself.
Choose one 3rd-level psychokinesis power.
The MB cost for that power is reduced to 0 whenever you manifest it.
In addition, you can manifest this power without expending
any psi die.
Once you do so,
you can't do so again until you finish a short or long rest.

\clearpage\section{Psi Knight}
\label{sec:psi_knight}
\DndSetThemeColor[DmgCoral]
With a thought alone, the silver-armoured half-orc effortlessly
shunts aside the first gnoll before plunging his longsword
into the abdomen of the next.
Then, with near-imperceptible speed, he appears behind the third gnoll,
who scarcely processes the gruesome end he meets.

In the silver-purple depths of the Astral Plane,
the githyanki astral skiff careens into the illithid nautiloid.
Before long, the erstwhile slaves of shattered Astromundi Nova
engage the mind flayers,
battling each other with thoughts alone.
Then the githyanki vanguard charges,
their silver swords slicing into the exposed flesh of their former
masters.

In the subterranean halls of a long-forgotten city,
the elf uses her mind to propel herself across a seemingly
bottomless crevasse.
Then, landing on the other side,
she shapes the very earth itself into a glowing handaxe,
which she picks up and throws at the awakened undead
bearing down upon her.

In the great open-air arena of Pax Illium,
the Prana Bindu champion bathes in the swelling roar
of the audience.
He fights with no weapon nor armour,
but with fist and mind alone.
Without any sign of strain,
he shoves a grappled gladiator to the floor
before delivering a swift kick into the abdomen of another,
careening them through through the electrified air.
The sound of their impact into the blood-soaked sand
is scarcely heard above the crowd's cheers.

\subsection{The Body and the Mind}
Psi Knights supplement their martial abilities
with the latent power of their mind to
achieve victory over their enemies.
Formidable opponents,
psi knights are rare and highly accomplished warriors
who reach an almost unparalleled balance between mind
and body.
Adventuring parties often readily accept a noble psi knight
by their side,
and the psi knight's enemies on the battlefield tremble
as they approach.

Like psions, psi knights manifest powers with myriad effects.
While they cannot manifest as many powers as psions,
they have substantial martial capacity
and have a greater ability to weather the tide of battle. 

\subsubsection{Creating a Psi Knight}
When creating a psi knight,
ask yourself why your character has chosen the path of psionics.
Perhaps they belonged to a small community of like-minded warriors
who understood the untapped potential of psionic powers.
Or perhaps they ventured down a solitary path,
testing the powers they learnt piecemeal in the din of battle.

Also ask yourself how they interact with the broader community.
Do they see themselves as an ordinary person with a fortuitous
talent?
Or do they feel that they rise above the masses?

\subsubsection{Quick Build}
You can make a psi knight quickly by following these suggestions.
First, Strength or Dexterity should be your highest ability score,
followed by Wisdom then Constitution.
Second, choose the soldier background.
Third, pick the \nameref{pwr:boots-of-the-demon-king} power
from the psi knight class power list.

\begin{DndSidebar}[float=htbp]{Multiclassing with the Psi Knight}
    \tocside{Multiclassing with the Psi Knight}In order to
    multiclass into the psi knight class,
    a character must have a minimum Strength or Dexterity score of 13
    and a minimum Wisdom score of 13
    (in addition to the usual score requirement for their base class).
    When you multiclass into the psi knight class,
    you gain proficiency with
    light armour, medium armour, shields, simple weapons
    and martial weapons.
\end{DndSidebar}

\begin{figure*}[t]
    \begin{ornamentedtabular}{c c >{\raggedright\arraybackslash}p{4.3cm} c c c c}[title={The Psi Knight}]
        \textbf{Level} & \textbf{P. Bonus} & \textbf{Features} & \textbf{Psi Dice} & \textbf{Mental Bandwidth} & \textbf{Powers Known} & \textbf{Psionic Threshold} \\
        1st  & +2 & Powers, Fighting Style, Weapon Mastery & 1   & 1  & 1  & 1 \\
        2nd  & +2 & Prana Bindu Initiate                   & 2   & 1  & 2  & 1 \\
        3rd  & +2 & Psionic Tradition                      & 3   & 1  & 3  & 1 \\
        4th  & +2 & Ability Score Improvement, Power Nap   & 4   & 2  & 4  & 2 \\
        5th  & +3 & Extra attack                           & 5   & 2  & 5  & 2 \\
        6th  & +3 & Psionic Tradition feature              & 7   & 2  & 6  & 2 \\
        7th  & +3 & Iron Will                              & 8   & 3  & 7  & 3 \\
        8th  & +3 & Ability Score Improvement              & 9   & 3  & 8  & 3 \\
        9th  & +4 & ---                                    & 11  & 3  & 9  & 3 \\
        10th & +4 & Psionic Tradition feature              & 12  & 3  & 10 & 4 \\
        11th & +4 & Calmness of Self                       & 13  & 4  & 11 & 4 \\
        12th & +4 & Ability Score Improvement              & 15  & 4  & 12 & 4 \\
        13th & +5 & ---                                    & 16  & 4  & 13 & 5 \\
        14th & +5 & ---                                    & 18  & 4  & 14 & 5 \\
        15th & +5 & Psionic Tradition feature              & 19  & 4  & 15 & 5 \\
        16th & +5 & Ability Score Improvement              & 21  & 5  & 16 & 6 \\
        17th & +6 & ---                                    & 22  & 5  & 17 & 6 \\
        18th & +6 & Psionic Tradition feature              & 24  & 5  & 18 & 6 \\
        19th & +6 & Ability Score Improvement              & 25  & 5  & 19 & 6 \\
        20th & +6 & Psionic Reserve                        & 27  & 5  & 20 & 6 \\
    \end{ornamentedtabular}
\end{figure*}

\subsection{Class Features}
As a psi knight, you gain the following features.

\subsubsection{Hit Points}

\begin{description}
    \item[Hit Dice:] 1d10 per psi knight level
    \item[Hit Points at 1st Level:] 10 + your Constitution modifier
    \item[Hit Points at Higher Levels:] 1d10 (or 6) +
        your Constitution modifier per psion level after 1st
\end{description}

\subsubsection{Proficiencies}

\begin{description}
    \item[Armour:] All armour, shields
    \item[Weapons:] Simple weapons, martial weapons
    \item[Tools:] None \vspace{4pt}
    \item[Saving Throws:] Constitution, Wisdom
    \item[Skills:] Choose two from Athletics, Acrobatics, Insight,
        Perception, Animal Handling, Intimidation, Survival
\end{description}

\subsubsection{Equipment}
You start with the following equipment,
in addition to the equipment granted by your background:
\begin{itemize}
    \item (a) chain mail or
          (b) leather armour, a longbow
            and 20 arrows
    \item (a) a martial weapon and a shield or
          (b) two martial weapons
    \item (a) a light crossbow and 20 bolts or
          (b) two handaxes
    \item (a) a dungeoneer's pack or
          (b) an explorer's pack
\end{itemize}

\subsection{Fighting Style}
You adopt a particular fighting style as your specialty.
Choose one of the following options.
You can't take a Fighting Style option more than once,
even if you later get to choose again.

\subsubsection{Blind Fighting}
You have blindsight with a range of 10 feet.
Within that range, you can effectively see anything that
isn't behind total cover,
even if you're blinded or in darkness.
Moreover, you can see an invisible creature within that range,
unless the creature successfully hides from you.

\subsubsection{Defence}
While you are wearing armour, you gain a +1 bonus to AC.

\subsubsection{Dueling}
When you are wielding a melee weapon in one hand
and no other weapons,
you gain a +2 bonus to damage rolls with that weapon.

\subsubsection{Great Weapon Fighting}
When you roll a 1 or 2 on a damage die for an attack you make
with a melee weapon that you are wielding with two hands,
you can reroll the die and must use the new roll,
even if the new roll is a 1 or a 2.
The weapon must have the two-handed or versatile property
for you to gain this benefit.

\subsubsection{Interception}
When a creature you can see hits a target,
other than you,
within 5 feet of you with an attack,
you can use your reaction to reduce the damage the target takes
by 1d10 + your proficiency bonus (to a minimum of 0 damage).
You must be wielding a shield or a simple or martial weapon
to use this reaction.

\subsubsection{Protection}
When a creature you can see attacks a target other than you
that is within 5 feet of you,
you can use your reaction to impose disadvantage on the attack roll.
You must be wielding a shield.

\subsection{Powers}
\subsubsection{Psi Dice}
The number of powers that you can manifest
is limited by the number of psi dice that
you have available.
This is the principal limit on the psionic output of a psi knight.
For example, a 9-th level psi knight with {\pklvlnine} psi dice
can manifest a power costing 1 psi dice {\pklvlnine} times.

The number of available psi dice per level
is shown in the psion class table.
You regain all expended psi dice when you finish
a long rest.

\subsubsection{Mental Bandwidth}
The powers you can manifest without straining yourself
is reflected by your mental bandwidth.
The total number of mental bandwidth slots you have
is shown in the class table.

\subsubsection{Powers Known}
At 1st level,
you choose one psion power of your choice
from the psi knight class power list.
When you gain a psi knight level,
you can learn additional powers from the class list
on top of the ones you already know.
The total numbers of powers you know cannot
exceed your powers known value,
as shown in the class table.

The powers you have selected in this fashion have been committed
to memory, and you need not prepare a selection of them each long rest.
Also,
whenever you take a long rest,
you may substitute any number of powers you already know
for ones you do not know on the psi knight class list.
The power that has been replaced counts as no longer being known.

\subparagraph{Psionic Threshold}
You cannot learn nor manifest powers with a level
greater than your psionic threshold,
shown in the class table.
As mentioned in \secref{sub:augmenting},
you also cannot augment powers to a level
beyond your psionic threshold.

\subsubsection{Manifesting Ability}
Your psionic ability modifier is your Wisdom modifier.
The save DC against your psionic powers and your
psionics attack modifier are therefore respectively:
\small\begin{equation*}
    \begin{gathered}
        \text{\textbf{Psionics save DC}}
            = 8 + \text{your proficiency bonus} + \\
                  \text{your Wisdom modifier} \\
        \text{\textbf{Psionics attack modifier}}
            = \text{your proficiency bonus} + \\
              \text{your Wisdom modifier}
    \end{gathered}
\end{equation*}\normalsize

\subsection{Weapon Mastery}
Starting at 1st level,
your training with weapons allows you to use the mastery properties
of two kinds of weapons of your choice with which you have proficiency,
such as longswords and handaxes.

Whenever you finish a long rest,
you can change the kinds of weapons you chose.
For example, you could switch to using the mastery properties of
halberds and flails.

\subsection{Prana Bindu Initiate}
Starting at 2nd level,
your mind-body training makes manifesting powers
belonging to the Prana Bindu discipline
vastly easier.
Prana Bindu powers require one less mental bandwidth slot
than mentioned in their description (minimum of 0)
and cost 1 less psi die (minimum of 0).
If the psi dice cost is reduced to zero in this manner,
it is called a \define{0PD power}\index{power!0PD}.
You can only manifest a number of 0PD powers
equal to your proficiency bonus per long rest.
Each additional 0PD power you manifest
beyond this limit costs the usual 1 psi die.

\subsection{Psionic Tradition}
When you reach 3rd level,
you embark on a journey down a particular
psionic tradition.
The tradition represents the aspects of
your mind-body mastery that you have decided
to focus on.
At this stage, two traditions are offered:
the Weirding Way, which focuses on your mastery
of devastatingly precise and supremely fast
Prana Bindu techniques,
and the Commander,
who couples their experience in battle
with profound psionic powers to meticulously
plan and execute combat stratagems.
Your tradition grants you features at 3rd level,
and again at 6th, 10th, 15th and 18th level.

\subsection{Ability Score Improvement}
When you reach 4th level,
and again at 8th, 12th, 16th and 19th level,
you can increase one ability score of your choice by 2,
or you can increase two of your ability scores by 1.
As normal,
you can't increase an ability score above 20 using this feature.

\subsection{Power Nap}
Starting at 4th level,
you may enter a state of meditation
in which you refresh your mind.
The meditation lasts 1 minute and
you must maintain concentration throughout.
If you do so,
at the end of the meditation you may regain a number of psi dice
equal to your Wisdom modifier multiplied by your psionic threshold.
You cannot accumulate psi dice past the maximum number you have available
as per your class(es) and level.

Once you have used this feature,
you cannot use it again until you have finished a long rest.

\subsection{Extra Attack}
Beginning at 5th level, you can attack twice,
instead of once,
whenever you take the Attack action on your turn.

\subsection{Iron Will}
Starting at 7th level,
your mind is able to override your body at times
when the untrained would falter.
Whenever you are reduced to hit points,
but not killed outright,
you can spend a psi die to immediately regain hit points
equal to your Wisdom modifier plus the number of levels
you have in the psi knight class.
Once you have used this feature,
you cannot use it again until you finish a long rest.

\subsection{Calmness of Self}
Starting at 11th level,
your unfettered desire to improve mind and body
has granted you a sense of inner calm.
You have resistance to psychic damage.
Moreover, if you start your turn charmed or frightened,
you can expend 3 psi dice and end every effect on yourself
subjecting you to those conditions.
In addition, you have advantage on saving throws to resist
effects from Voice psionic powers.

\subsection{Psionic Reserve}
Starting at 20th level,
when you roll for initiative and have no psi dice remaining,
you regain 8 psi dice.

\subsection{Weirding Way}
The Weirding Way champions incredible control over one's nerves
and reaction times---granted by Prana Bindu techniques---to
manoeuvre and strike at opponents before they can retaliate.
The name `Weirding Way' comes from the fact that,
to untrained opponents, the movements of those versed in
the Weirding Way appear unnatural and superhuman,
almost as if they are teleporting across the battlefield.

\subsubsection{Prana Bindu Prodigy}
Starting at 3rd level,
you can choose to learn any Prana Bindu tenet from
the psion discipline at \secref{sub:prana_bindu_tenets}.
The level requirements stated there do not apply to you.

\subsubsection{Seize the Initiative}
Starting at 6th level,
whenever a fight breaks out,
you can reliably act before your enemies.
If you spend a psi die,
you can grant yourself advantage on an initiative roll.

\subsubsection{CQC Dominance}
Starting at 10th level,
you make quick use of off-hand and blunt strikes to
achieve dominance in CQC (`close quarters combat').
You have advantage on all checks when taking the
grapple, shove, shove aside, or overrun action.

In addition,
whenever a creature within 5 feet of you ends their turn,
you can use your reaction and spend a psi die
to make a single attack against that creature.
The damage type is replaced with bludgeoning damage,
since you use your off-hand,
a kick, a punch or the blunt end of your weapon
to make the attack.
Whether or not the attack is made with a weapon
or is an unarmed attack is your choice.

\subsubsection{Slippery Movement}
Starting at 15th level,
your incredible agility makes its difficult for your
enemies to pin you down.
You have advantage on all saving throws against effects
which would slow your movement or reduce it to 0.
In addition,
if you become restrained or grappled,
you can use a bonus action on your turn
to attempt to repeat the save
or contest respectively which caused that condition.
You have advantage on this save.

\subsubsection{Lisan al Gaib}
Starting at 18th level,
your perfection of Prana Bindu methods becomes legendary,
and in the eyes of many your are considered a
sublime master of the body and mind.
You may pick an additional Prana Bindu tenet from
\secref{sub:prana_bindu_tenets}.

\subsection{Commander}
The Commander uses their experience and psionic abilities
to control and manipulate the battlefield as they desire.
Enemies fear a psi knight commander,
for they know that they will struggle to ever
gain tactical advantage over them and their allies.

\subsubsection{Wise Strikes}
Starting at 3rd level,
your battlefield experience ensures that your attacks
are as damaging as they can be for your enemy.
You may add your Wisdom modifier to all damage
rolls for melee attacks. 

\subsubsection{Knowledge is Power}
Starting at 6th level,
once per day as a bonus action,
you choose a hostile creature you can see within
30 feet of you.
Your mind links with them,
and you gain a deep understanding of their
strengths, weaknesses and peculiarities.
You may then ask the DM the exact amount of any of the following
characteristics of the hostile creature:
\begin{itemize}
    \item Its full stat array;
    \item Its AC;
    \item Its total hit points;
    \item Its to-hit modifier on one type of attack; or,
    \item Its damage roll, including modifiers, for one type
            of attack.
\end{itemize}

\subsubsection{Order of Battle}
Starting at 10th level,
your mind's eye expands,
and you begin to see the battlefield from a top-down perspective.
Enemies and allies are pieces on a board, which you can manipulate
with thought alone.
Once per day,
immediately after initiative is rolled at the start of an encounter,
you may attempt to declare the initiative order for
any number of combatants.
For each creature you intend to affect,
they must make a Wisdom saving throw against your psionics save DC.
If they fail, you may replace their initiative with any number.
If they succeed, their initiative is unaffected. 

This ability does not apply when you are surprised.
You also cannot alter the initiative order of creatures
in the encounter that you have no knowledge of,
those that join later in the fight (such as those
that are summoned),
and environmental effects or lair actions
which act on a set initiative count.

\subsubsection{Gambit}
Starting at 15th level,
your see fighters on a battlefield as pieces on a chessboard
that can be effortlessly manipulated---you simply turn your
mind to them, and they bend to your will.
You have a pool of 120 feet of movement.
On your turn,
you may use your action to attempt to move any number of creatures
cumulatively up to 120 feet in any direction,
including vertically.
Each creature you wish to move can resist this movement by
succeeding a Wisdom saving throw with DC equal to your
psionics save DC.

Any unspent movement remaining in your pool persists
and can be carried over to subsequent turns.
Your pool of movement replenishes whenever you finish a long rest.

\subsubsection{Only a Simulation}
Sometimes the tide of battle can be unpredictable.
Fortunately for you,
starting at 18th level,
your supreme mental abilities confer you the ability to
run combat simulations viewed internally by your mind's eye.
These simulations allow you to decide the best course of action.

Once per day, on your turn,
you can use an action to undo the previous round of combat,
allowing you to `replay' that round.
This ability reverses time to the point in the past
just prior to your previous turn,
undoing the effects of everyone else's actions in the meantime.
Once you have used Only a Simulation,
only you retain knowledge of what happened during the round
that is being replayed; after all, it was just a simulation.
However, you can communicate that knowledge verbally to your companions,
if desired,
and you can act on that privileged knowledge
in any way you desire.