\chapter{Introduction}
\epigraph{\itshape
  Abruptly, as though he had found a necessary key,
  Paul's mind climbed another notch in awareness.
  He felt himself clinging to this new level,
  clutching at a precarious hold and peering about.
  It was as though he existed within a globe with
  avenues radiating away in all directions\ldots yet this only
  approximated the sensation.
  
  \vspace{2mm}He remembered once seeing a gauze kerchief blowing in
  the wind and now he sensed the future as though it twisted
  across some surface as undulant and impermanent as that of
  the windblown kerchief\ldots
  
  \vspace{2mm}The thing was a spectrum of possibilities from the most
  remote past to the most remote future --- from the most
  probable to the most improbable. He saw his own death in
  countless ways. He saw new planets, new cultures\ldots
  
  \vspace{2mm}As swiftly as it had come, the sensation slipped away from
  him, and he realized the entire experience had taken the space
  of a heartbeat.
  
  \vspace{2mm}Yet, his own personal awareness had been turned over,
  illuminated in a terrifying way.\normalfont}
  {Frank Herbert\\\textit{Dune} (1965)}

\DndDropCapLine{T}{he Old Way is the ancient doctrine}
of the mind.
It holds that each individual has a latent psionic talent,
which, for many,
is never awakened.
But those who do awaken that which lies dormant
are granted new, incredible and potentially terrifying
vistas of awareness---and the promise of great power.

This supplement expounds the rules around psionics:
a broad term referring to the powerful effects
manifested by the mind.
In \secref{chap:using_psionics},
the basic mechanics are explained,
and in \secref{chap:list_of_powers},
an exhaustive list of all available powers is set out.
Psionics are accessible by two new classes: the psion
and the psi knight.
These are presented in \secref{chap:classes}.
Lastly, miscellaneous additions and rules addenda
are given in \secref{chap:miscellany},
including new feats and changes to existing races.\newpage

This psionics system is based heavily on the
\emph{Expanded Psionics Handbook} of 3.5e.
Many powers have been taken from that work
and translated into the language of 5e.
That being said, many features and powers are new.
Some of these powers are based on works of fiction
the reader may be familiar with...
In any case,
if a player engages with the new psionics framework,
they should come in with the assumption that the rules
and balance are to change rapidly and regularly;
this system has not been play-tested.

\subsection*{Using this Supplement}
This supplement does not overwrite or supersede any
of the established rules in the base game,
except where otherwise noted.
To aid in clarity,
this supplement uses boldface for \define{game terms}
at their first instance in the text.
Game terms are specific phrases of words which
carry an immutable packet of meaning.
Unlike natural language,
which has contextual dependence,
game terms mean the same thing everywhere.
This is done to avoid ambiguity as best as possible.

Questions of interpretation surrounding psionics
must be resolved by the DM,
given that this is a new, stand-alone system.
Questions of interpretation surrounding the interaction
between the base game and psionics are similarly resolved.
Although it was stated that this supplement does not
overwrite the base game,
some degree of conflict is inevitable.
Priority is given to what is written in this supplement
as opposed to what is written in the base game
where this occurs.
Questions of interpretation related to the base game
are resolved as normal.