\chapter{Miscellany}
\label{chap:miscellany}
\DndDropCapLine{L}{astly, some additional features}
peripheral to the core rules are given here.
\secref{sec:psion_psi_knight_multiclassing} deals with the
possibility of multiclassing into two classes both with
the ability to manifest powers.
\secref{sec:feats} adds a selection of psionics-related
feats into the game.
\secref{sec:races} tweaks some of the already existing
races in the game who have significant psionic potential.
\secref{sec:psionic_traits} gives a list of traits pertaining to psionics
that may be possessed by creatures.
\secref{sec:psionic_items} introduces the rules surrounding items with psionic
properties, with \secref{sec:list_of_psionic_items} detailing
a handful of the ones that can be found in the world.
\secref{sec:spells} details a selection of new spells which are intended
to interact with powers and psi-like abilities.
Finally,
\secref{sec:psionic_golem} gives the stat blocks for the
psionic golems created by adherents of the
Metacreativity discipline.

\section{Psion/Psi Knight Multiclassing}
\label{sec:psion_psi_knight_multiclassing}
There is the possibility that a character chooses to multiclass
into both the psion and psi knight classes.
If this is the case,
the following rules apply with respect to the Powers feature
of each class.
\subparagraph{Psi Dice}
    Psi dice add cumulatively between the two classes.
    For example,
    if a character has 3 levels in psion (6 psi dice)
    and then multiclass into psi knight,
    they would gain 1 psi dice,
    since at first level the psi knight starts at 1 psi dice.
    If they took a further level in psi knight,
    they would gain 1 psi dice,
    and so on.
\subparagraph{Mental Bandwidth}
    Your mental bandwidth is the highest mental bandwidth you
    have across both classes.
\subparagraph{Powers Known}
    The same logic as for psi dice applies for
    determining the number of powers you know.
\subparagraph{Psionic Threshold}
    The same logic as for mental bandwidth applies for
    determining your psionic threshold.

\section{Feats}
\label{sec:feats}
This section adds the following feats on top of the already
existing ones.

\DndFeatHeader{Psionics Initiate}[%
    Prerequisite: Wisdom 13 (psi knight); Intelligence 13 (psion)]
    Choose either the psi knight or psion class.
    You learn two 1st-level powers from the powers list
    (see \secref{sec:class_lists})
    of that class assuming you meet the prerequisites
    for the nominated class given above.
    Each power counts as a (1/day) innate talent for you.
    Also, if you can manifest powers ordinarily,
    you may manifest each power using any psi dice you have,
    and these powers do not count against your powers known.
    Whenever you finish a long rest,
    you can replace one of the powers you have chosen
    with another from the same powers list.

    Your psionics ability for these powers depends on the class
    you choose:
    Wisdom for the psi knight class; Intelligence for the psion class.

\DndFeatHeader{Voice Resistance}[%
    Prerequisite: None]
    You have trained extensively and rigorously in the ways
    of resisting the dominating effects of the Voice.
    You gain advantage on all saving throws against
    Voice powers.
    You also know whenever a Voice power is manifested,
    even if the manifester has the \emph{Subtle Voice} ability.

\DndFeatHeader{Psionic Litany}[%
    Prerequisite: The ability to manifest at least one power
]
You have committed a psionic litany to memory, granting you respite
and focus in fear-inducing and hostile conditions.
You gain advantage on Constitution saving throws that you make
to maintain concentration on powers.
This applies when you take damage
or are otherwise subject to a condition which
threatens your concentration.
You may also increase your psionics ability by 1, to a maximum of 20.

\section{Races}
\label{sec:races}
This section makes the following changes to the racial traits
of creatures in the base game.

\subsection{Gith}
\subsubsection{Githyanki}
\emph{Githyanki Psionics} is replaced as follows.
\subparagraph{Githyanki Psionics}
    Your psionic heritage grants you a limited selection of
    psionic powers as innate talents.    
    You know the \nameref{pwr:psi-hand} power (3/day innate talent).
    Starting at 3rd level,
    you can manifest the \nameref{pwr:leap} power (1/day innate talent).
    Starting at 5th level,
    you can manifest the \nameref{pwr:psionic-step} power (1/day innate talent).
    Intelligence or Wisdom is your psionics ability for these powers
    (choose when you select this race).
    If you select the psion or psi knight class,
    the psionics ability becomes the underlying ability
    associated with manifesting powers for that class.

    Each innate talent also counts as powers known for you,
    so, if you wish,
    you can manifest these powers using psi dice if you have any.

\subsubsection{Githzerai}
\emph{Githzerai Psionics} is replaced as follows.
\subparagraph{Githzerai Psionics}
    Your psionic heritage grants you a limited selection of
    psionic powers as innate talents.    
    You know the \nameref{pwr:psi-hand} power (3/day innate talent).
    Starting at 3rd level,
    you know the \nameref{pwr:sacrificial-shell} power or the
    \nameref{pwr:detect-thoughts-psionic} power.
    Each count as a 1/day innate talent.
    Starting at 5th level,
    you know the \nameref{pwr:leap} power (1/day innate talent).

    Intelligence or Wisdom is your psionics ability for these powers
    (choose when you select this race).
    If you select the psion or psi knight class,
    the psionics ability becomes the underlying ability
    associated with manifesting powers for that class.

    Each innate talent also counts as powers known for you,
    so, if you wish,
    you can manifest these powers using psi dice if you have any.

\section{Psionic Traits}
\label{sec:psionic_traits}
The following section outlines special traits relating to psionics
that a creature might possess.

\subsection{Power Resistance}
\label{sub:power_resistance}
When a creature with \define{power resistance}\index{power!resistance} is the target of
or is affected by a psionic power, whether manifested normally or
as an innate talent, or is the target of or affected by a psi-like ability,
that creature has a chance of resisting the power or ability's effects.
This chance is reflected in the value of the power resistance,
denoted as power resistance ($x$) or PR$x$.

When the conditions for power resistance are met,
the creature manifesting the power or using the psi-like ability
must make a d20 check with a DC equal to
the creature's power resistance value, adding their psionics attack modifier
to the roll.
If the check fails, the creature with power resistance
is not affected by the power or psi-like ability.

A creature with power resistance can choose to voluntarily
succumb to the effects of a power or psi-like ability, even if
the check fails.

\section{Psionic Items}
\label{sec:psionic_items}
Some items have an affinity for the minds of others.
Such special items are called \textbf{psionic items}\index{psionics items},
and are analogous to magic items.
However, unlike magic items,
a psionic item is generally totally inert unless
it is harnessed by a trained psyche.
To such individuals,
psionic items can offer an incredible reserve of power,
as well as great boons which supplement an individual's own powers.

\subsection{Integration}
A magic item is attuned to by its user.
Similarly, a psionic item is \define{integrated}\index{integrate} by its user;
the psionic powers conferred by a magic item cannot be used
unless it is integrated.
However, the rules for psionic item integration are different to
those for magic item attunement.

\subsubsection{MB Cost}
Each psionic item is associated with a mental bandwidth cost.
While a psionic item is integrated by a user,
a number of mental bandwidth slots equal to the MB cost of the item
are permanently occupied.
In addition, in order to integrate a psionic item,
the user must have available a number of mental bandwidth slots
equal to the MB cost of the item.
A user cannot integrate with a psionic item if, after integration,
all of their mental bandwidth slots are fully occupied.
This calculation includes slots occupied by powers that a manifester
is currently concentrating on.
Thus,
the limit on the number of psionic items a user can integrate
is determined by their mental bandwidth.

For example,
suppose a psion has 2 available mental bandwidth slots.
They have also integrated a psionic item with an MB cost of 1.
Thus, they have only 1 available mental bandwidth slot.
This slot could be occupied by a power or another psionic item,
but not if that psionic item has an MB cost greater than 1.

\subparagraph{0MB Items}
    Some psionic items have an MB cost of 0.
    Such items can be integrated by a user even if they do not
    have any free mental bandwidth slots.
    They can also be integrated by users with no mental bandwidth slots,
    such as those without any levels in psionic classes.

\subsection{Process of Integration}
Assuming a user meets the requirements set out above,
they must spend a short rest concentrating on and maintaining physical contact with
the item in order to integrate it.
The user may not spend hit die during this short rest.
In addition, if at any point the user's concentration is broken,
the integration process is broken and the user must start again.

A user cannot begin the process of integration if another user has already
integrated the item, or if they have already integrated a psionic item
with the same name and/or powers. 

\subparagraph{Ending Integration}
    After integrating a psionic item,
    a user may voluntarily end integration by spending another short rest
    concentrating on the item. As above, the user's concentration must not be
    broken for the duration of the short rest for this process to succeed.

\section{Spells}
\label{sec:spells}
In this section, a list of new spells designed to interact
with psionics are given.
\DndSpellHeader%
    {Logorrhea\label{spl:logorrhea}}
    {2nd-level Enchantment}
    {1 action}
    {Self (60-foot cone)}
    {V, S, M (a papyrus scroll and a vial of ink)}
    {Instantaneous}
You unleash a torrent of meaningless, sibilant sounds targeted
to distract those caught in the blast.
Any creature that can hear you and is
caught in a 60-foot cone centred on yourself must
make an Intelligence saving throw if they are concentrating on
a spell or power. If the target fails the save,
they immediately lose concentration. In addition,
regardless of whether they failed the save or not, a creature
caught in the cone has disadvantage on
Constitution saving throws made to maintain concentration on
a spell or power until the end of their next turn. %
%
\classes{Wizard, Sorcerer, Bard}
\DndSpellHeader%
    {Mental Link\label{spl:mental-link}}
    {3rd-level Abjuration}
    {1 action}
    {Touch}
    {V, S, M (a silver chain worth 50 gp)}
    {1 hour}
You link your mind with a willing creature, cushioning it from
the malign effects of hostile psionics.
While the target is within 60 feet of you, they have resistance
to any damage incurred from a psionic power
or psi-like ability, and may add a d4 to any saving throw
they make to resist a psionic power or psi-like ability.
However, whenever they take psychic damage---regardless
of whether it is caused by a psionic effect
or not---you take half that damage (after applying resistances),
with the target taking the other half.

The spell ends when either you or your target drops to 0 hit points.
You can also dismiss the spell as a free action on your turn.%
%
\classes{Wizard, Cleric}



\section{List of Psionic Items}
\label{sec:list_of_psionic_items}
\DndFeatHeader{Armour of Power Resistance}[%
    Psionic Item, MB 0]
While integrated with this armour,
you gain \textbf{power resistance} $\mathbf{x}$.
You therefore cannot be affected by a psionic power
unless the manifester succeeds on a manifesting check
against a DC of $x$,
with modifier equal to their psionic threshold.

\DndFeatHeader{Mindarmour}[%
    Psionic Item, MB 0]
While integrated with mindarmour,
you gain a +3 bonus on saving throws against all Voice powers,
as well as mind-affecting and compulsion powers.

\DndFeatHeader{Soulbreaker Weapon}[%
    Psionic Item, MB 0]
While integrated with a soulbreaker,
on a critical hit with a melee weapon attack with the soulbreaker,
the target must make a Wisdom saving throw.
The DC is equal to 8 + psionics attack modifier
(or just 8 + your Wisdom or Intelligence if you have no levels
in psionic classes).
If the target fails,
they are stunned until the end of their next turn.


\section{Psionic Golems}
\label{sec:psionic_golem}
The statblocks for the psionic golems described in
\secref{sub:metacreativity_tenets}
are given here, beginning on the following page.

\clearpage\begin{DndMonster}[float*=b,width=\textwidth + 8pt]{1st-Level Psionic Golem}
\begin{multicols}{2}
    \DndMonsterType{Medium construct, unaligned}
  
    \DndMonsterBasics[
        armor-class = {18 (natural armour)},
        hit-points  = {\DndDice{3d10 + 5}},
        speed       = {25 ft.},
      ]
  
    \DndMonsterAbilityScores[
        str = 14,
        dex = 11,
        con = 13,
        int = 1,
        wis = 3,
        cha = 1,
      ]
  
    \DndMonsterDetails[
        damage-immunities = {poison},
        condition-immunities = {blinded, charmed, deafened, exhaustion,
                                frightened, paralyzed, petrified, poisoned},
        senses = {darkvision 30 ft., passive Perception 6},
        languages = {---},
        challenge = 1,
        proficiency = +2
        ]
    % Traits
    \DndMonsterAction{Golem}
    The psionic golem cannot be put to sleep, is immune to disease
    and cannot be raised by necromancy or similar spells.
    The golem also cannot be healed by magic or other means,
    but can be repaired by manifested powers.

    \DndMonsterAction{MP Potential}
    The psionic golem has 1 MP, which may be spent on abilities
    in Menus you have access to.
    
    % Actions
    \DndMonsterSection{Actions}
    \DndMonsterAction{Multiattack}
    The psionic golem makes two slam attacks.
  
    \DndMonsterMelee[
      name=Slam,
      mod=+4,
      reach=5,
      targets=one target,
      dmg=\DndDice{1d6+2},
      dmg-type=bludegoning,
    ]
\end{multicols}
\end{DndMonster}

\begin{DndMonster}[float*=b,width=\textwidth + 8pt]{2nd-Level Psionic Golem}
\begin{multicols}{2}
    \DndMonsterType{Medium construct, unaligned}
  
    \DndMonsterBasics[
        armor-class = {18 (natural armour)},
        hit-points  = {\DndDice{4d10 + 6}},
        speed       = {30 ft.},
      ]
  
    \DndMonsterAbilityScores[
        str = 16,
        dex = 11,
        con = 15,
        int = 1,
        wis = 3,
        cha = 1,
      ]
  
    \DndMonsterDetails[
        damage-immunities = {poison},
        condition-immunities = {blinded, charmed, deafened, exhaustion,
                                frightened, paralyzed, petrified, poisoned},
        senses = {darkvision 30 ft., passive Perception 6},
        languages = {---},
        challenge = 1,
        proficiency = +2
      ]
    % Traits
    \DndMonsterAction{Golem}
    The psionic golem cannot be put to sleep, is immune to disease
    and cannot be raised by necromancy or similar spells.
    The golem also cannot be healed by magic or other means,
    but can be repaired by manifested powers.

    \DndMonsterAction{MP Potential}
    The psionic golem has 1 MP, which may be spent on abilities
    in Menus you have access to.
    
    % Actions
    \DndMonsterSection{Actions}
    \DndMonsterAction{Multiattack}
    The psionic golem makes two slam attacks.
  
    \DndMonsterMelee[
      name=Slam,
      mod=+5,
      reach=5,
      targets=one target,
      dmg=\DndDice{1d6+3},
      dmg-type=bludegoning,
    ]
\end{multicols}
\end{DndMonster}

\begin{DndMonster}[float*=b,width=\textwidth + 8pt]{3rd-Level Psionic Golem}
\begin{multicols}{2}
    \DndMonsterType{Medium construct, unaligned}
  
    \DndMonsterBasics[
        armor-class = {18 (natural armour)},
        hit-points  = {\DndDice{5d10 + 6}},
        speed       = {30 ft.},
      ]
  
    \DndMonsterAbilityScores[
        str = 16,
        dex = 11,
        con = 15,
        int = 1,
        wis = 3,
        cha = 1,
      ]
  
    \DndMonsterDetails[
        damage-immunities = {poison},
        condition-immunities = {blinded, charmed, deafened, exhaustion,
                                frightened, paralyzed, petrified, poisoned},
        senses = {darkvision 30 ft., passive Perception 6},
        languages = {---},
        challenge = 1,
        proficiency = +2
      ]
    % Traits
    \DndMonsterAction{Golem}
    The psionic golem cannot be put to sleep, is immune to disease
    and cannot be raised by necromancy or similar spells.
    The golem also cannot be healed by magic or other means,
    but can be repaired by manifested powers.

    \DndMonsterAction{MP Potential}
    The psionic golem has 1 MP, which may be spent on abilities
    in Menus you have access to.
    
    % Actions
    \DndMonsterSection{Actions}
    \DndMonsterAction{Multiattack}
    The psionic golem makes two slam attacks.
  
    \DndMonsterMelee[
      name=Slam,
      mod=+5,
      reach=5,
      targets=one target,
      dmg=\DndDice{2d6+3},
      dmg-type=bludegoning,
    ]
\end{multicols}  
\end{DndMonster}

\begin{DndMonster}[float*=b,width=\textwidth + 8pt]{4th-Level Psionic Golem}
\begin{multicols}{2}
    \DndMonsterType{Medium construct, unaligned}
  
    \DndMonsterBasics[
        armor-class = {18 (natural armour)},
        hit-points  = {\DndDice{6d10 + 12}},
        speed       = {30 ft.},
      ]
  
    \DndMonsterAbilityScores[
        str = 18,
        dex = 13,
        con = 17,
        int = 1,
        wis = 3,
        cha = 1,
      ]
  
    \DndMonsterDetails[
        damage-immunities = {poison},
        condition-immunities = {blinded, charmed, deafened, exhaustion,
                                frightened, paralyzed, petrified, poisoned},
        senses = {darkvision 30 ft., passive Perception 6},
        languages = {---},
        challenge = 4,
        proficiency = +2
      ]
    % Traits
    \DndMonsterAction{Golem}
    The psionic golem cannot be put to sleep, is immune to disease
    and cannot be raised by necromancy or similar spells.
    The golem also cannot be healed by magic or other means,
    but can be repaired by manifested powers.

    \DndMonsterAction{MP Potential}
    The psionic golem has 2 MP, which may be spent on abilities
    in Menus you have access to.
    
    % Actions
    \DndMonsterSection{Actions}
    \DndMonsterAction{Multiattack}
    The psionic golem makes two slam attacks.
  
    \DndMonsterMelee[
      name=Slam,
      mod=+6,
      reach=5,
      targets=one target,
      dmg=\DndDice{2d6+4},
      dmg-type=bludegoning,
    ]
\end{multicols}  
\end{DndMonster}

\begin{DndMonster}[float*=b,width=\textwidth + 8pt]{5th-Level Psionic Golem}
\begin{multicols}{2}
    \DndMonsterType{Large construct, unaligned}
  
    \DndMonsterBasics[
        armor-class = {19 (natural armour)},
        hit-points  = {\DndDice{7d10 + 12}},
        speed       = {35 ft.},
      ]
  
    \DndMonsterAbilityScores[
        str = 18,
        dex = 15,
        con = 17,
        int = 1,
        wis = 3,
        cha = 1,
      ]
  
    \DndMonsterDetails[
        damage-immunities = {poison},
        condition-immunities = {blinded, charmed, deafened, exhaustion,
                                frightened, paralyzed, petrified, poisoned},
        senses = {darkvision 30 ft., passive Perception 6},
        languages = {---},
        challenge = 5,
        proficiency = +3
      ]
    % Traits
    \DndMonsterAction{Golem}
    The psionic golem cannot be put to sleep, is immune to disease
    and cannot be raised by necromancy or similar spells.
    The golem also cannot be healed by magic or other means,
    but can be repaired by manifested powers.

    \DndMonsterAction{MP Potential}
    The psionic golem has 2 MP, which may be spent on abilities
    in Menus you have access to.

    \DndMonsterAction{Damage Threshold (5)}
    The psionic golem has immunity to all damage unless it takes
    an amount of damage from a single attack or effect equal to
    or greater than its damage threshold,
    in which case it takes damage as normal.
    
    % Actions
    \DndMonsterSection{Actions}
    \DndMonsterAction{Multiattack}
    The psionic golem makes two slam attacks.
  
    \DndMonsterMelee[
      name=Slam,
      mod=+7,
      reach=5,
      targets=one target,
      dmg=\DndDice{3d6+4},
      dmg-type=bludegoning,
    ]
\end{multicols}  
\end{DndMonster}

\begin{DndMonster}[float*=b,width=\textwidth + 8pt]{6th-Level Psionic Golem}
\begin{multicols}{2}
    \DndMonsterType{Large construct, unaligned}
  
    \DndMonsterBasics[
        armor-class = {19 (natural armour)},
        hit-points  = {\DndDice{9d10 + 12}},
        speed       = {40 ft.},
      ]
  
    \DndMonsterAbilityScores[
        str = 18,
        dex = 15,
        con = 19,
        int = 1,
        wis = 3,
        cha = 1,
      ]
  
    \DndMonsterDetails[
        damage-immunities = {poison},
        condition-immunities = {blinded, charmed, deafened, exhaustion,
                                frightened, paralyzed, petrified, poisoned},
        senses = {darkvision 30 ft., passive Perception 6},
        languages = {---},
        challenge = 6,
        proficiency = +3
      ]
    % Traits
    \DndMonsterAction{Golem}
    The psionic golem cannot be put to sleep, is immune to disease
    and cannot be raised by necromancy or similar spells.
    The golem also cannot be healed by magic or other means,
    but can be repaired by manifested powers.

    \DndMonsterAction{MP Potential}
    The psionic golem has 2 MP, which may be spent on abilities
    in Menus you have access to.

    \DndMonsterAction{Damage Threshold (8)}
    The psionic golem has immunity to all damage unless it takes
    an amount of damage from a single attack or effect equal to
    or greater than its damage threshold,
    in which case it takes damage as normal.
    
    % Actions
    \DndMonsterSection{Actions}
    \DndMonsterAction{Multiattack}
    The psionic golem makes two slam attacks.
  
    \DndMonsterMelee[
      name=Slam,
      mod=+7,
      reach=5,
      targets=one target,
      dmg=\DndDice{4d6+4},
      dmg-type=bludegoning,
    ]
\end{multicols}  
\end{DndMonster}

\begin{DndMonster}[float*=b,width=\textwidth + 8pt]{7th-Level Psionic Golem}
\begin{multicols}{2}
    \DndMonsterType{Large construct, unaligned}
  
    \DndMonsterBasics[
        armor-class = {19 (natural armour)},
        hit-points  = {\DndDice{11d10 + 12}},
        speed       = {40 ft.},
      ]
  
    \DndMonsterAbilityScores[
        str = 19,
        dex = 15,
        con = 19,
        int = 1,
        wis = 3,
        cha = 1,
      ]
  
    \DndMonsterDetails[
        damage-immunities = {poison},
        condition-immunities = {blinded, charmed, deafened, exhaustion,
                                frightened, paralyzed, petrified, poisoned},
        senses = {darkvision 30 ft., passive Perception 6},
        languages = {---},
        challenge = 7,
        proficiency = +3
      ]
    % Traits
    \DndMonsterAction{Golem}
    The psionic golem cannot be put to sleep, is immune to disease
    and cannot be raised by necromancy or similar spells.
    The golem also cannot be healed by magic or other means,
    but can be repaired by manifested powers.

    \DndMonsterAction{MP Potential}
    The psionic golem has 4 MP, which may be spent on abilities
    in Menus you have access to.

    \DndMonsterAction{Damage Threshold (12)}
    The psionic golem has immunity to all damage unless it takes
    an amount of damage from a single attack or effect equal to
    or greater than its damage threshold,
    in which case it takes damage as normal.
    
    % Actions
    \DndMonsterSection{Actions}
    \DndMonsterAction{Multiattack}
    The psionic golem makes two slam attacks.
  
    \DndMonsterMelee[
      name=Slam,
      mod=+7,
      reach=5,
      targets=one target,
      dmg=\DndDice{4d6+4},
      dmg-type=bludegoning,
    ]
\end{multicols}  
\end{DndMonster}

\begin{DndMonster}[float*=b,width=\textwidth + 8pt]{8th-Level Psionic Golem}
\begin{multicols}{2}
    \DndMonsterType{Large construct, unaligned}
  
    \DndMonsterBasics[
        armor-class = {19 (natural armour)},
        hit-points  = {\DndDice{15d10 + 12}},
        speed       = {40 ft.},
      ]
  
    \DndMonsterAbilityScores[
        str = 20,
        dex = 15,
        con = 20,
        int = 1,
        wis = 3,
        cha = 1,
      ]
  
    \DndMonsterDetails[
        damage-immunities = {poison},
        condition-immunities = {blinded, charmed, deafened, exhaustion,
                                frightened, paralyzed, petrified, poisoned},
        senses = {darkvision 30 ft., passive Perception 6},
        languages = {---},
        challenge = 8,
        proficiency = +3
      ]
    % Traits
    \DndMonsterAction{Golem}
    The psionic golem cannot be put to sleep, is immune to disease
    and cannot be raised by necromancy or similar spells.
    The golem also cannot be healed by magic or other means,
    but can be repaired by manifested powers.

    \DndMonsterAction{MP Potential}
    The psionic golem has 4 MP, which may be spent on abilities
    in Menus you have access to.

    \DndMonsterAction{Damage Threshold (15)}
    The psionic golem has immunity to all damage unless it takes
    an amount of damage from a single attack or effect equal to
    or greater than its damage threshold,
    in which case it takes damage as normal.
    
    % Actions
    \DndMonsterSection{Actions}
    \DndMonsterAction{Multiattack}
    The psionic golem makes two slam attacks.
  
    \DndMonsterMelee[
      name=Slam,
      mod=+8,
      reach=5,
      targets=one target,
      dmg=\DndDice{4d6+5},
      dmg-type=bludegoning,
    ]
\end{multicols}  
\end{DndMonster}

\begin{DndMonster}[float*=b,width=\textwidth + 8pt]{9th-Level Psionic Golem}
\begin{multicols}{2}
    \DndMonsterType{Huge construct, unaligned}
  
    \DndMonsterBasics[
        armor-class = {20 (natural armour)},
        hit-points  = {\DndDice{18d10 + 12}},
        speed       = {40 ft.},
      ]
  
    \DndMonsterAbilityScores[
        str = 20,
        dex = 15,
        con = 20,
        int = 1,
        wis = 3,
        cha = 1,
      ]
  
    \DndMonsterDetails[
        damage-immunities = {poison},
        condition-immunities = {blinded, charmed, deafened, exhaustion,
                                frightened, paralyzed, petrified, poisoned},
        senses = {darkvision 30 ft., passive Perception 6},
        languages = {---},
        challenge = 9,
        proficiency = +4
      ]
    % Traits
    \DndMonsterAction{Golem}
    The psionic golem cannot be put to sleep, is immune to disease
    and cannot be raised by necromancy or similar spells.
    The golem also cannot be healed by magic or other means,
    but can be repaired by manifested powers.

    \DndMonsterAction{MP Potential}
    The psionic golem has 8 MP, which may be spent on abilities
    in Menus you have access to.

    \DndMonsterAction{Damage Threshold (20)}
    The psionic golem has immunity to all damage unless it takes
    an amount of damage from a single attack or effect equal to
    or greater than its damage threshold,
    in which case it takes damage as normal.
    
    % Actions
    \DndMonsterSection{Actions}
    \DndMonsterAction{Multiattack}
    The psionic golem makes two slam attacks.
  
    \DndMonsterMelee[
      name=Slam,
      mod=+9,
      reach=5,
      targets=one target,
      dmg=\DndDice{4d6+5},
      dmg-type=bludegoning,
    ]
\end{multicols}
\end{DndMonster}