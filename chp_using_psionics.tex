\chapter{Using Psionics}
\label{chap:using_psionics}
\DndDropCapLine{T}{his chapter explains the nature of}
psionics, as well as the key rules behind
producing and maintaining psionic powers.
In \secref{sec:what_are_psionics},
a brief introduction to psionics is given
and their role in their game.
\secref{sec:manifesting_powers} explains how
psionic abilities---called powers---are produced
and the restrictions surrounding their use.
Lastly,
in \secref{sec:power_attributes},
the terminology used in power descriptions is explained.
The reader should familiarise themselves with each of these in turn
before diving into
the list of powers in \secref{chap:list_of_powers}.

\section{What are Psionics?}
\label{sec:what_are_psionics}
\define{Psionics}\index{psionics} tap into the latent ability of the mind
to master itself, its body and its environment.
Whereas spellcasters manipulate the magical essence
innate to the cosmological order,
users of psionics rely on their mind's power alone.
For many creatures,
using any psionic power,
let alone mastering a psionic discipline,
takes years, even decades of rigorous mental training.
That being said,
some creatures have a natural propensity for psionics,
for example the children of Gith, illithids and aboleths.

Many refer to psionics as \emph{the Old Way}.
This is because there is a school of thought that
the seeds of all creatures---the progenitor beings
which gave rise to existence itself---were
masters of the mind.
According to this doctrine,
magic and spellcraft were created by these
beings when the reality was moulded to as it is today.
To do this, they used psionics.

\subsection{Types of Psionic Abilities}
There are three types of psionic abilities:
\define{powers}\index{power}, \define{innate talents}\index{innate talent}
and \define{psi-like abilities}\index{psi-like ability}.

\subsubsection{Powers}
Powers are what most psionics users will be
interacting with.
The rules around powers are described
later in this chapter.

\subsubsection{Innate Talents}
A second type of psionic ability is an innate talent.
Innate talents are generally possessed by NPCs or
monsters which have some psionic ability.
They are also accessible if a character's race
has some degree of psionic heritage.
Innate talents are produced on a per day basis;
that is, a creature can use a certain innate talent
a number of times per day.
These are written as `\textit{x}/day innate talent'.
Innate talents can be used as a convenience,
representing an NPCs psionic ability but
foregoing the need to keep track of mental bandwidth,
psionic thresholds and psi dice.

To clarify,
an innate talent is linked to a particular power
in \secref{chap:list_of_powers}.
For example,
a creature might have
`\nameref{pwr:leap} (3/day innate talent)'.
The rules surrounding manifesting powers
as innate talents are the same as if
the power was manifested otherwise, except:
\begin{itemize}
    \item There is no mental bandwidth requirement;
    \item There is no psi dice cost;
    \item There is no psionic threshold requirement; and,
    \item An innate talent cannot be augmented unless
        otherwise specified.
\end{itemize}

\subsubsection{Psi-like Abilities}
Lastly, a psi-like ability is a psionic effect produced by an NPC or monster,
and is generally not accessible to player characters.
For example, a psi-like ability may be available as a specific
action in a creature's stat block.
Psi-like abilities are typically derived from a specific power,
but have minor alterations, as reflected in the stat block.

Like innate talents,
psi-like abilities have no mental bandwidth and psionic threshold requirement,
nor psi dice cost.
Like all types of psionic abilities,
psi-like abilities count as psionics for the purposes of, for example,
psionic resistance or vulnerability.

\begin{DndSidebar}[float=htbp]{Interaction with Magic}
    \tocside{Interaction with Magic}Since
    psionics and magic are disparate entities,
    how do they interact together?
    A creature that uses a psionic power is not
    casting a spell,
    and therefore any ability which negates the effects
    of magic cannot negate the power.
    For example, psionic powers cannot be affected by
    \spell{counterspell} or \spell{dispel magic}.

    However, certain psionic powers are explicitly stated
    to interact with magic.
    In addition, this supplement adds spells to the existing
    grimoire which explicitly interact with psionics.
\end{DndSidebar}

\section{Manifesting Powers}
\label{sec:manifesting_powers}
When a creature wills the effects of a psionic power
into existence,
they are said to be \define{manifesting}\index{manifest!-ing} that power.
In this context,
they are referred to as the \define{manifester}\index{manifest!-er}.
In order to manifest a psionic power,
the \define{power level}\index{power!level} must be equal to or below
the manifesting character's \define{psionic threshold}\index{psionic threshold}.
The psionic threshold is determined by the character's class.
If the power satisfies this test,
it can be manifested if they are able to
meet the power's \define{psi dice (PD)}\index{psi die (PD)} cost.
The manifesting character will also need to take note of their
\define{mental bandwidth (MB)}\index{mental bandwidth (MB)} and consider
whether or not manifesting the power will exceed this bandwidth.
Each of these concepts are explained below.

\subsection{Psionic Threshold}
The psionic threshold represents the strongest powers
that a psionics user can learn and manifest.
Powers have levels from 1 to 9 in the same way that spells do;
the higher the level, the more powerful the effects.
The psionic threshold is determined by a character's class
and level,
as shown in the class tables of \secref{chap:classes}.
When levelling,
a character can only learn powers with a level up to
their psionic threshold.
This is explained in more detail in the same chapter.

Each power has a level written in its description.
If a character attempts to manifest a power with level
greater than their psionic threshold,
the power fails to manifest,
however no cost is incurred in doing so,
see \secref{sub:psi_dice}.

\subsection{Mental Bandwidth}
Manifesting powers strains the mind and the body.
Mental bandwidth represents the amount of psionic strain that
a character can endure while using psionics,
and is determined by the character's psionic class and level,
see \secref{chap:classes}.
Mental bandwidth is divided into slots,
and all powers take up a different number of slots.

When a character manifests a power,
they note their total mental bandwidth
as determined by their class and level.
This is called the
\define{maximum mental bandwidth}.\index{mental bandwidth (MB)!maximum}
They then subtract from the maximum mental bandwidth
any slots which are taken up by powers they are already concentrating on,
see \secref{sub:concentration}.
The number determined after this subtraction is called the
\define{effective mental bandwidth}.\index{mental bandwidth (MB)!effective}
If the power to be manifested has a mental bandwidth
greater than the effective mental bandwidth,
the character must make a Constitution saving throw with
DC equal to 10 plus twice the number of slots
in excess of their effective bandwidth.
Note that it does not matter if the power to be manifested
does not require concentration---the power's mental bandwidth cost
still needs to be considered.

For example,
suppose Kadoth is a 10th level psion
and therefore has a maximum mental bandwidth of 6.
They are already concentrating on \nameref{pwr:adapt-body},
which takes up 4 mental bandwidth slots,
so their effective mental bandwidth is 2.
They then attempt to manifest the \nameref{pwr:time-warp} power,
which takes up 3 slots.
Since this puts them in excess of their effective bandwidth by 1,
they make a DC 12 Constitution saving throw ($10 + 2 \times 1)$.

\subsubsection{Becoming Strained}
If a character fails this Constitution saving throw,
the power fails to manifest
and the psi dice cost is still incurred.
In addition, the manifester gains the \textbf{strained}\index{strain} condition.

The severity and nature of the strained condition
depends on the number of slots
that were in excess of the effective bandwidth when the attempt
to manifest the power was made.
Each slot that was in excess counts as one level of strain.
The manifester then assigns each level of accrued strain
to \define{mind strain}\index{strain!mind} or \define{body strain}\index{strain!body},
which are shown in the tables below.
They may freely choose between either type of strain
for each level accrued.
The effects are cumulative,
such that a manifester with a body strain level of 3
incurs the effect from the 1st, 2nd and 3rd rows of the
body strain table.
Body strain and mind strain are also simultaneous;
if a manifester has any levels of mind strain and body strain,
they incur the associated penalties of each.
Lastly,
if a manifester already has any level of mind or body strain
after failing the Constitution saving throw,
the newly-acquired levels of strain are added to the current levels.

In Kadoth's example,
they were only in excess by 1.
They therefore assign themselves one level of strain,
and may choose either mind strain or body strain.
If they choose body strain and already have 1 level of body strain,
they would reach level 2 of body strain.

In the tables below,
\define{physical}\index{physical} checks and saving throws refer to those
that use Strength, Dexterity or Constitution as the
underlying modifier,
whereas \define{mental}\index{mental} checks and saving throws refer to those
that use Intelligence, Wisdom or Charisma as the underlying modifier. 

\begin{table}[htbp]%
    \begin{DndTable}[width=\columnwidth,
                     header=Body Strain]{
                     c X}
        Level & Effect \\
        1  & AC bonus from Dexterity reduced to 0 \\
        2  & Speed halved \\
        3  & Disadvantage on physical checks and saving throws \\
        4  & Hit point maximum halved \\
        5 & Death
    \end{DndTable}
\end{table}

\begin{table}[htbp]%
    \begin{DndTable}[width=\columnwidth,
                     header=Mind Strain]{
                     c X}
        Level & Effect \\
        1  & Can no longer take reactions \\
        2  & Can no longer take bonus actions \\
        3  & Disadvantage on mental checks and saving throws \\
        4  & MB halved (round down) \\
        5  & Death
    \end{DndTable}
\end{table}

\subsubsection{Recovering from Strain}
A character loses the strained condition,
including all of the accumulated levels of mind or body strain,
when they finish a long rest.

In addition, during a short rest,
a character may spend hit dice to remove
cumulative levels of mind or body strain on a 1 to 1 basis.
For example,
if a character was three slots in excess of their
effective mental bandwidth when they attempted
to manifest a power,
and as a result they have decided to
accrue three levels of solely body strain,
they can spend one hit die to remove the third level,
another to remove the second level,
and a final one to remove the first level.
If the manifester no longer has any levels of mind or body strain
after this process, the strained condition is removed altogether. 
A character cannot remove one level of strain
and roll to gain hit points with the same hit die.

\subsection{Psi Dice}
\label{sub:psi_dice}
When manifesting a power,
a character must expend a number of psi dice
equal to the number stated in the power's description.
The number of psi dice a character has is determined
by their class and level, see \secref{chap:classes}.
% A character also gains additional psi dice per level
% based on their psionics ability modifier (see below).
If a character has less psi dice than the total psi dice cost
of the power, the power fails (but no dice are expended).

The total base psi dice cost of a power is determined by its level.
The cost is equal to one less than double the power's level.
This is shown in the table below.
For example, a 1st level power incurs a cost of {\lvlone} psi die,
whereas a 5-th level power incurs a cost of {\lvlfive} psi dice.
\begin{table*}[htbp]%
    \begin{DndTable}[width=\textwidth,
                     header=Psi Dice Cost by Level]{
                     X X X X X X X X X X}
         Level         & 1 & 2 & 3 & 4 & 5 & 6  & 7  & 8  & 9 \\
        \textbf{Cost}  & \lvlone & \lvltwo & \lvlthree & \lvlfour & \lvlfive
                       & \lvlsix & \lvlseven & \lvleight & \lvlnine
    \end{DndTable}
\end{table*}

Some powers can be \define{augmented}\index{augment} with extra psi dice, in which
case they usually have a stronger effect, see \secref{sub:augmenting}.
If a power can be augmented,
this will be explicitly stated in its description.
The cost for augmenting a power is also given in its description.

\subsection{Concentration}
\label{sub:concentration}
If stipulated in the power description,
a power requires uninterrupted concentration
in order to be continually used.
There is no limit to the number of powers that may be
simultaneously concentrated on,
however each power concentrated on will take away from the
available mental bandwidth slots, as mentioned above.

Whenever a manifester takes damage while
concentrating on any number of powers,
they must make a Constitution saving throw.
The DC of the save is either equal to 10 or half the damage taken,
whichever is higher.
Each separate source of damage prompts a separate save.
If the save fails,
the manifester loses concentration on all the powers they were
concentrating on.
In addition, the DM may call the manifester to make a
save with any DC in a circumstance that would
jeopardise the manifester's concentration,
like being thrown off a flying broom or
being battered by an intense blizzard.

A manifester may at any time on their turn
stop concentrating on any power.
However, a manifester will automatically stop concentrating on a power
if they become incapacitated or if they die.

\begin{DndSidebar}[float=htbp]{Concentrating on Powers/Spells}
    \tocside{Concentrating on Powers/Spells}If a character
    can both cast spells and manifest powers,
    they may not simultaneously concentrate on a spell
    and any number of powers.
    Additionally, some spells, such as \spell{sleet storm},
    can specifically disrupt a spellcaster's concentration.
    A spell or effect which calls for a Constitution saving throw
    if a creature is concentrating on a spell
    also applies if a creature is concentrating on a power.
    The save DC is identical in either case,
    and the effects for losing concentration on powers
    are described above.
\end{DndSidebar}

\section{Power Attributes}
\label{sec:power_attributes}
In \secref{chap:list_of_powers},
the available powers are given.
In this section,
the terminology and structure of the power descriptions are explained.

\subsection{Level and Discipline}
Beneath the name of the power,
the power level and discipline are given.
Each power is associated with a discipline,
of which there are six.
These are described below.

\subsubsection{Prescience}
Strictly, the discipline of prescience relates to psionic powers
which allow the manifester to peer through time and see future events.
Prescient manifesters describe the future as
an undulating cloth blowing in a swift breeze,
with the hills and valleys of its surface representing
future possibilities. 

However,
the discipline of prescience incorporates additional tenets,
and its adherents are also characterised by an uncanny ability to
understand the past and know things not normally known to
the untrained mind.

\subsubsection{Prana Bindu}
The discipline of Prana Bindu stresses that the mind and body
must work in close concert with each other.
Adherents have a profound mastery of nerve and muscle,
and can react to situations with speeds nearly imperceptible
to the untrained eye.
The discipline also allows one to alter their internal equilibrium
and metabolism at will,
granting the body the ability to heal and adapt to hostile environments.

The discipline of Prana Bindu is especially favoured by psi knights,
powerful warriors who fuse martial and mental prowess.
See \secref{sec:psi_knight} for details on the psi knight class.

\subsubsection{Voice}
The \define{Voice}\index{Voice} is a dangerous but supremely powerful tool
used to dominate the minds and wills of others.
Adherents hone their speech in such a way that,
only by subtle changes in pitch, volume and speed,
their Voice can control the behaviour of others,
compelling them to act against their own will.

In addition, adherents to the discipline can manifest their Voice
in the minds of others, allowing them to telepathically
communicate with other creatures.

\subsubsection{Metacreativity}
The discipline of metacreativity allows the manifester to mould
matter as they desire,
turning the mundane into useful tools or deadly weapons.
A metacreator can also in this manner craft constructs
which bend to their mind,
allowing myriad servants to carry out their bidding.

\subsubsection{Spacefolding}
A spacefolder warps the fabric of space and time to
transport themselves and others.
This also allows them to move effortlessly in conditions
that would otherwise impede or slow down movement. 
Many powerful spacefolders can even cross the gap between
the planes themselves.

\subsubsection{Psychokinesis}
The discipline of psychokinesis allows one to transform
their raw mental energy into destructive power,
unleashing havoc on the battlefield.
Psychokineticists can deal terrifying amounts of damage
and leave their enemies trembling in awe and fear.

\subsection{Manifesting Time}
Each power has a \define{manifesting time}\index{manifesting time},
which may be a bonus action, action, or reaction.
Unlike spellcasting,
there is no level restriction to manifesting powers
on the same turn where one has been manifested
with a bonus action. 

In addition, some powers may have longer casting times,
in which case on each round the manifester must use their action
to manifest the power while maintaining concentration.
While the manifester is maintaining concentration in this way,
the power will take up the specified number
of mental bandwidth slots given in the power's description.
If their concentration is broken, the power fails,
but no psi dice are expended,
and the mental bandwidth slots are freed up.

\subsection{Range}
A power's \define{range}\index{range} indicates the extent
to which its effects can reach when manifested.
The \define{target}\index{target} (see below) of the power must be within range.

There are three types of ranges: self, touch and a range
specified in feet.
Powers with a range of self will always target the manifester.

Much like spells,
once a power is manifested,
its effects are't limited by range.
For example,
if a manifester is concentrating on a power,
and the target moves out of line of effect or out of
the range of the original power,
the power does not end and concentration is maintained.
\textit{Exception:} some powers explicitly state otherwise.

\subsubsection{Targets}
\label{subs:targets}
All powers have a target or targets.
A power's target is typically mentioned in the power description.
The description will also specify whether or not the target
of a power needs to be seen.
Thus, not all powers require \define{sight}\index{requirements!sight} of the target.

Additionally, some powers will
require the manifester to have \define{line of effect}\index{requirements!line of effect}
towards the target.
This means that the manifester needs to be be able to draw a line
to the target without that line passing through an obstacle,
even if the target can be seen through that obstacle.
Thus, a target cannot be behind total cover.
Powers which have the line of effect requirement
will explicitly state this in their descriptions.
Powers with a range of self or touch
will never require line of effect.

\subsection{Cost}
The \define{cost}\index{cost} specifies the mental bandwidth and psi die
that must be allocated and expended respectively to manifest the power.
For example,
`MB 1, PD 1'
means that the power takes up 1 slot of mental bandwidth
and costs one psi die to manifest.

\subsection{Duration}
The \define{duration}\index{duration} of a power is how long the effects
of a power remain active.
Powers can be instantaneous,
while others can last for any length of time
as specified in the powers description.

If the power requires concentration throughout its duration,
this will be indicated alongside the duration.

\subsection{Areas of Effect}
Some powers specify in their description an area of effect
over which the power manifests.
In this case, the rules governing these areas of effect
are identical to those of spellcasting, which begin at
phb 202.

\subsection{Affecting Targets}
Some powers specify that the manifester makes an attack
roll against each target,
whereas some specify that all targets need to make
a type of saving throw.

If the power specifies an attack roll,
then the manifester rolls a d20
and adds their \define{psionics attack modifier},\index{psionics!attack modifier}
which is their \define{psionics ability modifier}\index{psionics!ability modifier}
(determined by class)
plus their proficiency bonus.

If the power specifies a saving throw,
then each target must succeed on a saving throw
with DC equal to 8 + the manifester's psionic attack modifier.
This DC is called the \define{psionics save DC}\index{psionics!save DC}.
The type of saving throw
is specified in the power's description.

\subsection{Augmenting Powers}
\label{sub:augmenting}
Some powers will indicate that they can be augmented,
which is analogous to upcasting for a spellcaster.
When augmenting a power,
the manifester spends additional psi dice on top of the cost
incurred from the level of the power itself
(see \secref{sub:psi_dice} for the base cost by power level).

The cost required to augment a power will be specified
in the power's description (\define{augment cost})\index{augment!cost}.
For example, the description might mention,
`For every two additional psi dice you spend\dots'
In this case, the cost to augment the power is 2 psi dice.
Each time the power is augmented,
its level increases by 1.
In the example above,
spending an additional 6 psi dice increases the power's level by 3.

A power cannot be augmented to a level beyond the manifester's
psionic threshold.
For example,
if a manifester has a psionic threshold of 3,
they may spend 4 psi dice to augment a power from
1st to 3rd level with an augment cost of 2,
but may not go beyond 3rd level i.e. spend beyond 4 psi dice.

Some powers can be augmented in a multitude of ways,
each with different effects
and potentially different augment costs.
Every time you pay the augment cost,
the power's level is increased by one
irregardless of the type of augment you choose.

\subsection{Requirements}
The power's description will indicate additional
\define{requirements}\index{requirements}
that must be satisfied before the power can be manifested,
somewhat akin to the components in a spell's description.
However, unlike spells,
many powers are not associated with a verbal component.
When a power is manifested,
quite often there are scarcely any hints of its existence,
save for the actual effect or effects it produces.

Powers which have no additional requirements will state
`\textbf{Requirements:} None' in their descriptions.
Otherwise, the requirement(s) will be listed.
These include the requirement that the manifester can
see the target (Sight),
and that the target has line of effect towards the target
(Line of Effect).
Other requirements are described below,
and all possible requirements are summarised in the
table below.
\begin{table}[htbp]%
    \begin{DndTable}[width=\columnwidth,
                     header=Power Requirements]{
                     l X}
        Requirement & Description \\
        Sight & The manifester must be able to see the target(s) \\
        Line of Effect & The manifester must have line of effect
                            towards the target(s) \\
        Mundane Matter & The manifester must be touching mundane matter \\
        Speech & The manifester must be able to speak,
                    and the target(s) must be able to hear
                    and understand the manifester
                    unless otherwise stated \\
        None & No additional requirements apply for the power
    \end{DndTable}
\end{table}

\subsubsection{Voice Powers}
\label{subs:voice_powers}
Powers belonging to the Voice discipline sometimes involve
the manifester altering the frequencies, pitch and
speed of their natural voice, as well as their choice
of language, such that it compels
creatures to act in certain ways.
Whenever such a power makes reference to the manifester
using their Voice (e.g. `Your Voice does \textit{x}'),
the power is listed with the requirement \define{Speech}\index{requirements!speech}.
This means that
\begin{itemize}
    \item The manifester must be able to speak;
    \item The target(s) must be able to hear the manifester; and,
    \item The target(s) must be able to understand the manifester
\end{itemize}
in order for the power to have any effect.
If any of the latter two conditions are not met,
the power simply does not manifest as a valid target
cannot be specified.
If the speaking condition is not met,
the power fails but no psi dice are expended.
\textit{Exception:} some powers with the Speech requirement
will specify that any of the peripheral conditions listed above
do not apply.

The manifester's Voice is distinctly audible
up to the maximum range of the Voice power,
plus an additional 2d6 times 10 feet
(roll when the power is manifested).
Additionally, the use of the Voice is immediately obvious;
a creature that hears the Voice will be able to distinguish
it from ordinary speech,
even if they know not of the Voice or
the nature of the power which has been manifested.

If a power does not make reference to the manifester
using their Voice and thus lacks the Speech requirement,
this means that the manifester relies on their
\define{inner Voice}\index{Voice!inner}: the silent projection of their will
into the mind of another.
The manifester need not be able to be speak,
nor need the target(s) to hear and understand them
when manifesting this power.
The manifestation also does not produce a physical sound.

\subsubsection{Metacreativity Powers}
All powers of the Metacreativity discipline have the
\define{Mundane Matter}\index{requirements!mundane matter} requirement.
Mundane matter, which is moulded by the manifester,
refers to common substances on the
Material Plane,
like soil, plant matter, rock, metal,
glass, and so on.
The matter must be in the solid phase,
and not the liquid, gas or plasma phase.
It does not include matter imbued with magic
or some other arcane effect,
unless that matter is the raw essence of Limbo,
which can be warped with one's mind.

Thus,
if the manifester is not touching mundane matter
when they manifest a Metacreativity power,
the power fails (but no cost is incurred in the attempt).