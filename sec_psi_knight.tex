\section{Psi Knight}
\label{sec:psi_knight}
\DndSetThemeColor[DmgCoral]
With a thought alone, the silver-armoured half-orc effortlessly
shunts aside the first gnoll before plunging his longsword
into the abdomen of the next.
Then, with near-imperceptible speed, he appears behind the third gnoll,
who scarcely processes the gruesome end he meets.

In the silver-purple depths of the Astral Plane,
the githyanki astral skiff careens into the illithid nautiloid.
Before long, the erstwhile slaves of shattered Astromundi Nova
engage the mind flayers,
battling each other with thoughts alone.
Then the githyanki vanguard charges,
their silver swords slicing into the exposed flesh of their former
masters.

In the subterranean halls of a long-forgotten city,
the elf uses her mind to propel herself across a seemingly
bottomless crevasse.
Then, landing on the other side,
she shapes the very earth itself into a glowing handaxe,
which she picks up and throws at the awakened undead
bearing down upon her.

In the great open-air arena of Pax Illium,
the Prana Bindu champion bathes in the swelling roar
of the audience.
He fights with no weapon nor armour,
but with fist and mind alone.
Without any sign of strain,
he shoves a grappled gladiator to the floor
before delivering a swift kick into the abdomen of another,
careening them through through the electrified air.
The sound of their impact into the blood-soaked sand
is scarcely heard above the crowd's cheers.

\subsection{The Body and the Mind}
Psi Knights supplement their martial abilities
with the latent power of their mind to
achieve victory over their enemies.
Formidable opponents,
psi knights are rare and highly accomplished warriors
who reach an almost unparalleled balance between mind
and body.
Adventuring parties often readily accept a noble psi knight
by their side,
and the psi knight's enemies on the battlefield tremble
as they approach.

Like psions, psi knights manifest powers with myriad effects.
While they cannot manifest as many powers as psions,
they have substantial martial capacity
and have a greater ability to weather the tide of battle. 

\subsection{Creating a Psi Knight}
When creating a psi knight,
ask yourself why your character has chosen the path of psionics.
Perhaps they belonged to a small community of like-minded warriors
who understood the untapped potential of psionic powers.
Or perhaps they ventured down a solitary path,
testing the powers they learnt piecemeal in the din of battle.

Also ask yourself how they interact with the broader community.
Do they see themselves as an ordinary person with a fortuitous
talent?
Or do they feel that they rise above the masses?

\subsubsection{Quick Build}
You can make a psi knight quickly by following these suggestions.
First, Strength or Dexterity should be your highest ability score,
followed by Wisdom then Constitution.
Second, choose the soldier background.
Third, pick the \nameref{pwr:boots-of-the-demon-king} power
from the psi knight class power list.

\begin{DndSidebar}[float=htbp]{Multiclassing with the Psi Knight}
    \tocside{Multiclassing with the Psi Knight}In order to
    multiclass into the psi knight class,
    a character must have a minimum Strength or Dexterity score of 13
    and a minimum Wisdom score of 13
    (in addition to the usual score requirement for their base class).
    When you multiclass into the psi knight class,
    you gain proficiency with
    light armour, medium armour, shields, simple weapons
    and martial weapons.
\end{DndSidebar}

\begin{figure*}[t]
    \begin{ornamentedtabular}{c c >{\raggedright\arraybackslash}p{4.3cm} c c c c}[title={The Psi Knight}]
        \textbf{Level} & \textbf{P. Bonus} & \textbf{Features} & \textbf{Psi Dice} & \textbf{Mental Bandwidth} & \textbf{Powers Known} & \textbf{Psionic Threshold} \\
        1st  & +2 & Powers, Fighting Style, Weapon Mastery & 1   & 1  & 1  & 1 \\
        2nd  & +2 & Prana Bindu Initiate                   & 2   & 1  & 2  & 1 \\
        3rd  & +2 & Psionic Tradition                      & 3   & 1  & 3  & 1 \\
        4th  & +2 & Ability Score Improvement, Power Nap   & 4   & 2  & 4  & 2 \\
        5th  & +3 & Extra attack                           & 5   & 2  & 5  & 2 \\
        6th  & +3 & Psionic Tradition feature              & 7   & 2  & 6  & 2 \\
        7th  & +3 & Iron Will                              & 8   & 3  & 7  & 3 \\
        8th  & +3 & Ability Score Improvement              & 9   & 3  & 8  & 3 \\
        9th  & +4 & ---                                    & 11  & 3  & 9  & 3 \\
        10th & +4 & Psionic Tradition feature              & 12  & 3  & 10 & 4 \\
        11th & +4 & Calmness of Self                       & 13  & 4  & 11 & 4 \\
        12th & +4 & Ability Score Improvement              & 15  & 4  & 12 & 4 \\
        13th & +5 & ---                                    & 16  & 4  & 13 & 5 \\
        14th & +5 & ---                                    & 18  & 4  & 14 & 5 \\
        15th & +5 & Psionic Tradition feature              & 19  & 4  & 15 & 5 \\
        16th & +5 & Ability Score Improvement              & 21  & 5  & 16 & 6 \\
        17th & +6 & ---                                    & 22  & 5  & 17 & 6 \\
        18th & +6 & Psionic Tradition feature              & 24  & 5  & 18 & 6 \\
        19th & +6 & Ability Score Improvement              & 25  & 5  & 19 & 6 \\
        20th & +6 & Psionic Reserve                        & 27  & 5  & 20 & 6 \\
    \end{ornamentedtabular}
\end{figure*}

\subsection{Class Features}
As a psi knight, you gain the following features.

\subsubsection{Hit Points}

\begin{description}
    \item[Hit Dice:] 1d10 per psi knight level
    \item[Hit Points at 1st Level:] 10 + your Constitution modifier
    \item[Hit Points at Higher Levels:] 1d10 (or 6) +
        your Constitution modifier per psion level after 1st
\end{description}

\subsubsection{Proficiencies}

\begin{description}
    \item[Armour:] All armour, shields
    \item[Weapons:] Simple weapons, martial weapons
    \item[Tools:] None \vspace{4pt}
    \item[Saving Throws:] Constitution, Wisdom
    \item[Skills:] Choose two from Athletics, Acrobatics, Insight,
        Perception, Animal Handling, Intimidation, Survival
\end{description}

\subsubsection{Equipment}
You start with the following equipment,
in addition to the equipment granted by your background:
\begin{itemize}
    \item (a) chain mail or
          (b) leather armour, a longbow
            and 20 arrows
    \item (a) a martial weapon and a shield or
          (b) two martial weapons
    \item (a) a light crossbow and 20 bolts or
          (b) two handaxes
    \item (a) a dungeoneer's pack or
          (b) an explorer's pack
\end{itemize}

\subsection{Fighting Style}
You adopt a particular fighting style as your specialty.
Choose one of the following options.
You can't take a Fighting Style option more than once,
even if you later get to choose again.

\subsubsection{Blind Fighting}
You have blindsight with a range of 10 feet.
Within that range, you can effectively see anything that
isn't behind total cover,
even if you're blinded or in darkness.
Moreover, you can see an invisible creature within that range,
unless the creature successfully hides from you.

\subsubsection{Defence}
While you are wearing armour, you gain a +1 bonus to AC.

\subsubsection{Dueling}
When you are wielding a melee weapon in one hand
and no other weapons,
you gain a +2 bonus to damage rolls with that weapon.

\subsubsection{Great Weapon Fighting}
When you roll a 1 or 2 on a damage die for an attack you make
with a melee weapon that you are wielding with two hands,
you can reroll the die and must use the new roll,
even if the new roll is a 1 or a 2.
The weapon must have the two-handed or versatile property
for you to gain this benefit.

\subsubsection{Interception}
When a creature you can see hits a target,
other than you,
within 5 feet of you with an attack,
you can use your reaction to reduce the damage the target takes
by 1d10 + your proficiency bonus (to a minimum of 0 damage).
You must be wielding a shield or a simple or martial weapon
to use this reaction.

\subsubsection{Protection}
When a creature you can see attacks a target other than you
that is within 5 feet of you,
you can use your reaction to impose disadvantage on the attack roll.
You must be wielding a shield.

\subsection{Powers}
\subsubsection{Psi Dice}
The number of powers that you can manifest
is limited by the number of psi dice that
you have available.
This is the principal limit on the psionic output of a psi knight.
For example, a 9-th level psi knight with {\pklvlnine} psi dice
can manifest a power costing 1 psi dice {\pklvlnine} times.

The number of available psi dice per level
is shown in the psion class table.
You regain all expended psi dice when you finish
a long rest.

\subsubsection{Mental Bandwidth}
The powers you can manifest without straining yourself
is reflected by your mental bandwidth.
The total number of mental bandwidth slots you have
is shown in the class table.

\subsubsection{Powers Known}
At 1st level,
you choose one psion power of your choice
from the psi knight class power list.
When you gain a psi knight level,
you can learn additional powers from the class list
on top of the ones you already know.
The total numbers of powers you know cannot
exceed your powers known value,
as shown in the class table.

The powers you have selected in this fashion have been committed
to memory, and you need not prepare a selection of them each long rest.
Also,
whenever you take a long rest,
you may substitute any number of powers you already know
for ones you do not know on the psi knight class list.
The power that has been replaced counts as no longer being known.

\subparagraph{Psionic Threshold}
You cannot learn nor manifest powers with a level
greater than your psionic threshold,
shown in the class table.
As mentioned in \secref{sub:augmenting},
you also cannot augment powers to a level
beyond your psionic threshold.

\subsubsection{Manifesting Ability}
Your psionic ability modifier is your Wisdom modifier.
The save DC against your psionic powers and your
psionics attack modifier are therefore respectively:
\small\begin{equation*}
    \begin{gathered}
        \text{\textbf{Psionics save DC}}
            = 8 + \text{your proficiency bonus} + \\
                  \text{your Wisdom modifier} \\
        \text{\textbf{Psionics attack modifier}}
            = \text{your proficiency bonus} + \\
              \text{your Wisdom modifier}
    \end{gathered}
\end{equation*}\normalsize

\subsection{Weapon Mastery}
Starting at 1st level,
your training with weapons allows you to use the mastery properties
of two kinds of weapons of your choice with which you have proficiency,
such as longswords and handaxes.

Whenever you finish a long rest,
you can change the kinds of weapons you chose.
For example, you could switch to using the mastery properties of
halberds and flails.

\subsection{Prana Bindu Initiate}
Starting at 2nd level,
your mind-body training makes manifesting powers
belonging to the Prana Bindu discipline
vastly easier.
Prana Bindu powers require one less mental bandwidth slot
than mentioned in their description (minimum of 0)
and cost 1 less psi die (minimum of 0).
If the psi dice cost is reduced to zero in this manner,
it is called a \define{0PD power}\index{power!0PD}.
You can only manifest a number of 0PD powers
equal to your proficiency bonus per long rest.
Each additional 0PD power you manifest
beyond this limit costs the usual 1 psi die.

\subsection{Psionic Tradition}
When you reach 3rd level,
you embark on a journey down a particular
psionic tradition.
The tradition represents the aspects of
your mind-body mastery that you have decided
to focus on.
At this stage, two traditions are offered:
the Weirding Way, which focuses on your mastery
of devastatingly precise and supremely fast
Prana Bindu techniques,
and the Commander,
who couples their experience in battle
with profound psionic powers to meticulously
plan and execute combat stratagems.
Your tradition grants you features at 3rd level,
and again at 6th, 10th, 15th and 18th level.

\subsection{Ability Score Improvement}
When you reach 4th level,
and again at 8th, 12th, 16th and 19th level,
you can increase one ability score of your choice by 2,
or you can increase two of your ability scores by 1.
As normal,
you can't increase an ability score above 20 using this feature.

\subsection{Power Nap}
Starting at 4th level,
you may enter a state of meditation
in which you refresh your mind.
The meditation lasts 1 minute and
you must maintain concentration throughout.
If you do so,
at the end of the meditation you may regain a number of psi dice
equal to your Wisdom modifier multiplied by your psionic threshold.
You cannot accumulate psi dice past the maximum number you have available
as per your class(es) and level.

Once you have used this feature,
you cannot use it again until you have finished a long rest.

\subsection{Extra Attack}
Beginning at 5th level, you can attack twice,
instead of once,
whenever you take the Attack action on your turn.

\subsection{Iron Will}
Starting at 7th level,
your mind is able to override your body at times
when the untrained would falter.
Whenever you are reduced to hit points,
but not killed outright,
you can spend a psi die to immediately regain hit points
equal to your Wisdom modifier plus the number of levels
you have in the psi knight class.
Once you have used this feature,
you cannot use it again until you finish a long rest.

\subsection{Calmness of Self}
Starting at 11th level,
your unfettered desire to improve mind and body
has granted you a sense of inner calm.
You have resistance to psychic damage.
Moreover, if you start your turn charmed or frightened,
you can expend 3 psi dice and end every effect on yourself
subjecting you to those conditions.
In addition, you have advantage on saving throws to resist
effects from Voice psionic powers.

\subsection{Psionic Reserve}
Starting at 20th level,
when you roll for initiative and have no psi dice remaining,
you regain 8 psi dice.

\subsection{Weirding Way}
The Weirding Way champions incredible control over one's nerves
and reaction times---granted by Prana Bindu techniques---to
manoeuvre and strike at opponents before they can retaliate.
The name `Weirding Way' comes from the fact that,
to untrained opponents, the movements of those versed in
the Weirding Way appear unnatural and superhuman,
almost as if they are teleporting across the battlefield.

\subsubsection{Prana Bindu Prodigy}
Starting at 3rd level,
you can choose to learn any Prana Bindu tenet from
the psion discipline at \secref{sub:prana_bindu_tenets}.
The level requirements stated there do not apply to you.

\subsubsection{Seize the Initiative}
Starting at 6th level,
whenever a fight breaks out,
you can reliably act before your enemies.
If you spend a psi die,
you can grant yourself advantage on an initiative roll.

\subsubsection{CQC Dominance}
Starting at 10th level,
you make quick use of off-hand and blunt strikes to
achieve dominance in CQC (`close quarters combat').
You have advantage on all checks when taking the
grapple, shove, shove aside, or overrun action.

In addition,
whenever a creature within 5 feet of you ends their turn,
you can use your reaction and spend a psi die
to make a single attack against that creature.
The damage type is replaced with bludgeoning damage,
since you use your off-hand,
a kick, a punch or the blunt end of your weapon
to make the attack.
Whether or not the attack is made with a weapon
or is an unarmed attack is your choice.

\subsubsection{Slippery Movement}
Starting at 15th level,
your incredible agility makes its difficult for your
enemies to pin you down.
You have advantage on all saving throws against effects
which would slow your movement or reduce it to 0.
In addition,
if you become restrained or grappled,
you can use a bonus action on your turn
to attempt to repeat the save
or contest respectively which caused that condition.
You have advantage on this save.

\subsubsection{Lisan al Gaib}
Starting at 18th level,
your perfection of Prana Bindu methods becomes legendary,
and in the eyes of many your are considered a
sublime master of the body and mind.
You may pick an additional Prana Bindu tenet from
\secref{sub:prana_bindu_tenets}.

\subsection{Commander}
The Commander uses their experience and psionic abilities
to control and manipulate the battlefield as they desire.
Enemies fear a psi knight commander,
for they know that they will struggle to ever
gain tactical advantage over them and their allies.

\subsubsection{Wise Strikes}
Starting at 3rd level,
your battlefield experience ensures that your attacks
are as damaging as they can be for your enemy.
You may add your Wisdom modifier to all damage
rolls for melee attacks. 

\subsubsection{Knowledge is Power}
Starting at 6th level,
once per day as a bonus action,
you choose a hostile creature you can see within
30 feet of you.
Your mind links with them,
and you gain a deep understanding of their
strengths, weaknesses and peculiarities.
You may then ask the DM the exact amount of any of the following
characteristics of the hostile creature:
\begin{itemize}
    \item Its full stat array;
    \item Its AC;
    \item Its total hit points;
    \item Its to-hit modifier on one type of attack; or,
    \item Its damage roll, including modifiers, for one type
            of attack.
\end{itemize}

\subsubsection{Order of Battle}
Starting at 10th level,
your mind's eye expands,
and you begin to see the battlefield from a top-down perspective.
Enemies and allies are pieces on a board, which you can manipulate
with thought alone.
Once per day,
immediately after initiative is rolled at the start of an encounter,
you may attempt to declare the initiative order for
any number of combatants.
For each creature you intend to affect,
they must make a Wisdom saving throw against your psionics save DC.
If they fail, you may replace their initiative with any number.
If they succeed, their initiative is unaffected. 

This ability does not apply when you are surprised.
You also cannot alter the initiative order of creatures
in the encounter that you have no knowledge of,
those that join later in the fight (such as those
that are summoned),
and environmental effects or lair actions
which act on a set initiative count.

\subsubsection{Gambit}
Starting at 15th level,
your see fighters on a battlefield as pieces on a chessboard
that can be effortlessly manipulated---you simply turn your
mind to them, and they bend to your will.
You have a pool of 120 feet of movement.
On your turn,
you may use your action to attempt to move any number of creatures
cumulatively up to 120 feet in any direction,
including vertically.
Each creature you wish to move can resist this movement by
succeeding a Wisdom saving throw with DC equal to your
psionics save DC.

Any unspent movement remaining in your pool persists
and can be carried over to subsequent turns.
Your pool of movement replenishes whenever you finish a long rest.

\subsubsection{Only a Simulation}
Sometimes the tide of battle can be unpredictable.
Fortunately for you,
starting at 18th level,
your supreme mental abilities confer you the ability to
run combat simulations viewed internally by your mind's eye.
These simulations allow you to decide the best course of action.

Once per day, on your turn,
you can use an action to undo the previous round of combat,
allowing you to `replay' that round.
This ability reverses time to the point in the past
just prior to your previous turn,
undoing the effects of everyone else's actions in the meantime.
Once you have used Only a Simulation,
only you retain knowledge of what happened during the round
that is being replayed; after all, it was just a simulation.
However, you can communicate that knowledge verbally to your companions,
if desired,
and you can act on that privileged knowledge
in any way you desire.

\subsection{Soulwarrior}
The Soulwarrior is inseparable from the Soulweapon:
an otherwise mundane weapon rendered extraordinary through the psionic talent of its wielder.
With the Soulweapon as their psionic nexus,
the Soulwarrior adapts to the split-second changes of the battlefield,
alternating through different Forms to press their advantage.
Enemies thus learn to fear the Soulwarrior's unshakable bond between mind and weapon.

\subsubsection{Soulweapon}
When you reach 3rd level,
you designate a weapon currently in your possession as your \define{Soulweapon}.
The type of the weapon must belong on the weapons table,
but can be magical or psionic in nature.
You can only have one Soulweapon at a time.
In order to designate a weapon in such a way,
you must spend 1 minute maintaining concentration with the weapon in your hand.
Once this bond has been formed, 
you can break it by again concentrating on the weapon for 1 minute while holding it.

Your Soulweapon has a handful of unique properties, as described below.

\subparagraph{Disarm Resistance}
Whenever you would be disarmed of your Soulweapon,
you can spend one psi die to maintain possession of the weapon.

\subparagraph{Psionic Propulsion}
Your Soulweapon gains the Thrown (Range 20/60) property if it does not already have it.
If it already has this property but with a longer range, use that version instead.

\subparagraph{Recall}
While your Soulweapon is within 120 ft. of you,
you may use a bonus action to immediately recall the weapon to your hand,
in which case it takes the shortest possible unobstructed path.
While the weapon is moving in this way,
you may make a single attack with the weapon against any creature
whose space intersects the path of the weapon.
If another creature is holding the weapon when you try to move it,
they must succeed on a Strength saving throw with DC equal to your psionics save DC
or have the weapon forced from their hands.

\subsubsection{Forms of Combat I}
Starting at 3rd level,
you gain access to a single combat \define{Form}.
A Form is a type of combat stance which grants a \define{passive} and \define{finisher} effect.
The passive effect is always realised while you are \define{adopting} the Form,
while the finisher effect is typically triggered by use of an action.
You may use a bonus action to adopt a Form at any time,
as long as your Soulweapon is in your hands.
There are a number of ways in which you \define{release} the Form:
\begin{itemize}
    \item By using the finisher effect of your current Form;
    \item By spending another bonus action;
    \item By becoming incapacitated or dying; or,
    \item By losing grip of your Soulweapon.
\end{itemize}
In any case, whenever you release a Form, the passive and finisher effects cease.
If you have released a Form by using the finisher effect,
you cannot adopt the same Form again until you finish a long rest.

When you reach higher levels,
you learn more combat forms;
otherwise, at this level, you may only prepare one Form for use.
You can change the Form (and later Forms) you have prepared
whenever you finish a long rest.
The list of all Forms is shown in the table below.
\begin{table*}[htbp]%
    \begin{DndTable}[
        width=\textwidth,
        header=Forms
    ]{l X X}
        Form           & Passive                        & Finisher                                    \\
        Heat Haze Fury & Your weapon attacks have
                         advantage, but attacks made
                         against you have advantage.    & As part of one of your attacks when you
                                                          take the attack action, your weapon explodes
                                                          with fiery energy. The target of the attack
                                                          takes additional fire damage equal to
                                                          the weapon die or dice if the attack lands.
                                                          If not, the form remains unreleased and
                                                          you may attempt then finisher again.        \\
        Wind Within    & Subtract double your proficiency
                         bonus from the total damage
                         you take from an attack against
                         you as well as the total damage
                         you deal with an attack.       & As an action, you rapid movements create
                                                          a forceful current that forces creatues back.
                                                          Each creature within 20 feet of you must
                                                          succeed on a Strength saving throw against
                                                          your psionics save DC or be pushed back 15
                                                          feet and knocked prone.                       \\
        Stalwart Heart & You cannot make weapon attacks
                         or unarmed strikes, but your
                         psionic attacks are made with
                         advantage and the MB cost of
                         powers you manifest is reduced
                         to zero.                       & As an action, you recover the PD expended for
                                                          the last power you manifested.                \\
        Peerless Poise & Any ranged attacks that hit
                         you are instead deflected
                         back to the attacker,
                         hitting them instead,
                         but melee attacks made
                         against you always hit.        & As an action, you unleash a duplicate of the
                                                          last ranged attack that hit you. Make a
                                                          ranged psionics attack against a single target.
                                                          The damage dealt on a hit is the same as the
                                                          original attack.
                                                          Any restrictions on the original attack,
                                                          like range and line of effect,
                                                          also apply for this attack.                    \\
        Emotionless
        Retribution    & You cannot manifest powers.
                         However, whenever a creature misses
                         you with a melee attack,
                         you can make a ranged psionics
                         attack against the creature.
                         You do not have disadvantage
                         if you are within 5 feet of
                         the enemy. If the attack hits,
                         the target is pushed back
                         15 feet with a
                         telekinetic thrust.            & As an action, you assault a single creature
                                                          with a barrage of loose material.
                                                          Choose a 5-foot cube of mundane matter.
                                                          This cube hurtles towards a target,
                                                          who must make a Dexterity saving throw
                                                          against you psionics save DC.
                                                          They take $n$d8 bludgeoning damage on
                                                          a failure, where $n$ is your
                                                          proficiency bonus, or half as much
                                                          on a success.                                 \\
    \end{DndTable}
\end{table*}
% When an enemy hits you with a
% weapon attack
% or unarmed strike,
% you may make a
% Dexterity (Acrobatics) or
% Strength (Athletics) check
% with DC equal to the
% attacker's attack roll.
% If you succeed, you tumble
% out of harm's way or deflect
% the strike,
% causing the attack to miss.

\subsubsection{Forms of Combat II}
Starting at 6th level,
you gain access to a second combat Form.

\subsubsection{Forms of Combat III}
Starting at 10th level,
you gain access to a third combat Form.

\subsubsection{Forms of Combat IV}
Starting at 15th level,
you gain access to a fourth combat Form.

\subsubsection{Forms of Combat V}
Starting at 18th level,
you gain access to your final (fifth) combat Form.